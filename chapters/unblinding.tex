\chapter{Un-Blinding Us With Science?}
\chaptermark{Un-Blinding Us}
\textit{Wednesday, 27 January 2010}
\bigskip

I have been known to aver that running is a very simple sport. And I remain convinced that, notwithstanding the odd optional gadget, it is. In my post from a couple of weeks ago, I argued that running's relative simplicity had left it more or less untouched by technological change, at least when compared with other sports. But this raised a broader question concerning the role of science-based knowledge in the sport of running: What has science actually taught us about running; what, in other words do we know about running based on sound scientific research, as opposed, say, to less formalized modes of knowledge production? One early researcher in the then emerging field of exercise physiology was known to have offered that object of his investigations would be, to paraphrase: To discover the scientific basis of what the best coaches already knew based on trial and error; in other words, to verify a body of knowledge that had been accumulated and transmitted by means of actual practice rather than through the formal application of scientific method. This may have been a bit of false modesty on the part of the researcher-- surely he also believed he would discover new principles that would perhaps contradict, but at the very least improve on, what the best coaches knew, or thought they knew. The statement did, however, represent a clear acknowledgment of the centrality of experiential knowledge in the sport of running-- and on the part of a professional scientist, no less. Now more than 40 years on, we might wonder about the actual fruits of this still very fledgling but fast-growing field of research: Has exercise science only confirmed truths known by the best coaches?; Or, has it perhaps corrected misunderstandings and misconceptions embedded in everyday coaching practice, and even uncovered wholly new principles and the means for exploiting them in pursuit of better performance?

But first, a qualification: I have no formal scientific training myself (natural scientific training, that is; I have an unfortunate surplus of social scientific training!) I have, however, tried to make a point of keeping abreast of the science of racing and training as best I can. If you'd like to read about the science of running and exercise from an actual scientist-- and one with a high-level running background to boot-- check out journalist Alex Hutchinson's informative and highly readable blog Sweat Science

To begin, we might consider the state of knowledge about racing and training in the days before the advent of what would become today's "exercise science". What, exactly, was the content of the informal knowledge that the above mentioned researcher aimed to substantiate?

The first, most basic, and ultimately most consequential practical discovery in sports like running-- i.e. sports based on the testing of basic physiological limits-- was simply that of the "overcompensation" principle. Now utterly commonsensical, the idea that a basic physiological system-- e.g. a pattern of muscular contraction, short term and explosive or long term and continuous-- could be induced by means of systematic repetition, sometimes to exhaustion, to become more rather than less adequate to the original challenge (i.e. stronger rather than tired out and weaker), had first to be discovered through actual practice. This discovery would become the very basis of what we now refer to as "training" in simple strength and endurance sports (i.e. sports with a relatively small technical skill dimension). Before the discovery that exposing the organism to systematic stress could actually make it more rather than less capable, coaches and athletes operated according to the theory of energy conservation; or, the idea that the body possessed a finite amount of vital energy that must be carefully preserved and marshaled in order to be powerfully released on the field of play. (For a brief and illuminating discussion of the discovery of "training", see Beamish and Ritchie Fastest, Highest, Strongest [2006]).

"Training" based on the overcompensation principle supplanted the older practice of "energy conservation" when athletes who exposed their bodies to controlled stress and recovery regimes began to seriously out-compete those who didn't (I leave it to Beamish and Ritchie to explain the social and political forces driving the new interest in winning that led to the discovery of this paradigm-shifting physiological principle.) But, in running in particular, coaches would very shortly thereafter begin a process of refinement of the basic principle of systematic "training" that would last up until the advent of exercise science in the late post-WWII period.

From the 1920s until the early post-WWII period, most runners trained by directly replicating the demands of their racing distances-- i.e. by running repeated bouts at or faster than their goal race pace, sometimes up to 5 times a week. (Today's runners might try imagining what it would feel like to run 10-20x400m at mile/1500m race pace five days in a row!) Far more effective than simply conserving energy for its cathartic release on race day, this kind of training nevertheless had it limits. As one can imagine, it was very psychologically stressful. For this reason, and because it probably over-stressed certain basic adaptive processes (but who really knew in scientific terms?), athletes often failed to improve after a couple of years on this kind of regime.

As a result, this kind of intense, daily, race-specific training would eventually give way to the practice of "periodization", in which runners trained at different speeds and over different terrains, depending on the time of year (which, as with the invention of "fartlek" and hill training, was also a concession to the vagaries of climate, geography and scarce resourses-- namely, the absence of groomed running tracks). The older method would survive in a limited way in contexts where facilities existed, and, more specifically, where rapid, short term gains were sought-- for instance, the U.S. college system and, to some extent the North American high school system which fed it. But, by the 1970s, most runners trained according to the principle of periodization. The most celebrated developer and proponent of this variegated approach would become the now legendary New Zealand coach Arthur Lydiard, famed for his uncanny success rate in turning athletes found ready-to-hand, some in his own neighborhood(!), into Olympic medalists and world record holders.

The periodization approach would lead to the development of the basic distinction between so-called "aerobic" and "anaerobic" running (really, just longer, slower running versus shorter, faster running). Out of this basic distinction would arise the techniques of the meduim length, intermediate paced run (the "tempo" or "aerobic threshold" run) and the Farlek session (see my November 17th post). And Lydiard himself would also place special importance on uphill running and "bounding" (a product of the particular environment of New Zealand, no doubt). Over time, coaches from every continent would create their own variations on the "periodized" training system, but it would remain the same in its essence up until the advent of "exercise physiology"-- and, many would argue, beyond to the present. And, while there had been rudimentary scientific investigations into the physiological mainsprings of the system (many in the context of USA-USSR Cold War military/scientific rivalry) "periodized" training remained the rather pristine product of simple trial and error on the part of coaches and athletes in the field. In the 40 or 50 years since the advent of systematic training, athletes had repeatedly opened up vast new frontiers of performance strictly on the basis of intuition, casual experimentation, and the informal dissemination of best practice within a remarkably open community of enthusiasts and competitors (hostile political establishments notwithstanding). (For an interesting glimpse into this world, see Bob Phillips short biography of Czechoslovakian distance running legend Emile Zatopek entitled Za-to-pek!).

It was precisely this "best practice" that the above-mentioned early exercise physiologist set out to investigate by means of the scientific method and modern diagnostic technology. To proceed directly to the question at hand, what can we say have been fruits of his and others' investigations over the past 30-40 years? What has been confirmed, what falsified, and what, perhaps, newly discovered? The record, I would suggests, is mixed, and the precise value of exercise physiology-- or more broadly now, "exercise science"-- for the practice of run-training remains questionable.

Of the contributions that science has made to our understanding of how to improve distance running performance, I would include the following five in the category of "unquestionable":


\begin{enumerate}
    \item The simple confirmation that distance running proficiency is rooted in "aerobic capacity". Lab analysis confirming that distance runners typically have higher maximum volume of oxygen per kilogram uptake capacities (MV02) than non-runners, and that the best runners tend to have greater capacities still, clearly established the physiological basis of the sport, and pointed the way towards future improvements. And the discovery through longitudinal studies that MV02 could be improved through run-training only confirmed what the best athletes and coaches already new. Today, there are debates about the role of so-called "running economy" (the speed of a runner relative to his/her MVO2, which is an aspect of an athletes overall "aerobic power") and how to improve it, but the broader role of MV02, or aerobic capacity, in explaining distance running success is now beyond question.
        
    \item The discovery of the role of the spectrum of muscle fibre composition (so-called slow and fast-twitch fibers) in determining relative success in running events of different distances. "Muscle fiber" theory revealed that there were probably immutable physiological determinants of running success at the different extremes of the running distance spectrum; that, notwithstanding some cross-trainability of these different fibres, distance runners and sprinters were likely born into their respective event groups. Scientific confirmation of this basic reality, while it has been more useful at the extremes than at the margins of the spectrum, has helped to inform the event choice, and even the specific training, of thousands of athletes.

    \item Clarification of the physiological effects of training at the cellular level. Basic research in exercise physiology has gone a long way in specifying what happens to our muscle and blood cells when we train for running at different intensities. The simple discovery that our muscles, including our heart, become stronger and better able to store and make use of different energy sources, and that our blood volumes increase over time in response to our muscles' training-induced demand for more oxygen to ignite energy metabolism, has been a powerful support for the idea that training improves performance, and that it can likely do so over many years. Now, athletes have a scientific basis for continuing to pursue their sport competitively for years beyond what would have been considered "peak age" 50 or 60 years ago. One result is that we have now seen runners in their mid and late 30s win Olympic medals in running events where precisely these kinds of long term training adaptations would seem to be most relevant-- the 10,000m and Marathon.
        
    \item The theory of "lactic acid". The discovery that exercising muscles produce this substance the nearer the point of failure they reach-- even though the reasons why they produce it, and even its role in muscle physiology, have recently been shown to be unclear-- represents a breakthrough that has enabled many coaches to more carefully regulate the intensity of training on a day-to-day basis. For whatever the reason, and to whatever longer term effect, the presence of high levels of lactate in an athlete's body is a clear marker of training stress. And while the measurement of blood lactate levels is still a somewhat expensive and invasive procedure, it is now within the grasp of at least the club-level athlete, if not the recreational age-group athlete.
        
    \item The discovery of the role of dietary carbohydrates in the physiology of distance running. While knowledge about the optimal ratios of carbohydrates to fats and protein in the diets of runners continues to be revised, all runners are now aware of the primacy of carbohydrates in fueling performance in endurance events, putting paid to the "steak and eggs breakfast" theory of yesteryear!
\end{enumerate}

Beyond these five very broad contributions-- which vary in terms of the extent to which they add to, supplant, or only confirm elements of already existing practical knowledge in the field-- we find a welter of narrower discoveries and claims (many of which are ably documented and vetted in Alex Hutchinson's blog referenced above).

Among these are studies that confirm the performance enhancing properties of various drugs not intended for such use, and form the basis for their banning. In these instances, the contribution of science to the sport is unquestionable. Other recent studies have examined things like the effectiveness of long established conventional training modalities such as static stretching and post-workout massage (both found to be ineffective in reducing injury, and perhaps even counter-productive), and the benefits of barefoot versus shod running (still very inconclusive). With more funding for this kind of targeted research, we can no doubt expect to see conventional coaching wisdom turned on its head. We can also no doubt expect to see more scientific validation for emerging and established training practices, such as various kinds of strength and flexibility training.

Notwithstanding the light that science has been able to shine on the accumulated practical knowledge of run-training-- and, I would predict, in spite of future scientific discoveries-- running looks set to remain the very technically simple (if practically VERY difficult!) sport it has always been. If it manages to retain what's left of its sporting integrity in the face of the inevitable spate of new doping technologies (with "gene-doping" being by far the most threatening), running is likely to remain the very simple challenge of athlete vs. himself and the elements that it has always been, even at the highest levels. Our sport is, after all, the only one besides soccer in which athletes from some of the poorest nations on earth compete on more than even terms with athletes from the richest nations. If science and technology had ever played more than a secondary role in running, this could never have happened. Moreover, runners should take great heart in the simplicity of their sport. Its homely charm is what preserves it as a respite in a world increasingly dominated by technological inter-mediation. We can likely rest assured that machines and chemicals will never fully colonize it. It is likely to persist in its more or less pure form (buffet belts notwithstanding!) as long as we're able to preserve congenial, natural spaces in which to do it; which is the greater challenge of technology faced by all the world.
