\chapter{Risking Failure: Racing and Becoming}
\chaptermark{Racing and Becoming}
\textit{Monday, 8 June 2009}
\bigskip

It is the end of the spring racing season and I am once again struck by the realization that our best laid plans as coaches and athletes end in failure (in an objective sense) just as often, and perhaps more so, than in triumph. Why, then, do we so often persist? Why do we set about planning another cycle of hard training while still waist deep in the ashes of our earlier construction? Are we simply deluded, or is there perhaps a deeper reason for this tendency to look constantly forward even-- or perhaps especially-- in the throes of despondency over our recent failures?

There are many short answers to this question: hope springs eternal in the human heart, and all of that; we learn and grow from our mistakes, which are only fully apparent in failure; some of us love the process of planning and preparing as much as the final execution, etc. All of these short answers, however, only hint at what I believe to the deeper, bigger reason for our persistence, and this answer speaks squarely to what I call the culture-creating possibilities inherent in competitive sport: we persist in the face of failure, I would argue, because to do so makes us more fully human. I would go even further, in fact, and say that we persist until we encounter failure-- we, in a sense, pursue failure-- because in doing so we realize our humanity more fully and completely. We are genuinely hurt by defeat and failure-- if we weren't it would mean that we did not really care; but, in risking it, and ultimately encountering and overcoming it, we realize our distinctness as a meaning-creating species.

When embarking on a training plan with a clear set of competitive goals in mind, we are always intuitively aware of the potential for failure. We do not operate in ignorance of this risk; we actually embrace it by setting our goals beyond anything we have achieved before, and by allowing ourselves to deeply desire those goals. This desiring of goals and concomitant courting of failure is what charges what is, in an inherent sense, a meaningless activity with meaning. Clearly, human beings have always run for some mundane, practically useful purpose-- from catching their food or escaping becoming food, to catching the bus-- and many still do run for purely external purposes-- namely, to "stay in shape" or to prolong life. Running is lifted out of this realm of practical necessity to become a form of culture-- a vehicle for the self-expression and self-actualization-- however, when it is approached as an end in itself, as it is when we do it simply to enjoy the feeling of movement (as children often do), or when we attempt to do it longer and faster than we have before, as we do when we train to compete. It is only in this latter instance, however, where we attempt to become more than we currently are, that running takes its full place alongside of other "cultural" activities. (Now, as we well know, the pursuit of running as a competitive sport can sometimes turn in back into an utilitarian activity, such as when it is enlisted in the pursuit of financial or political gain, or when the competition is taken over by scientists looking for gains by intervening directly at the cellular level; but, that is a subject for another installment.)

What, then, does the experience of failure among competitive runners tell us about the cultural possibilities of the sport? In short, failure is integral to the act of establishing goals and the investment of emotional energy and meaning in the pursuit of those goals-- the very things that open us up to real disappointment and even despair. And, in a way, failure awaits everyone who continues to set goals. Furthermore, it is only in the moment of failure that the cultural possibilities of competitive running are fully revealed. Failure reveals to us how much of ourselves we invested in the process and how completely we offered up our hearts and spirits for breaking. In failure, we also confront our limitations and learn to better accept ourselves as we are, even as we strive to become a little more. Failure, if understood as integral to the process of striving to overcome, can make us deeper, wiser, more interesting, and thus more fully human. This is true, of course, of all meaningful human pursuits. In competitive running, however, failure tends to be uniquely clear, unambiguous and undeniable, as much as we might try to qualify it or soften it in the first instance (when you fail in running, they document it on a list!). When a collectivity fails the reasons are often very complex and the ultimate responsibility broadly diffused. Likewise when individual failure occurs in a more technologically mediated activity, such as in sports like cycling or motor racing, it can be more easily displaced onto others, such as technicians and equipment makers. When runners fail, on the other hand, their failure feels-- and, in a sense, really is-- uniquely their own, and the ensuing sense of weakness and inability is that much more acute. This deeper sense of personal failure and inability, however, only increases the cultural potential of competitive running-- its potential to encourage human growth and becoming.

These, in any case, were my reflections while attending the two major competitive events of the spring season for our group as a whole-- the Ottawa International Race Weekend and the Ontario high school track championships. These were also, of course, major events for many other coaches and athletes, so the opportunity to observe, as I have many times before, the colourful and poignant drama of triumph and failure was ample. Competitive success is always preferable to failure in the first instance, of course; it is what we plan and strive for. As I took in the usual array of ecstatic, celebratory, broken and tearful young athletes, however, I realized yet again how they represent two sides of a coin. Today's celebrants would inevitably find themselves in the camp of despair at some point, and the broken would go on to enjoy competitive success again at some other time and place. And all, if they continued the serious pursuit of the sport, would continue to become more fully and richly human simply by having had the temerity to plan and prepare, in the face of failure, to reach higher.

This, by the way, helped me understand why the friendships I have made through running have been so profound and lasting over the years-- far more so than any I've made through other channels. Competitive runners, if they have stuck with it long enough to have dreamed big and failed (and this, incidentally, includes runners who have actually made Olympic teams, broken national records, and been ranked high internationally), are actually richer, often wiser, and more interesting people than many you might meet. I attribute this more than anything else to the experience of having encountered their limitations through repeated failure. This, I think, is central to what makes serious runners what they are for always, and kindred spirits for all time, regardless of how many years it may have been since they trained or raced seriously. This may well be true of other people from some other fields of endeavour, but it's runners that I know best.




