\chapter{X-Country: A Race Only a Mudder Could Love}
\chaptermark{Cross Country}
\textit{Tuesday, 3 November 2009}
\bigskip

Cross country running: What is there to love about it? It is exhausting and frequently filthy. It is run during the some of the worst weather of the year, and there is no chance for a P.B. And yet so many of us, particularly we 'lifers', do love it, in spite of its unique rigours and general unpleasantness. In fact, we tend to love it almost more than we love running itself.

Like so many of the best things in life, X-C running is an acquired taste. And those of us who love it tend to have acquired this taste early in our running lives, usually in school. The taste for cross country is so slow to take, in fact, that most of us didn't even know we loved it until years after our first leafy, muddy foray. Cross country running was, for many of us, the first kind of real distance running we ever did, since it came first in the school sports calendar. Most of us simply did it because it was there, and because we liked running better, and were better suited to it in body and temperament, than the other fall sports on offer. At first, there was only the difficulty-- the cold, the mud, the steep hills, the struggling through the first colds and flus of the year, and the crowds of other competitors, pushing us, stepping on us, and blocking our way along narrow trails-- and perhaps a small taste of victory here and there. Later, though, cross country-- its intensity, feel, and its smells-- would become intermingled with our melancholy nostalgia for autumn and our early school days in general. (And runners, being naturally comfortable in our own company, and given to introspection, are often prone to melancholy nostalgia.) For those of us who continued to do the sport beyond our high school years, memories of cross country racing, and of traveling with our team mates to run cross country races, would become integral to some of our fondest recollections of early adult life. Many of us would begin lifelong friendships and love affairs on cross country training fields and race courses, and on the buses that delivered us to these places. By our mid-20s, many of us would have acquired an attachment to the sport that would one day see us return to stand on chalked start lines, beside wooden states festooned with coloured tape-- red for the left turns, right for the white, just like the political spectrum-- long after our muscles had lost their bounce and our hair, if we still had it, its original hue. To run cross country, we would discover, is to time-travel: In the throes of competition, chest burning in the cool, dry air, and nose full of the sweet, musty smells of grass, mud and fall decay-- the very same air and the same smells as on our first childhood trips "over the country"-- we discover that less about us has changed than we thought, and we find comfort in this amidst the extreme challenge of the activity itself.

I am, of course, a running 'lifer', and I share this love of cross country, for all of these reasons. As a coach of late-starting masters runners, however, I have been pleased to discover that its harsh appeal is not confined to its power to evoke the past. New runners certainly find X-C difficult, more difficult than the road races which which they are more familiar, and they are often a little baffled by the preparations required to tackle it-- the shoes with the long spikes, the special clothing required to repel the fall and winter elements, and the training over soggy, hilly, and lumpy terrain; but, once they have had a taste, they are often found returning for more. Much of the appeal of cross country running is simply the extremity of the challenge, which is exhilarating in itself. After that comes the temptation to mastery: having tried it once, many runners are often interested to see how much better they could do with a little more practice. New entrants to cross country running are also subject to infection by the enthusiasm of those us who have done it many times before. There is bound to be some curiosity to see and experience what the fuss is all about. Finally, these days, many adults are attracted to X-C as a result of watching their children take to the trails in school races (with many no doubt discovering that it is far harder than it looks!).

Lately, it has occurred to me that X-C has another universal appeal for serious runners, and in particular those living at more northern latitudes. Training for and racing X-C is, quite simply, the best way to get through November without resorting to alcohol, reclusion, and other acts of quiet desperation. Without X-C to occupy our minds and bodies, it would be far more difficult to face the prospect of increasingly dark days, the imminent prospect of winter, training on the pavement (or worse, the treadmill), and a horizon of meaningful races far to distant to even glimpse. It is true that the fruits of spring and summer success are sown in the darkest months of the year; but, that is a mere abstraction when confronted with a 4:30pm sunset, on a rainy, windy and cold Wednesday afternoon, following a trying day at work. Having the most important X-C races of the season in November nicely dispatches running's bleakest month. With the focus squarely on conquering the hills, mud and cold, and in the company of crowds of brightly clad fellow enthusiasts, November is put squarely in its place.

No matter where you are in North America, it is not too late to register for your local,regional, or national X-C championship! Check your federation or local club's website for details.

