\chapter{George's Pictures}
\chaptermark{George's Pictures}
\textit{Friday, 20 February 2015}
\bigskip


\epigraph{Oh as I was young and easy in the mercy of his means, \\ Time held me green and dying \\Though I sang in my chains like the sea.}{---Dylan Thomas, 
\textit{Fern Hill}}

What, I asked myself, is so mesmerizing about George Aitkin's vast trove of amateur (but sometimes excellent) photographs of runners doing what we do? George's collection began when he himself took up the sport some 40 years ago, and it includes many shots of me and people I knew well as friends and competitors (including some of the people whose exploits hooked me on the sport, and kept me hooked all these decades). What exactly am I looking for when I stare at these old shots, I wonder. Like my old training logs, gear, and body weight, George's photos are yet another reminder that running remains my strongest link to the past-- and to young adulthood in particular, as I imagine it does for my fellow lifers, whose comments and mute "thumbs up" accumulate in the spaces beneath each shot. Beyond eating and the usual daily ablutions, there is nothing I have done with the same frequency my entire life as train for and compete in the sport of foot racing. I am vividly reminded of this fact every time I see my much younger self in one of George's frames. But I realize I'm in search of more than this when I scan the faces of athletes (myself included) featured in these old pictures. When I study them closely, the racing shots in particular (and the vast majority are racing shots) seem to me to offer are glimpses into the minds of runners at what were, at the time, moments of utmost importance, requiring intense and earnest focus. Younger runners will be inclined to see these photos as perhaps odd and funny. They will recognize the activity clearly enough (cross country courses and tracks still look largely the same, and running form is timeless), but they are likely to be drawn to the 70s and 80s fashion, hairstyles, and terrible footwear on brilliant display in many shots-- the styles are old, but not yet old enough to look classical, like Bannister's long forelock and Oxford all-white uniform. Or they might notice the cars in the background of the road race shots, all strange angles and skinny tires. I see these things too, of course. But what I mainly see is what Dylan Thomas, the language's most famous, and perhaps best, poetic nostaligist sought to convey in the above lines: heartbreaking innocence and courage in the face of a decline whose imminence is much greater than the young athletes pictured can possibly be aware. There's plenty of joy and fun on display too, but what draws me in is the poignancy of young runners giving so much of themselves in moments of, at the time, utmost importance. I know I have this response because I have now lived through it all as a runner and seen how it ends up (and perhaps also because there are so many fewer photographs of me and my generation of athletes than there will eventually be of subsequent ones). The race we are doing, or about to do, is always the most important one, particularly when we are young and still breaking new ground, wondering how far our abilities will take us. But then it is over, and we move on to the next one, until, eventually, whole swaths of particular races are blended into a single period, their particulars often forgotten or conflated. In bringing these particulars back into relief, and from a then anonymous spectator's point of view, these pictures both jar and fascinate us. We see ourselves as we were then seen, and as we now see younger runners still totally immersed in training and competing. The effect is a genuine kind of nostalgic wonder (not a sentimental one) uniquely available to us because our youthful striving was highly public and physical, and could thus be clearly represented visually. Old photos always hint at what beneath the surface of their subjects, or outside the frame, but old running photos seem to suggest these things in especially stark and poignant ways, at least to an old runner like me.




