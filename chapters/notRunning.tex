\chapter{Not Running: What To Do When It's The Only Thing Worse Than Running}
\chaptermark{Not Running}
\textit{Monday, 21 January 2013}
\bigskip



The short answer to the above question is that you drag yourself out the door for a run, once again. After all, every other non-running activity, while it may be pleasurable in itself, by definition involves NOT running and is therefore less tolerable than running (if not during said activity, then certainly after, when the fact of not having run settles in). If there are other activities whose participants often loathe the prospect of doing what they do only slightly less than not doing it at all (recreational drug use?) I'd like to know what these activities are. In my experience, runners are unique in often hating in particular what they love in general. And this is the time of year-- mid-winter, when the options are often slipping around in the frozen dark or counting the seconds on an electronic console-- when this loathing tends to peak. (In fact, the particular date on which I write-- the third Monday in January, so-called Blue Monday-- is statistically the most depressing date on the calendar here in North America).

The longer answer, and one I contemplate when the challenge of getting out the door is at its greatest, is to try and think about why I (still) love running after 33+ years of doing it an average of 29 out of 30 days, year-round. I'll admit that, after this long, it's often zombie-like unflexiveness that gets me out the door or onto the t-mill. I often think about nothing at all, as far as that's possible, and, with apologies to Nike, just do it. On the other hand, I do often think-- and talk with other runners-- about what it is I like most about running and being a runner, even when I hate the thought of having to do it. Here, in no particular order, are some of the things I reflect on:

\begin{enumerate}
    \item Competing successfully is ALWAYS fun, and this only happens when you make the commitment to train as consistently as humanly possible. Racing is one of those things that richly rewards simple, brute consistency, and harshly punishes its opposite. Distance running is unique in that talent or "skill" will get you nowhere without dogged persistence. The imperative to get out the door and spend time on one's feet is the great leveler in this sport, and those who can manage this will often beat those who can't, regardless of natural aptitude. If there is an indispensable talent required for succeeding in this sport, it is the talent for getting one's ass out the door every day!

    \item Running can be surprising on the simplest level. It's only because I'm good at getting myself to do it that I have learned that even the most foreboding running experiences-- that cold day you didn't want to step out into, or that workout you were dreading-- can end up being the most fun, satisfying, or even enlightening. Chances are, the run or workout you don't want to do will end up being like all the others, but sometimes it will turn out to be fantastic and memorable for some reason, big or small (e.g. a moment of rare and unexpected natural beauty, a chance meeting with a running friend, a performance breakthrough). The possibility of these experiences is eliminated every time you DON'T make it out the door.

    \item The weather is sometimes not as bad as it seems, and becoming master of it can be satisfying. Winter running in particular, which we can legitimately hate for a whole host of reasons, can actually be some of the best running weather of the year-- something we often forget when the season is approaching in November. There is unique beauty in a crisp, clear, and white winter day, when one's body temperature has risen enough to enjoy it, and a dusting of dry snow can provide a little shock absorption on the concrete or asphalt, creating a comfortable running surface that is entirely unique to running in winter climes.

    \item HAVING run, no matter how bad the run or workout itself, is always uniquely pleasurable, and is the deeper reason we find it preferable to not running, even at the worst of times. Rare is the runner who will come in the door saying "I really wish I hadn't done that; now I feel worse" (i.e. barring injury or illness), and that post-run glow can be the most intense and unalloyed pleasure that any of us feels on a regular basis. There is a reason, after all, that something as simple as daily aerobic running (note: not skiing, or playing hockey) is touted as an effective treatment for mild to moderate clinical depression. And there is only one sure route to this feeling; you can't HAVE run unless you've taken that first step.

    \item Simply continuing to BE a runner is worth the effort. Doing a sport as demanding as running, and doing it the way it is supposed to be done, is a source of intense pride for all serious runners. And you can't be a runner in general; to truly be a runner is a decision that must be renewed every day. If you decide you simply don't have the drive to get out today, what's to stop you from making the same decision again tomorrow, or any day after that? From the point of view of the serious runner, no day is any different or more important than any other in the larger scheme of things; they are all tiny pieces of a single whole; there are no great days without a multitude of lesser ones. The runner who decides not to run today for no other reason than it is difficult has already quit being a runner, he/she just doesn't know it yet.

\end{enumerate}
