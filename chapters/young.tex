\chapter{Preparing Young Athletes for the Long Run}
\chaptermark{Young Athletes}
\textit{Tuesday, 24 November 2009}
\bigskip

On no other topic are my coachly musings as likely to provoke reaction as on that of kid's running and youth development in general. While my ideas do sometimes provoke arguments on other subjects, in most instances people are ultimately willing to defer to the accumulated knowledge and experience of my 30-plus years of intense involvement in the sport (they shouldn't always, but they often do). On the subject of kids' running, and long term athlete development, on the other hand, many more people tend to have their own sometimes strong views, and they tend to hold them regardless of how little knowledge and experience they might have-- and especially if they have a personal stake in the argument, as the parents and/or coaches of young runners do. In this week's installment, I offer some explanation for why this might be so, along with a quick summary of my views on the topic.

In the years since I started running, the sport has grown from a fringe activity pursued by mainly by students (older ones-- high school and university), "health nuts", and other (mainly male) "oddballs", to a mass-based, multi-billion dollar health and leisure phenomenon. The simultaneous mass growth of the sport among mainstream adults, male and female, and younger school children has meant that running has become an increasingly family-oriented activity. Kids frequently now grow up accompanying their parents to road races (and sometimes even X-C and track meets), and parents now flock to watch their children compete in school races. Signs of this transformation are everywhere. The finish areas of large road races are now frequently full of children waiting to greet exhausted mothers or fathers, and the parking lots of school races now over-flow with parents taking time from their work days to watch children as young as 6 compete in school X-C and track events. This is in stark contrast to only 20 years ago, when fewer adults with small children ran, and perhaps even fewer took time off work to watch their kids compete. (In fact, when I chat with veteran runners as young as their mid-30s, we frequently remark on how few times our parents had ever watched us compete when we were younger. And I don't think I knew anyone whose parents themselves ran competitively. My own father watched me race perhaps twice in my life, and my mother, who actually became a runner herself in middle age, perhaps five times, even though I was considered a "star" performer from about the age of 15. They were supportive, of course; but they just didn't think this entailed attending all of my races.)

The advent of running as a "family" pursuit, while generally to be welcomed, has had a few important implications where kid's running and long term athlete development are concerned. Increased parental participation in running has tended to mean that kids are now becoming involved in more serious competition and training at younger ages, with their parents often acting as training partners and coaches (or, at the very least, as keenly interested bystanders). When combined with an explosion of on-line, magazine, and book-based information on all aspects of running, the result has been a vast multiplication of people with both an emotional stake in the sport via their children's involvement, and a certain amount of casual expertise. (When looking for some sports-based precedent for this phenomenon, we might consider hockey in Canada or baseball, football and basketball in the U.S., in which the figure of semi-expert "parent-coach" or "sports parent" is now a stock-- and somewhat humourously stereotypical-- one.) And, just as in other sports where the parent-coach/aficionado has become a central figure, the parents of successful age class runners in particular are often the most heavily involved, sometimes claiming an expertise and authority disproportionate to their actual level of knowledge or experience-- such as an American runner-dad who launched a website to promote his home-grown coaching theories, using his precocious daughter as the basis of his authority!

When I encounter resistance to my ideas about kids running, it is most often from the parents and/or coaches of younger and more successful kids-- those who tend to be intensely involved in the sport outside of a school program. I regularly receive inquiries about coaching from the parents of children under 13 (I restrict participation in my club group to kids 13 and over), and most understand, or at least offer no resistance, when I explain to them the basis of my views on kids and running. I have, however, had some pointed disagreements on this subject, and I know that my views are not always shared by other youth coaches and parents in the sport. In my experience, the parents and coaches of heavily involved young runners almost always mean well, and believe they are acting in their young athlete's best interest. They tend to believe that, if they are only facilitating and not compelling their child or athlete's involvement, they are doing no harm as far as his or her long term development and relationship with the sport are concerned. What they often unaware of-- and sometimes willfully, because of the deep pride and sheer enjoyment they experience in watching their young charges succeed-- is that intense early involvement in the sport, and the competitive success that tends to go with it, very often leads to early difficulties and premature abandonment of the sport. And, while it is true that not all young athletes take up the sport with view to reaching its highest competitive levels, and that a few do indeed manage to reach these levels following a childhood of intense early involvement, it is, or should be, the responsibility of adults to give young athletes the best chance of reaching their full potential in the sport, just as they would in any other life-endeavour, whether athletic, academic, or cultural.

Giving young runners the best chance of reaching their full potential involves, first, understanding the specificities of the sport itself, and being open to learning as much about its science and lore as possible. I have arrived at my own views about kid's running, and long term athlete development in general, through my own experience, of course, but also through familiarizing myself with whatever formal research exists on the subject (of which there is, unfortunately, not nearly enough). For what they're worth, I would summarize the results of my observations and analysis as follows:

\begin{enumerate}
    \item Young running prodigies, defined as kids who run far ahead of the next best in their age cohort, very rarely convert their age-group success into adult, or even senior high school, success. A casual perusal of the early age-class results for North America over a 20 or 30 year period record is sufficient to bear this out. And, in fact, former prodigies seem to drop out of the sport at about the same rate and at the same ages as non-prodigies-- which is somewhat surprising, given the apparently much greater incentive for early age group stars and record holders to continue in the sport. The precise reasons for this are subject to debate, but I suspect a number of factors are at play. My own view is that the almost inevitable evaporation of prodigies' early advantage over their peers, which may have been in the first place the result of natural physical precociousness or, more often, the early introduction of systematic training, and the age at which that loss of advantage occurs, combine to create pressures on young athletes to which quitting the sport may seem, at the time, like a reasonable response. Prodigies who are no longer winning races easily, and who have often been training hard for years (either in running or in some other aerobic sport, such as swimming), must often feel as though they are falling behind when their peers begin to match them, even while they remain among the very best in their cohort. Breaking records and winning races, often against much older competitors, must, after all, be a tough act to follow in a young life, and being reduced to simply one of the best can probably seem like failure.

    \item The vast majority of today's top runners, while often very good as young runners, were not what anyone would call prodigies, and quite a few were far from it. When I began thinking systematically about this problem years ago, I made a habit of collecting stories of athletes who were very ordinary age class performers, or very late-starters in the sport, yet who managed to reach its highest levels as adults. In this file can be found everything from world record holders and Olympic champions (such as Sebastian Coe, John Walker, and Robert Cheruiyot) to some of Canada's current top athletes (such as Reid Coolsaet, who has now won more National Senior Championships than he ever won age-class medals!). This pattern, perhaps more than anything else, marks distance running off as different from most other sports, and certainly from those other sports with which most North American parents would be familiar-- e.g. hockey, gymnastics, and swimming--, where intense early involvement and age group success would seem to be a stronger predictor of long term success.

    \item Complete maximization of personal potential in running takes a very long time, and those who lose their enthusiasm for the training process at an early age never become as good as they might have. It is not unusual for runners to perform lifetime bests and win Olympic and World Championships medals at ages 32-plus. The oldest Olympic champions in the marathon, for instance, were 37 for men and 38 for women, and the current men's world record holder set the mark at the age of 35. This late maturation is possible because success in the sport is a function of some very basic physiological adaptations-- adaptations that can proceed for decades with the right program, and under the right circumstances. Being a very good runner also requires a certain amount of wisdom, patience, and emotional resilience-- characteristics that are developed through life experience, and are therefore found more often in adults than in children or adolescents. Very few people who have not experienced the process, or witnessed it up close, can fully understand the extent of the difficulty of progressing from a very good age class runner to a national or world class athlete, and so it is easy for them to believe that the fastest 12 year olds are the most likely to become the most successful adult runners. In reality, a successful career in running is actually made up of two or three different careers, with the success of each dependent on the careful negotiation of the last. And the challenges in each phase are all, in their own ways, equally acute. Such is the difficulty of making it "all the way" that it is a small wonder that \textit{anyone} ever does it!

\end{enumerate}

The lesson that can be taken from all of this, and the one I try to impart when called upon to give advice on kid's running and long term athlete development, is that the optimal plan to ensure that young runners both enjoy the sport and retain the best chance of reaching its highest levels, is one that involves relatively late-starting (age 13 or older), infrequent and mainly local competition, and non-specialization till the age of at least 15 for girls and 16 for boys. There is no evidence to support a theory that those who start systematic, year-round training at early ages gain any long term advantage on their peers, and plenty of anecdotal evidence that such early involvement may actually be counterproductive for the average kid. And, perhaps more importantly, I see no evidence that kids who start sooner and more seriously have any more fun with the sport than those who don't!

My long term observations have thoroughly convinced me that the development model prevalent in other kids' sports is wholly inappropriate for a sport like running; which, because it is based primarily on the simple development of gross physiological capacities, is primarily drill or work-based, and lacking in a significant play element. To put it another way, competitive running is not, when it pursued seriously, truly a children's sport. And, unlike the other sports that kids typically play, it is entirely possible to reach the highest levels in running without ever having pursued it seriously as a child; in fact, later starting is probably optimal for long term success. Granted, there will always be outliers who will defy the odds and build successful adult careers on the basis of intense prepubescent involvement; but, addressing the problem of development entails consideration of what is optimal in light of the fact that the long term response to training of any individual athlete cannot be known in advance. Modeling development on what we know about the average or typical path to success of top runners is the responsibility of everyone who works with young runners, particularly when it does not involve any sacrifice of enjoyment for these athletes. The pursuit of an experimental or counter-intuitive path to development represents a risky form of self-indulgence on the part of youth coaches and the parents who support them. Defiance of the developmental odds may pay off in individual cases, but the costs of losing this gamble can be great for the young athlete. Pursuit of a more grounded and proven path, at the very least, eliminates the basis for regrets and second-guessing if and when a young athlete decides to abandon the sport before his/her full potential has been realized.




