\chapter{The Living Relic That I have Become}
\chaptermark{Living Relic}
\textit{Monday, 22 June 2009}
\bigskip

The other day, I entered my first National 10000M Track Championship since winning the event on a lovely Montreal summer evening in back in 1998. Since the announcement that the 10,000m would be recombined with the main meet for the first time in almost a decade, and held on the beautiful new track at Varsity Stadium in Toronto, I'd been toying with idea of entering. My hesitation stemmed from a feeling that, at nearly 46 years of age, I had no business contesting a race meant to determine our nation's fastest long distance runners. My idea was take this one available opportunity for a competitive track 10000m to attempt to break former Toronto Olympic club-mate Jerry Kooymans' national age group record (31:43) to go with my recent national age group road best (31:12). I was also concerned, however, to avoid looking like an eccentric old crank, struggling along, laps behind the leader, trying to make a point that no one really understood or cared to see made. I also didn't want to get in the way, physically, of all the younger, faster runners.

Then, on the day of the entry deadline, I had a look at the list of confirmed entries. In doing so, I learned, first,that I was one of only 17 or so athletes (3 for the women!) interested in accepting challenge of racing 25 laps of the track-- an extreme and pure test of the distance runner's basic mental and physical chops. Second, I learned that, with a seed time based on my recent road performance of 31:11, I wouldn't need to fear getting in the way of too many of my younger competitors. Most entrants, as it turned out, had seeds closer to mine than to the fastest athletes in the field-- Reid Coolsaet, Dylan Wykes and Andrew Smith. Recalling that, in bygone decades, I'd had to face a strict standard of around 30mins flat to gain entry to a field that often numbered in the 20s of entries, I wondered, as I have so many times before, how it had come to this. The top athletes in this year's field are good, but no better than the top seeds I would have faced 25 years ago. Behind them, however, there are now only a tiny number of athletes fast enough to meet the entry standard from 25 years ago.

Since turning 40, I've become used to accepting congratulations for being able to hold my own, to some degree at least, against competitors young enough to be my offspring. To anyone interested, however, I've always tried to explain that my results are less the result of my peculiar abilities-- whether owing to good genetics or simple persistence-- than the product of the rather precipitous decline of Canadian long distance running beginning in the early 1990s. I only appear to be exceptional, I explain, because I am a living relic from a once great (or at least much better, in both absolute and globally relative terms) running civilization. While quite good at my peak, I was not by any stretch the best of my own era. I am still able to compete with the good runners of today not because I was freakishly good in my own prime, nor even because I'm particularly well preserved physically. I can still compete, rather, because I am the product of a period in Canadian distance running in which the merely good runners were so much faster than they are today. I am unique only in the fact of having had a combination of the ability and the desire to stick with it longer than almost all of my contemporaries. Make no mistake, I have lost much in terms of basic capacities. As a former 19 year old with the ability to run well under 30mins on the road for 10kms, and with 20 years of serious racing and training on top of that, I simply had so much more to lose in terms of psychological skills, racing savvy, aerobic power and endurance. The result is that, in spite of my decline, I still find myself more or less in the competitive mix in today's weaker fields. Thus I remain: a kind of living record of the way things were, standards-wise, not so long ago. Don't have the time or patience to do a comparative analysis of the race results and rankings from 25 years ago and today? Just look at me: a good but never great runner from the 80s and 90s who is nevertheless, at the age of nearly 46, still able to enter a national championship without fear of becoming a complete spectacle.

While I can't deny getting some enjoyment out of still being able to go head-to-head with much younger athletes, I also confront the current state of stagnation and decline in Canadian long distance running in my guise as a coach; and, as a coach, I often wish today's young athletes, led by the athletes with whom I work, would restore a standard of performance in this country that would retire me from open competition for good! Adducing explanations for the decline of Canadian (and, indeed, North American and European distance running)running has by now become a kind of electronic cottage industry (to which, in fact, I have been a direct contributor). Answers typically range from: the social and cultural, tinged with the moral ("kids don't want to work hard anymore/are distracted by various electronic inducements to physical passivity" or they are "over-scheduled and burned-out"); to the pseudo-scientific and racialist ("African kids have genetic advantages that make them unbeatable by non-African kids, non-African kids know it, and have given up all hope"; to the political ("phys-ed/physical fitness in schools has been progressively de-emphasized and/or cut, making the average kid far too unfit for distance running", or, "Canada has tougher drug testing than many other countries, which has allowed a demoralizing gap to open up between our best and the drug-fueled global standard"). I have sympathy for some variants of all of these broad explanations for the decline. I think, for instance, that athletically talented kids tend both to play too many different sports at the same time, rather than seasonally, and to take their sports too seriously at too young an age, aided and abetted by over-invested adults. The result tends to be large numbers of talented young adults with no taste for serious sport during their prime developmental years. I also think that there are structural factors that have shifted the balance in the sport in favour of athletes from parts of the developing world, and East Africa in particular-- structural, but not racial or genetic, however. Differences in the age demographics of the developing versus the developed world have, among other things, served to skew the performance lists in favour of East African countries as against those of the developed world. Add the fact that the average Kenyan is over 20 year younger than the average Canadian-- and in the prime age range for distance running-- to the economic reality that international distance running offers fairly ready access to hard foreign currency for the sons and daughters East Africa's poor and it is a wonder that their dominance of the sport is not more total than it now is. And, no doubt, the almost overwhelming "African-ness" of distance running today is bound to create a subtle disincentive to non-African kids the world over. Non-African kids can can certainly be excused for not seeing distance running as "their" sport to any significant degree(in a way that, for example, a young British athlete circa 1970 might have done). (It is interesting to note that Britain's top young athlete on the men's side-- Mohamed Farrah-- is of African descent, as in Canada's top man, Simon Bairu. Where some might be inclined to explain this in racial or genetic terms, I would be more inclined to see it in terms of the power of a good example-- at least until someone comes along to furnish the still elusive proof of a broad "natural" advantage among African runners).

My choice in confronting the decline of competitive long distance running in Canada has been, however, to emphasize the possibilities rather than the obstacles to the re-growth of the sport in this country. (I should emphasize, however, that I would and will remain a fan of the sport regardless of who is winning internationally; it's just that my "sphere of influence" happens to be Canada). More Canadian kids than ever before are being introduced to distance running, even if often too early and too seriously; and, the standards of performance in the age group ranks are much stronger than they have ever been (a paradox, considering the decline at the top). In fact, the main problem in Canadian long distance running today concerns the "long distance" part. As the entry numbers for the middle distance events at this year's nationals attest, middle distance running is alive and well in this country (even if it, too, is somewhat lacking in competitive depth). The problem is that far too many competent young middle distance athletes abandon the sport after high school or university without ever having tried their hand at the longer distances, which tend to require a much longer apprenticeship to master, even with the best training program (without expert supervision, it's highly unlikely that a young athlete could successfully negotiate the transition from, say, miler to marathoner). My strong suspicion is that many of our second tier middle distance athletes are, in fact, potential first tier long distance athletes, toiling in the wrong event range. The current system of youth development in Canada-- although, in truth, there is little that is systematic, in the sense of longer term goal orientation, about it-- tends to produce scores of good, and a few very good, 800/1500 and occasionally 5000m runners, but very few 10000m-marathon runners at all, let alone fast ones. Beginning in grade school, the standard approach to youth running is based on an intense seasonal orientation, focused on low volume, frequent racing, and 3 or more intense track or tempo sessions per week. This approach is a proven method for turning out vast quantities of athletes capable of running very fast 800 and 1500s by their mid-teens (witness this year's OFSAA results, where sub-2:00 800m and sub 4:00 1500s were being run routinely by 15/16 year old boys and sub-2:20 800s and sub-4:50 1500s were de rigeur for 14 year old girls). As a basis for producing sufficient numbers of prime-age long distance runners, on the other hand, this approach is now a proven dead end. Chances are, an athlete who has only ever trained in this way will balk at the suggestion of spending 5-10 years in their 20s and early 30s running two or three times the volume they had become used to as a youth and junior athlete. Most will find it easier to continue (if they continue at all beyond their school years) doing what they have been used to, rather than exploring their longer distance potential.

My medium term hope for Canadian distance running is that this trend will begin to turn around, just as it has in the U.S. over the past 10-15 years. I'm hoping that many more of our talented high school and college-aged middle distance runners will become interested in contemplating an athletic future in the longer distances-- 10000m to marathon, and road racing in general. In fact, it has been one of my principle aims in establishing Physi-Kult running to encourage more young athletes to take this step, and to make my knowledge about how to do it as widely available as I can. I have tried to begin this process in my own work with teenage athletes. My coaching practice with age group athletes-- still admittedly experimental at this stage-- has been designed to encourage slower but sustainable year-to-year improvement, and to promote an understanding in young athletes that one's full potential takes years realize. In terms of day-to-day practice, this entails an emphasis on longer, easier running, with total amounts increasing yearly, along with fewer hard interval sessions and races than most youth programs entail. Working against the grain in this way is difficult. It would be much easier in many ways to do things in the now established way-- that is, open my group to very young athletes and "hot-house" them to fast performances in the younger age class ranks. Some kids (and their parents) might be happier this way, at least for a while; but, this would only feed into the trap that is claiming scores of potentially talented young distance runners-- a trap that ends with both unfulfilled athletic potential and possible disillusionment with running in any form in adult life. My choice, therefore, it to try to build a coaching practice around what I know to be the surest path to long term success. In the process, I hope to ensure that, in 20 years, the idea of a 46 year old running in our national track championships will once again seem absurd!




