\chapter{Jack Daniels Primer \#2 -- Counting Your Steps}
\chaptermark{Daniels \#2}
\textit{30 March 2009}
\bigskip

How can you spot an experienced runner from distance? Is it her speed? Her physique? Her clothing? Perhaps, but I would say it's something more basic and, in fact, subconscious to the viewer. It is her cadence. At any speed, in any clothing, and in a wide variety of body shapes, our eye can tell the difference between a beginning runner and someone who has been at it for a while. The secret is in the cadence: experienced runners, almost universally, will have a far quicker cadence than a beginning runner, and that cadence will be remarkably uniform at all speeds, from jogging to racing a mile.

Jack Daniels began wondering about the problem of cadence-- whether there was, in fact, an ideal cadence for optimal runner performance-- over 20 years ago. As a result, he decided, while watching the 1984 Olympics in Los Angeles with his wife (who acted as his assistant), to actually count the number of steps that Olympian runners from 800m to the Marathon typically took while competing. He (or rather they) discovered that there was a remarkable uniformity when it came to turnover among the worlds best runners. All of them-- male or female, short or tall-- took between 180 and 200 steps per minute, with the higher numbers recorded at the lower end of the distance range. He wasn't able to say precisely why, but he had obviously discovered a manifestation of some very basic physiological (probably neurological) principle. No runner, after all, is ever told, or ever consciously attempts, to take between 180 and 200 steps per minute; and yet here were the best distance runners in the world, all turning over more or less in sync. Armed with his data, Daniels then proceeded to count the steps of less accomplished runners (typically, those newest to the sport) and found that there cadences were typically much slower-- sometimes as slow as 160 steps per minute.

Since I started coaching beginning runners-- both masters runners and kids-- I have been intrigued by Daniels' findings. On counting my runners steps, I was able to confirm Daniels' original findings-- beginners typically stepped more slowly than more experienced runners, even when they were able to run faster than their more experienced team mates. Furthermore, I noticed that, with few exceptions, beginners would increase their cadence with their experience level until they moved into the magical range. I would often become aware of this without even looking for it, particularly in the case of kids. Now, some beginning runners had higher cadences than others (occasionally even approaching 180 steps per minute), and some experienced runners had cadences slightly below 180 steps per minute at slower speeds (including, interestingly, Dylan Wykes, whose cadence while in college did not surpass 180 steps per minute until he reached 5mins per mile). However, there remained a very strong correlation between experience and cadence level.

The question many runners ask when offered this information is: what can I do if my cadence is slower than 180 steps per minute? The short answer is, of course, simply run more and be patient. However, this is a trickier problem for that small minority of more experienced runners who still stride slowly; and, contrary to common sense, it doesn't help to try to simply take more steps. When any runner tries to consciously increase turnover beyond their intuitive range, they typically chop their stride unnaturally, or else speed up. (Go ahead, try it.) If there is a way to address this problem in relatively experienced runners (and I'm not convinced there is an easy solution), I suspect it may have to do with improved core strength and running posture. I've noticed that slower steppers tend to have a little more forward lean when they run, and that if they concentrate on "running tall"-- that is, pushing the hips forward, causing the heal to recover more quickly after toe-off-- their stride rate will often increase. It does no good, of course, to attempt this new posture without the strength to support it for long periods of time; hence, the need to address core strength.

So, have fun counting your and your running partner's stride rates on your next run (I know you will)! And next time you think you've spotted a "real" runner at a distance, count his or her steps-- and don't be surprised at the number.




