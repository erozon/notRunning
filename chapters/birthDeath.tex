\chapter{XC: Birth, Death, and Rebirth(?)}
\chaptermark{Birth, Death, and Rebirth}
\textit{Wednesday, 2 December 2015}
\bigskip

In our part of the world, distance running careers are typically born on the green grass of early autumn, and individual competitive seasons die in the cold mud of late November. But the 2015 Canadian running season's death in the dirt of Kingston's Fort Henry Hill might also have been the demise of something much longer lived: the era of gender inequality in XC racing distances.

With the first strike of gender inequality's death knell having been sounded at the highest levels of the sport-- the IAAF, which rather suddenly announced the introduction of gender equality at the senior level-- and the second strike (an end to inequality at the junior levels) possible in the next couple of years, it is quite likely that this year's Canadian XC Championship will be the last to offer shorter distances to female athletes. It is equally likely that similar change will be enacted at lower levels, and in most jurisdictions. Once birthed, equality matures very rapidly, and fends for itself.

But what's so important about gender equality in XC running when we have had it in running's other disciplines for decades? Why would it matter that women and girls will now be allowed (or required, depending your view of the change) to run the same distances as boys and men when they have already been doing so, and all the way up to the marathon? Can anyone really believe that girls and women are incapable of racing the same distances as boys and men when the race is being held on natural as opposed to artificial surfaces?

Resistance to gender equal distances that is not of the straight-up troglodytic variety does not maintain that girls and women are not capable of racing as far on grass and dirt as boys and men. The case for continued inequality is more subtle than this. Its terms, however, tell you just about all you need to know about why adopting equal distances is so important, at least for those of us who truly care about distance running as a competitive sport for both men and women. We continue to be told, and not without some empirical basis, that girls and women themselves typically don't want to, don't think they should, and don't want to be required to run the same distances as boys and men, if those distances are going to be significantly longer than the distances they currently run. We are sometimes told this by female competitors and coaches themselves. We are also sometimes told that girls and women will abandon the sport in significant numbers if required to apply the same effort as boys and men to train for and complete the required racing distances. Again, this is not entirely without empirical basis. But, it is the very fact that these assertions have some empirical basis that is the greatest indictment against 30+ years of unequal race distances (and attendant training expectations)in this sport: Unequal racing distances have helped shape girls and women's own perceptions of the meaning and purpose of the sport for their gender, and it has done so in a way that does a profound disservice to the athletic potential (and sometimes aspirations) of a particular subset of female athletes-- those with the potential do do better over traditional (i.e. men's) XC distances. The prolonged practice of unequal racing distances in XC has made female athletes unwitting agents of their own exclusion, or agents of exclusion of other women with slightly different but relevant athletic makeups.

XC racing is typically the first form in which young runners of both genders encounter the sport of distance running, because XC running is overwhelmingly a school-based sport, and because it has been typically offered in the fall or winter months. Before they ever attempt a middle distance track race, athletes of both genders will typically have raced a longer distance over grass and mud. In many jurisdictions, those distances will have been gender-equal at the earliest ages, when the sport is typically done for fun (an extremely challenging form of fun, but that's another story). During precisely the years when athletes typically choose to approach the sport as a serious competitive endeavour, boys and girls begin to be offered different distances. It is at this highly formative moment-- a moment when they often still feel they are, and sometimes actually are, the full athletic equal of boys their own age-- that female athletes receive their first lesson in their own alleged athletic fragility and psychological inferiority (a lesson often reinforced by the larger culture). You may not yet understand why, girls are told, but you are not suited to running the same distances as boys in the long distance sport of XC. You are henceforth consigned to a more suitable, miniaturized version of what the boys and men will do. And to this broader sociological lesson is eventually added a sport-specific one: If you are bigger and stronger, and possess greater ability over the middle track distances (that represent the long distances in the school system), you will continue to succeed in the sport of XC disproportionately to boys and men of similar physical makeup, who will typically fall further behind as the competitive distance increases (disproportionately to that of girls/women). If you are female and good at this mini version of XC, you will likely grow to enjoy the disproportionate competitive success you get to experience during the fall, while your male counterparts are often learning to live with temporary or permanent consignment to the middle of race packs. And you may even begin to feel a sense of entitlement to the sport you have been allowed-- even encouraged, by coaches who have become attached to the separate and supposedly equal status quo-- to colonize. If you are female and aren't as well suited to this shorter version of the real (read: men's) sport, you will likely never get to find out how relatively unsuited you actually are. You will likely continue to play along for the other benefits the sport offers (team camaraderie and competition), but you will have been systematically denied an equal opportunity to discover and enjoy the benefits of your particular physiological gifts. If you are lucky, you may eventually meet a coach, perhaps from outside of the school-based system, who will go out of his/her way to introduce you to long distance track racing, or even triathlon. If you are part of the unlucky majority, however, your particular talents will remain buried forever. And, for those who care, the loss will be not only yours but the sport's. Thus it is today in XC exactly as it was in the sport writ large before the introduction of women's long distance racing on the roads and track, when the longest distance female athletes were "permitted" to race was 800m.

But gender equality in XC race distances remains more important than on the roads and in track precisely to the extent that XC racing is seminal to the sport of distance running as a whole for both genders. Our first impression of what distance running is is offered by school XC; and, if we're Canadian and not able or inclined to run in the NCAA, our best opportunity by far to discover and develop our ability over the longer distances will be in school-based XC. If we're male, however, that opportunity will be a more meaningful one than if we are female, as long as the practice of unequal racing distances prevails.

The Canadian XC Championships in Kingston marked the death of yet another competitive season, but whether it also marked the death of an entire (and over-long) era of fundamental inequality in the sport itself and its rebirth in a more egalitarian one-- an era in which female distance runners on the cusp of committing to the sport will never remember a time when they didn't race as far as boys and men, and in which female athletes with true long distance running ability will be allowed to prevail-- is in the hands of the sport's coaches and administrators in every jurisdiction, starting at the top. And while the logic and substance of the argument for equality is overwhelming, the power (at least in the short term) of those who sit atop the sport is real, and the psychological rootedness of the gender unequal status quo very deep in some quarters. Expect some defenders of unequal racing distances to retrench behind jurisdictional walls, to fight back with technical arguments and special pleading about the uniqueness of their systems or teams, and to point to the practices of other jurisdictions (even other sports) as justification for going slowly or for not acting at all. And, of course, also expect many to defend the status quo by citing the interests and desires of current female athletes themselves (never minding the fact that the perspective of these athletes is in large part a product of the unequal system itself, or can be explained by the fact that many of them have a vested competitive interest in the unequal status quo). But also expect these agents to become embarrassed by their own arguments in proportion to their exposure to outside scrutiny; and, if that scrutiny is sustained enough, to eventually accede to the logic of equality. Finally, expect them to one day pretend they were always on the side of equality. Just as no one will ever admit to having opposed equality in track and road racing (or having supported any other of the ideas now residing on history's scrap heap), everyone will one day always have been in favour equality in XC racing!




