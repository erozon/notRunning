\chapter{Feeling Groupy}
\chaptermark{Feeling Groupy}
\textit{Wednesday, 20 January 2010}
\bigskip

Approximately half of the 60-odd athletes-- ages 14 to 50ish-- under my tutelage carry out all, or very nearly all, of their training alone, while the other half trains mostly in sub-groups made up of runners of similar speed (albeit frequently of different ages). I, personally, have spent virtually my entire career training completely alone, easy runs included. Which kind of athlete, you might ask, is better off when it comes reaping the full benefits of their training efforts-- the grouper or the loner?

And the answer is\ldots wait for it\ldots: It depends. That's right, "it depends", which is no less true an answer for being boringly hedged. It depends on the athlete and it depends on the dynamics of the group in question. And, it depends on the kind of training being undertaken.

When it comes to athletes, you might be tempted to think that those who prefer to train in groups do better if they're able to do so. The truth, however, is that those who strongly prefer group training are often those most dependent on it to get their workouts done, and at the right pace. The athlete who has come to rely on a group to train properly, or at all, is probably best advised to spend some time learning the discipline of training alone. This is a problem that many younger and exclusively school-based athletes must confront; if they want to continue their careers into adulthood, they must often learn the discipline of working out alone. For some this comes more easily than for others. But all athletes who choose to train exclusively in groups should learn to do some training alone; first, because they may occasionally find that a group is not available to them; and second, because learning to train alone can actually help improve their racing skills. Learning to pace without the aid of a group, and developing the ability to push hard without the familiarity of regular training partners to act as cues, can make one a more well-rounded athlete, able to thrive in a wider range of racing situations. I'm also convinced that a stint of training alone can make an athlete mentally stronger. Learning to train alone when we've never done it before, and are perhaps afraid to try, can reveal heretofore unknown depths of resolve and inner strength from which to draw in difficult race situations.

Likewise, athletes who prefer to train alone exclusively can benefit from going against their personal grain once in a while. Since racing is almost always a group activity-- albeit one in which the members are not necessarily inclined to act supportively(!)-- it is necessary for all athletes, and particularly those inclined to be loners and/or control freaks when it comes to their workouts, to learn to run with others around, and to do it in spite of-- indeed partly because of-- the chaos that sometimes ensues when five or more fired-up runners all attempt to negotiate a given training pace. As a lifetime loner (albeit less by choice than by the simple necessity of very rarely having more than a couple of athletes of similar speed ready to hand), I always felt a little awkward and generally more tired at a given pace when training in a group than when going it alone-- that is, unless I was leading the repeat. I couldn't help but think that I might have been a more effective racer had I been able to do more group training.

As a general principle, and for the vast majority of athletes, group training is a very powerful tool. The vast majority of the world's top runners, and an even greater percentage of the best runners at lower levels, are the products of group training. In practice, however, the effectiveness of group training will vary according to the "culture" and performance profile of the group in question. Simply put, when the members of a training group are cooperative and mutually supportive, when they understand that a workout is a means to becoming fitter for racing and not itself a competition, and when the range of abilities within a group is very narrow, group training is at its most effective. This is why the best coaches will tend to avoid the "one big group" approach to workouts-- i.e. in which the entire group, regardless of ability, storms off into the session at the same time, sorting themselves as they go along, and usually after most of them have already undermined the purpose of the session by starting too fast. Effective use of group training starts with a clear understanding that the best way for athletes to improve is NOT by trying to go as fast as possible in every session, or by futilely chasing a faster athlete week in and out. With this principle clearly understood, the knowledgeable coach will separate his/her larger group into sub-groups based on both ability and personal compatibility, disregarding both age and gender if necessary.

Finally, the effectiveness of group training varies with the kind of work being done. I find that group training works most effectively at either end of the intensity spectrum.

For longer, easy runs, a group dynamic can be very useful, even when a group is not all that evenly matched in terms of speed. On easy days, faster runners can often run with slower runners if faster runners are willing to go at the slower end of their easy pace range, and vice-versa; or, if their pace ranges don't overlap, the faster runner is prepared to run some extra time to make up for the slower pace. Given that the most important thing about easy days is simply getting them done consistently, the ability of a group to provide company and a little welcome distraction once in a while makes it a very useful support for this kind of training.

For the fastest kinds of training we do-- i.e. intervals at 1500/mile or 800m race pace-- the group dynamic can be highly effective not for its capacity to facilitate disassociation and enable us to simply "get through the session"; but rather, for the way it encourages us to focus and relax. When trying to run at middle distance race speeds, the premium is on the ability to relax and "float", so as to forestall the inevitable moment of muscular failure. In my experience, there is something about having others around going at similar speeds that facilitates this kind of relaxation; in particular, having another body in front, and thus not having the responsibility for establishing the correct pace for the work-bout, seems to make running at close to tops speeds for prolonged periods of time just a little bit easier. Then, of course, for those who actually plan to race middle distance events, some familiarity with what to do when there are other bodies in very close proximity-- all vying for space along the shortest line around the track-- and changing speeds as they either falter or forge ahead, is essential for top performance. And, the greater ratio between stress and recovery times involved in faster training makes it easy to regroup before each repetition without compromising the session for the slightly faster members of the group.

It is when attempting to train at intermediate paces that working in a group can become sub-optimal in the longer term for individual members. At these intermediate speeds-- in particular, max V02 paces and tempo run paces, which make up the vast bulk of the faster running that most runners do, and which involve the very careful combining of volume and pace-- group dynamics can sometimes undermine the training of individual members, even when the group is fairly evenly matched. In this kind of running, where the difference between controlled running-- training-proper, if you will-- and time-trialling is a very fine, and where sessions are typically frequent and very demanding, the group-driven competitive inducement to run even a little bit faster than is optimal can slowly degrade an athlete's performance over time, even when the additional intensity stimulates some rapid short term gains. This risk is particularly acute for the slowest member of any group, who must sometimes run a little faster than might be optimal in a given session in order to remain in touch with the group. The reverse might be said for the faster members of a group-- that they must sometimes run slower than might be optimal for them in a given session in order to remain in the group. This is certainly true, but the effects of running workouts perhaps a little too slowly on a regular basis are far less destructive than the effects of trying to run even a little too fast every week.

With these qualifications duly registered, my advice to all runners is to avail themselves of the group training option where possible, and assuming all other factors are equal. The trouble, of course, is that all relevant factors-- such as the convenient availability of a good group, combined with adequate coaching expertise-- are rarely equal. The majority of the "solo" athletes with whom I work train this way by necessity-- either because their work and family demands compel them to train at irregular times, or because the available group options do not come with sufficiently expert coaching guidance. These athletes have contacted me in order to make the best of their training options, subject to the constraints of their daily routines. In situations where the group option, combined with good coaching, is available, however, athletes hoping to maximize their performance are well advised to take it. And even athletes who can't make the group option work for them are advised to find a cooperative and equally endowed training partner or two at least once in a while, in order to experience, if only in a small way, some of the benefits of the group experience. Until now, I have said nothing about the social benefits of group training. The question of performance aside, running with a regular partner or group can be the basis for deep and abiding personal bonds. (My local group has become a group of lifelong friends, who socialize together on a regular basis). And when these social benefits are combined with those of performance, the result is a powerful synergy that can propel individual athletes to levels that they would not have imagined possible. Happy athletes training alongside athletically compatible partners are a potent force for success at any level.

In the best training groups, success feeds on itself until it becomes almost routine and to be expected. Witness, for example, the exploits of Guelph's Speed River Track Club, which combines senior elite with university and junior level athletes. Individual members of this group have tended to perform far above the levels they achieved as individuals, or as members of other groups. Now that this group has achieved "critical mass" in terms of its size, cohesion, quality of coaching, and professionalism of administrative support, the process of achieving national carding, winning national championships, or qualifing for national teams, has become utterly demystified for its individual members, who only have to look at the person beside them to know what success at this level looks like and precisely how it is attained.




