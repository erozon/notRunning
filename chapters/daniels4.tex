\chapter{Daniels Primer \#4 (and more) -- What is a "Tempo" Run?}
\chaptermark{Daniels \#4}
\textit{Wednesday, 1 July 2009}
\bigskip

When I first encountered the work of Jack Daniels, in the form of the first edition of his now famous Daniels' Running Formula, the "ingredient" I found most interesting, and initially the most useful, was his "tempo" or "t-pace" run. For years, I had been doing harder, sustained runs of 6-8k, which I referred to as "tempo" runs; but, I had little idea of how, precisely, these were best approached, and I had an even dimmer understanding of how they were supposed to make me a better runner, beyond the psychological dimension of preparing me to concentrate, and hurt, for longer than time it took me to run 400m to 2k intervals that were the main staples of my training at the time. Without this basic understanding, I usually approached these runs as train-through time trials. Indeed, I hit some prodigious speeds in these sessions-- sometimes as fast as 2:55/km for as far as 8k. Because I attacked these runs with such ferocity, I was never been able to do them more than once every 10-14 days, and I gave them up entirely during the main racing seasons. Through Daniels, I was to learn a whole different approach to the sustained, faster-paced run; one that helped add several more good years to my open racing career, I am convinced.

With the help of Daniels, I was to put the "tempo" back in my tempo runs, which was ultimately to make all the difference.

As readers of this blog (or of Daniels himself) will know, the secret of Daniels' famous "formula" is his careful delineation of a series of training paces corresponding to various percentages of a given athlete's velocity at maximum oxygen uptake (different from a standard maximum volume of oxygen uptake, or "V02 max," which is a measure of the amount of oxygen an athlete can take in divided by his/her body weight). In his lab studies, Daniels observed that athletes could typically sustain their velocity at maximum oxygen uptake for an average of about 12mins, or approximately the time it takes to race 3 to 5kms, after which they would begin to slow dramatically, with associated increased levels of lactic acid in the muscles (I say associated because it remains unclear as to the actual role, if any, of lactic acid accumulation in causing an athlete to lose velocity when they typically do). Daniels also discovered that this velocity at V02 max was a very reliable guide to an athlete's performance at distances longer than 5k-- that it tended to be a kind of universal measure of a distance runner's basic ability.

As coaches had discovered in the early part of the last century, running at or close to this speed ("racing speed") enabled a distance runner to become faster over time. As with lifting progressively heavier weights to increase muscular strength and power, running at speeds close to what Daniels was to call max V02 velocity provoked an over-compensation effect (a.k.a. a training effect) which, over time, increased an athlete's velocity at V02 max and improved his overall distance running performance. Theoretically, the more running an athlete could do at his V02 max velocity, the faster runner he would become. The catch, however, as Daniels and the other coaches of his era had begun to discover, was that the ability of an athlete to perform bouts of running at V02 max tended to be rather strictly circumscribed; in short, runners tended to become excessively tired and mentally stressed if they attempted to perform more than a given amount of work at this effort level. Innovative coaches, therefore, began to experiment with having athletes complete runs of longer (often many times longer) than the 12 mins they could run at maximum aerobic velocity, but at speeds much slower than this speed, which they sometimes referred to as "recovery" runs. And occasionally, depending on an athlete's racing specialty, coaches prescribed much shorter and faster runs with substantial recovery periods. Over time, many coaches began to observe that these sub- and super- maximum speeds seemed to have their own somewhat independent effect on an athlete's max aerobic speed. Coaches operating on the basis of training principles developed by the great Arthur Lydiard, for instance, observed that very long bouts of easy running, in the almost complete absence of V02 max paced running, could improve an athlete's speed at V02 max.

Daniels' "running formula" appears, as I have suggested, as an attempt to make systematic our use of these various super- and sub-maximal running paces by relating them to the physiologically adaptive responses they provoke, and by offering a practical method determining them for each athlete (his "VDOT" system). Daniels "threshold" or "T"-pace is located precisely at the transition from "easy" running, in which no lactic acid accumulation occurs in the muscles, and V02 max velocity, in which lactic acid accumulation begins to steadily increase. The training effect of running at this pace for periods of 20 to 60mins (although usually no more than 40mins is possible in a non-racing situation) is, according to Daniels, a tendency to increase physiological efficiency, or the body's ability to utilize oxygen, which is shown to have a knock-on effect in terms of increasing velocity at V02 max. It is this physiological efficiency, according to Daniels, that helps explain how a distance runner can become faster over time without increasing his V02 max, or why runners with lower V02 max readings can quite often out-race runners with higher measures.

In the lab, Daniels observed that rested athletes could typically maintain this "threshold" pace for about 60mins, meaning that it correlated with race paces for distances of 13-20kms, depending on the speed of the athlete. In his own coaching practice, Daniels tended to prescribe sessions of "t-pace" running lasting at least 20mins and as long as 40mins once or twice a week, with further bouts inserted into specialized sessions for marathoners.

Having more or less understood how to use his other prescribed paces in my training, it was Daniels suggestion that sustained runs at as much as 15 seconds per mile slower and up to 30\% further than I had be doing my own "tempo" sessions that stood out most for me upon discovering his book. This revelation also set me to thinking about my earlier transitional years-- from middle distance to long distance runner. It occurred to me that at various points in my career, such as during my summer of discontent following my dismal final year of high school, I had probably inadvertently done sizable chunks of my "easy" runs at close to what Daniels was calling "threshold" pace. In fact, whenever I had gone more than a week or two without doing structured workouts-- either during break periods or when returning from injury-- I would tend to gravitate to this pace at the end of my runs at least 3-4 times per week (without the stress of regular workouts, I tended to like to run fast much of the time). I had, I figured, probably already been benefiting from this kind of running; indeed, it probably figured fairly prominently in my becoming a long distance specialist in the first place.

All it took from here was to formalize my heretofore informal practice of occasionally running 20-40mins "steady" (the word that appears in my old training logs to describe this kind of effort) by adding it into to my cycle on a regular basis. This I began to do in my early 30s, at which point I noticed an improvement in my consistency in workouts as well as a feeling of greater mastery at racing distances 10k and longer. In addition to the physiological benefits I began to enjoy from slowing down and lengthening my regular tempo runs, as well as doing them more frequently, I improved my ability to relax, concentrate, and control my pace in races lasting the typical duration of these sessions. As a coach, I have seen this kind of running, once mastered, work wonders in improving my athletes' performances at distances 5k and longer. So important and useful have I found this kind of running that I have even incorporated it into my program for younger runners. (In the case of younger runners, however, the immediate "bang-for-the-buck" of tempo running is not as great as for older and more experienced runners, mainly because younger runners are not doing sufficient total weekly volumes to tackle tempo sessions of much longer than about 15mins-- nor should they. It is, however, important that young athletes see this kind of running as a part of the normal training regimen of the serious distance runner, and that they learn to do it properly. Another bout of hard, V02 max running will always improve the performance of young athletes in the short term; but, if they plan to progress beyond the age-class ranks, they will sooner or later have to learn to execute proper tempo sessions-- and I prefer sooner to later.)

I think tempo running remains the most important, yet most misunderstood, form of distance training there is. The most common mistake among athletes and coaches remains that of doing them as undeclared time trials rather than as strictly controlled sessions. Because of their typical length, tempo sessions can become counterproductive if performed at too high an effort on a regular basis. In the pursuit of precise effort management, it is also important to stage tempo sessions on relatively flat terrain with stable footing. It is simply not possible to zero-in on proper "threshold" pace if steep hills or poor footing cause the athlete to work either too hard or not hard enough in a tempo session. The best way of determining one's "t" pace remains to run a race on a flat, fast course lasting about 60mins and use that pace as an average. When this isn't possible, it works to run for 20mins at one's perceived threshold pace immediately followed by an all-out section of 1-2kms. An increase in pace of greater than about 15 secs per km means that the perceived pace is almost certainly too slow, and a failure to increase the pace by more than 5 secs means it is almost certainly too fast. It's also possible to use heart rate as a gauge of t-pace; but, since most people don't have measure of their true max heart rate, and because heart race can be affected by variables unrelated to one's running effort, caution is advised if adopting this approach. In the end, I encourage my athletes to develop an accurate feel for their correct t-pace, so that they can properly maintain this effort when course and weather conditions vary. A useful rule of thumb for monitoring t-pace is a variation of the old "talk-test" for determining easy run pace (i.e. on an easy run, one should be able to carry on a conversation without undue discomfort). During a t-pace sessions, one should be able to speak, albeit only in short sentences and with lots of "recovery." When supervising t-pace sessions, I like to check an athlete's effort level by asking them a simple question within seconds of finishing their workout or section. If they can answer within about 5 seconds, they've probably nailed their pace pretty closely. Finally, a t-pace session does not have to consist of one single bout of running. It is equally effective--and even preferable in warmer weather, for example-- to run sections of 4 to 15mins split by short recoveries (about 60 secs per 5mins run).




