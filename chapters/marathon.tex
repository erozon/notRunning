\chapter{"So, you're a runnner? Ever run a marathon?"}
\chaptermark{Ever run a marathon?}
\textit{Monday, 6 July 2009}
\bigskip

There is not a serious runner anywhere who has not, at some point, had a conversation with a non-runner starting with this question. And anyone who's been at it as long as I have will know that non-runners have always understood running and the marathon to be synonymous, or at least thought that the goal of all runners was to run a marathon-- and furthermore, that any runner who hadn't run one "yet" simply lacked the conditioning to run that far! I regularly share a laugh about this with a close friend of mine who happens to be a two time Olympian and former Canadian record holder. For years she has had to patiently explain to non-running acquaintances, upon their discovery of just who she is (and such is her genuine modesty that people can have known her for years before becoming aware of her accomplishments), that she specialized in distances shorter than the marathon. In fact, I think one of the main reasons she now wishes she had gotten around to trying the marathon before injuries cut her career short is to be spared having to explain to non-runners how a former "professional" runner never managed to run 26 miles during her career! In the past 20 years, however -- the same 20 years that have seen the decline of elite distance running in places like Canada, the U.S. and the UK, about which more in a moment-- the association of running with the marathon seems to have become even more automatic, to the point where many runners themselves (albeit usually newer runners) now think of running only in terms of the 26er.

How has this happened and what does it mean for running as a competitive sport? And, is the association of distance running and the marathon necessarily wrong (should all distance runners aim to try a marathon in their future)?

The coincidence of the mainstream mass-popularity of the marathon in North America and Europe and the decline of running as a serious sport in these places (as evidenced, e.g., by relative and absolute declines in elite and serious recreational performance levels) is, like the simultaneous increase in youth participation and the sharp decline of prime-age competitors in places like Canada over the past 15-20 years, a paradox in need of reconciliation. How can it be that the second "running boom", unlike the first one-- the "jogging" craze set off in North American, many would say, by Frank Shorter's victory in the Munich Olympic marathon-- has failed to produce a corresponding increase in the number of serious competitors and in performance levels? This is particularly puzzling in the case of women, whose new-found interest in beginner's running clinics, and willingness to pay the increasingly steep cost of entering races, have been the engine of this new boom. In the 70s boom, women were also important players, albeit more as ground-breakers at the elite level than as place-fillers. The achievement of parity with men in terms of racing distances (as recently as 1972, the longest championship event for women was 1500m), including the inauguration of a women's Olympic marathon, and the professionalization of road racing for both genders, was in large part the legacy of a storming of the distance running scene by a cadre of remarkable female athletes-- people like Joan Benoit, Grete Waitz, Ingrid Kristiassen, and Rosa Mota, to name just a few-- during the 70s and 80s. As a product of this first boom myself, and as an avid reader of the magazines and books it spawned, I recall the frequency of the "fitness jogger to superstar" story line among elite women road racers in those days. This first boom, it seemed, was an immediate driver-- rather than simply a facilitator-- of elite women's running, or at least road racing, in that relatively fewer of the sport's big female names seem to have been serious, elite competitors in the age class ranks. The first running boom both drew women into the sport and dramatically raised the level and depth of performance standards, precisely as one would expect.

The paradox of running's second boom where performance is concerned can be explained, I think, in terms of its principle drivers-- the urban road race (often a marathon) as charity and/or tourist "event", and the "learn to run" business-- which often operate in symbiosis. During the first boom, road races were principally athletic contests. The most successful of these early races-- the Peachtree 10k in Atlanta, The Boilermaker 15k in Utica, the Bloomsday 12k in Spokane, the Bay To Breakers in San Francisco and, of course, the New York City Marathon-- certainly managed to attract significant numbers. In their early days, however, these races were first and foremost footraces, rather than community fitness/charity "events". Participants, whatever their age or gender, attempted to race them rather than simply complete them, and it was considered a token of failure to be reduced to walking at any point. Today, by contrast, a significant portion of the fields in these and other races aim only to cross the finish line under their own power, and often actually plan to walk significant portions of the course. This shift is amply documented in the vastly increased average finish times for almost all major road races today.

Some of the difference between then and now can be explained simply in terms of changing demographics-- the age of the average road racer, after all, has increased along with the median age in general in all developed societies. Some of it, however, is the product of a specific kind of marketing by these and other smaller races. And here is where the "learn to run" business enters the picture. Where the exploits of high profile distance runners such as Frank Shorter, Bill Rodgers and Joan Benoit tended to drive the growth of running in the first running boom, the second one has been propelled by figures like Jeff Galloway, John Bingham and, in Canada, Running Room founder John Stanton, whose stock-in-trade has been to encourage would-be runners to enter the sport with the aim simply of completing races-- i.e. without particular regard for finishing time or place. The "learn to run business", which now extends beyond the Galloways, Binghams and Stantons in the form of local store-based clinics and trainer-led operations, also tends to encourage race participation, even up to the marathon distance, for runners with only a few weeks or months experience (which is part of the reason it must discourage concern for finish time and place). The potential of these outfits to provide a steady supply of participants willing to pay increasingly high fees to participate in road races has not been lost on the promoters of road races themselves. The bigger races now routinely contract the best known advocates of the "complete not compete" movement (if they can actually afford them!) as speakers at clinics and expos as a means of both building their numbers and servicing race participants who are by now far more likely to have heard of John Stanton than world record holder Haile Gebresellasie.

From a strictly public health perspective, there is little to find fault with in this arrangement. Formerly inactive people are encouraged to become physically active, and everyone involved in the enterprise-- from race organizers and sponsors, to charities, local communities, businesses, to race participants themselves-- is receiving value for their time, effort and money. From the point of view of running as a competitive sport, however, the "second boom" model of running-- i.e. as strictly a fitness pursuit, centered around simply completing races of various distances at any speed and with as little preparation as possible-- offers very little. And it is not simply a matter of recognizing race winners-- most races do, and many still offer prize money (although, interestingly, far less than 20 years ago in relation to the dollars generated, and once inflation is factored in). The problem is that the shift of emphasis from "competing to completing" is founded on the assumption that serious training and racing are beyond the powers of ordinary people, and even perhaps dangerous and unhealthy, if one listens to some advocates of the new approach. Furthermore, over time, the idea that road racing is something to be done primarily for health maintenance-- symbolized by the "fun run"-- has made it far less attractive to younger runners, if young people are even aware that road races are serious sporting events at all, as distinct from the casual exercise their parents or even grandparents might do on a Sunday morning. And it's not a case of the "completer" ethos being opposed to running for extrinsic rewards. Completers are not discouraged from flaunting their accomplishments as such. The emphasis on simply completing the race is really a matter of setting the bar far lower than need be, encouraging new runners to settle for much less than they could accomplish, and failing in the process to offer them the deeper rewards associated with realizing one's full, long term athletic potential. To be critical of the completer movement in defense of running as a competitive sport is not to defend elitism against populism; quite the opposite. It is to affirm the potential of the average runner, regardless of her basic ability, to train longer and harder; and, in the process, to deepen her experience of running.

The association of running with the marathon has thus become more automatic in the wake of running's second boom simply because running is now more than ever associated with completing races, and the marathon is still the longest-- and therefore the most challenging-- race the average person has ever heard of. Increasingly, new runners are encouraged to think of their running journey as more or less complete once they have managed to get themselves from start to finish in marathon. All that remains is to repeat the accomplishment, and perhaps to seek out new and more stimulating venues in which to do so. I read the trend towards ultra running, Iron Man triathlon, and "adventure racing"-- still relatively small in comparison with the marathon at this point-- as a logical extension of the "completer" ethos. If running is now all about simply completing the longest distance possible, why stop at 42 kms?

The failure of this second running boom to produce a corresponding increase in elite and serious recreational performance is equally simple to understand. The prevalence of the completer ethos is far from the only reason fewer people now see running as a competitive sport. As referred to in earlier posts, there's also the effect of the contemporary youth development model (in North American at least) that has, I believe, contributed to a marked reduction in the numbers of serious adult elite and recreation runners. Nevertheless, the de-sportification of running that has occurred as a result of transformation of road races into exercises in mass fitness and community spiritedness, aided by the "learn to run" business, has, over a number of years, contributed to the decline of serious competitive running in places where this phenomenon has been most pronounced. These drivers of running's second boom have not necessarily reduced the numbers of serious runners in absolute terms, but they explain why running as a sport has failed to thrive in proportion to the sheer volume of new participants.

Finally, what about the question of whether or not all distance runners should, as a matter of course, aim to run a marathon at some point in their careers? Is the average person, after all, completely wrong in assuming that long distance running and the marathon are synonymous? If a runner is only interested in having a quick and conversation-stopping answer to the question/challenge referred to in the title of this post, then I would say: yes, do 3 or 4 longer runs, enter a race, get yourself through it in one piece, and move on. If, on the other hand, a runner is fine with looking like a failure in the eyes of his or her non-running family and work associates, and is content to be the best possible shorter distance runner he/she can be, then I would say not to bother. For those runners intrigued by the idea of attempting a marathon, I would offer the following advice. Assess your aptitude for this longest of serious racing distances by gauging your body's response to longer, easy runs. If you find runs of 2 hour plus generally disagreeable, either because they make you excessively sore or because they significantly impair your ability to complete your other workouts at a reasonable level, then attempting a marathon build-up and race may not be for you in the long run. Not every runner's body is suited to the marathon, just as not every runner's body is suited to the 100 meters. Approached seriously, the marathon is an extreme event, and only those with high aptitude for handling its special rigors have a great likelihood of success. And when the risks of failure are not only a disastrous race day performance, but weeks of fatigue, soreness to the point of injury, and generally un-enjoyable daily running, then the decision to attempt a marathon should not be taken lightly, even by those who have enjoyed great success at intermediate distances. Finally, even for those with obvious aptitude, I would not recommend attempting a marathon until the 2nd of 3rd year of serious training for road racing, and not before the age of 22.

Without a doubt, training for and racing the marathon is a classic test of the distance runner's mental and physical wherewithal. And, when it goes according to plan, there is perhaps no greater sense of accomplishment than crossing the line in a marathon race, considering the sheer number of variables to be successfully managed. The pursuit of this special high, however, is never worth the cost of destroying one's enjoyment of training for and racing distances for which one might be better suited. While possessed of a special aura and cache, the marathon is, nevertheless, and from the point of view of the running body, simply another road race. And, in the end, it remains just as significant an accomplishment-- arguably an even greater and more satisfying accomplishment-- to perform consistently well for a number of years at the shorter distances than it is to claim one or two great marathon successes, particularly if the price paid is a body no longer able to enjoy the simple pleasure of daily running.





