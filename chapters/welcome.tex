\chapter{Welcome to Physi-KultRunning.com}
\chaptermark{Welcome}
\textit{22 January 2009}
\bigskip

Welcome to Physi-KultRunning.com! I intend to use this page for several purposes: to keep the site current; to communicate interactively with prospective Physi-Kult athletes and other interested parties; to develop an ongoing compendium of answers to basic running questions; to support the “results” section of the site with anecdotal reports on the training and racing activities of Physi-Kult athletes, especially those of the senior and masters elites, Dylan Wykes, Emily Tallen, Rejean Chiasson, Paula Wiltsie, and me; and, finally, to offer my views on a range of issues pertinent to the sport. I hope to offer frequent updates, so stay tuned.

I’ll kick things off with a brief (although still far longer than anything else I’ll offer in this blog!) account of my own journey from schoolboy runner to senior elite, and now to elite masters competitor and coach, in the hope that it will allow some greater insight into my overall approach to the sport.

\bigskip
\textbf{Skinny Little Suburban Tarzan:}
\bigskip

I should probably begin by giving some credit to the Walt Disney Productions Corp. for inadvertently providing the tiny spark of interest that would within a few years, and with the help some crucial intervening factors, become the flame of my passion for distance running. For had I not, at the age of 10, seen their now long forgotten (and deservedly so) cinematic twist on Tarzan story, The World’s Greatest Athlete, in which a blond ape-boy, who, while running at freakish speed alongside his wild animal companions, is spotted by a pair of safari-ing American college track coaches, who then proceed to plot his capture and transformation into a points-spinner for their school team, I would not have even been certain what “track and field” was when I heard an announcement for tryouts at my new primary school; and, I certainly would not have thought of the sport as potentially cool, even sexy, had it not been for the movie’s repeated slow-mo shots of the muscular monkey-boy running, jumping and throwing. So it was with Disney-supplied images of myself sprinting powerfully and jumping preposterously long and high, all to the amazement and longing of the girls in my 6th grade class, that I did respond to that announcement and did discover the sport of track and field at the age of 11—all 80lbs and 4'10'' of me.


Predictably, my attempt to make the team in the glamour events depicted in the movie—those requiring speed and power— ended in failure. I did ultimately make the team, however, but only by “winning” the only event no one else seemed interesting in contesting-- the 800m “dash”. Eight hundred meters was, without a doubt, the furthest distance I had ever run without stopping, or anyone I had ever met had run in one go. (This was, after all, small town Canada in the early 1970s, where the typical “long distance runner”, if there was even one to be seen, was a skinny, bearded, professor-type in skimpy shoes, a dingy, plain t-shirt, and yet to be fashionable dark socks —the kind of fringe figure my ex-weightlifter father dismissed as mentally unbalanced. And unmanly, too, I was to understand!) When later that month I managed to win this longest of events at our local inter-school meet (no doubt against a field of other “default” school qualifiers), I was hooked-- at least partially. Like most Canadian boys circa 1974 (and still today, I suppose), I continued to play hockey and fantasize about NHL stardom. I went out for track again the next two springs, however, even becoming, by the 8th grade, pretty competent in at least one of the glamour events—the high jump. After sprouting to a towering 5’, 4” and spending countless hours in the summer between grades 7 and 8 perfecting my technique (using a bamboo pole held up by my father’s portable squat racks and a pair of old mattresses as a pit) I claimed the extremely rare 800/1500/High Jump triple victory at our school board championships! My High Jump Summer was also, and not by coincidence, the summer of 1976—the summer of the Montreal Olympics, and of Canadian Greg Joy’s Silver Medal in that event. It was also, however, the summer of Walker, Juantorena and Viren.


The athletic exploits, and the media attention paid to them, of John Walker, mile world record holder, 1500m gold medallist, and bona fide rock star of middle distance running (he of the long mane and beaded necklace), Cuban Alberto Juantorena—El Caballo—unheralded winner of the unprecedented 400m/800m double, and Lasse Viren, who would join the ranks of the demigods Nurmi and Zatopek by capturing his second consecutive 5000m/10000m double gold, was more than sufficient to dispel any lingering doubts about the status of distance running as real sport; it was, I now knew, perhaps the most real and exciting of Olympic sports, and not at all the sad preserve of those deficient in Tarzan-ian speed and power. Although it would be another 3 years until I abandoned other sports and began training year-round, I date my spiritual birth as a distance runner to that summer.


\bigskip
\textbf{A Little Bit of Big-Time Track:}
\bigskip

As my home town was to remain for a few more years a relative backwater when it came to track and field facilities and competent coaching for fledgling distance runners, things could well have ended there for me. But, as luck would have it, a tiny dribble from the stream of big-time Canadian track and field, which till then had flowed mainly through south-western Ontario, was to trickle east in the form of Brad Hill-- disillusioned big city businessman and former East York Track Club teammate of Canadian international distance stars Bill Crothers and Bruce Kidd. In 1977, Brad, then 35, had relocated to Kingston to escape the manic lifestyle of the ladder-climbing management exec, and to pursue his deferred athletic dreams by trying his hand at masters track and field. I met him through my best friend, a fellow Montreal ‘76 convert to the church of track and field, when Brad spotted him training on his own in a local gym and recruited him as a sprint and weight-training partner. Brad began coaching me in the dead of winter, 1978, after I had achieved modest success training on my own through my first year and half of high school. Under Brad’s guidance, I would eventually make the jump from local to national calibre age-class middle distance runner.


What I was ultimately to receive from Brad was, however, much more than a set of decent track P.B.s. From Brad I learned, first, that it was permissible, even required, to aspire to the highest level in the sport no matter how or where one began, or what the opinion of others. Having known personally two of the best runners in the world, Brad was profane in his impatience with people who insisted on measuring success in local terms-- in terms of regional or age-class accomplishments. (Actually, Brad was just plain profane, which was a big part of his charm as far as we were concerned!). Everyone, Kidd and Crothers included, began their athletic life as a “local”, a “nobody”, he reminded us; but, those who, like Kidd and Crothers, would reach the highest levels were invariably the ones not afraid to imagine themselves one day competing on the biggest of athletic stages, and who trained in accord with that vision. He also said that, no matter where we ended up on the competitive pile, we could only say we had “won” if we had maximized our personal potential, and we could only do that if we aimed beyond our local horizons. For me, as for most of the young athletes in our little group, it was this last piece of advice that was to be the most important (none of us, after all, could choose our genetic endowment). While I was ultimately to fail in my tangible athletic quest—to make an Olympic team— I am certain that I would not have gone as far as it did, enjoyed the journey as much, or taken so much of lasting value from it, had Brad Hill not passed on to me his disdain for thinking and believing “small”.


The second of Brad’s gifts to me was vivid instruction in the limits of my capacity to train! As a high school senior, I ran further and harder than anyone I competed against, and I knew it. So, while the immediate results of my efforts were mixed, I was to enter the wide-open world of adult athletics with both a fantastic aerobic foundation and, more importantly, a realistic sense of what it would take to prosper at the national and international elite level, should I turn out to have the kind of body that could take me there. There is today, as always, much debate about the appropriate amount and intensity of running for developing runners. In part as result of my own experience, I am of the view that athletes 16 years and older who have decided they would like to commit themselves to the sport should be introduced, albeit in a gradual and intermittent way, to the kinds of training they will be required to complete in order to continue to develop in the adult ranks (younger athletes, I strongly believe, should always remain non-specialists, and should never train or compete at distance running on a year-round basis). For most young athletes, the introduction of this kind of training will typically not happen until they reach their first year of post-secondary competition; but, for some, it may be appropriate to introduce more challenging training (including weekly volumes in access of 110kms and structured sessions as long as 8kms) in the final year and a half of secondary school.



\bigskip
\textbf{Athlete (Temporarily) Interrupted:}
\bigskip

My own entry into post-secondary sport was to be the beginning of what I would look back on as my “lost years” for development. While my university running program was far from the worst on offer (in fact, we experienced a degree of team success, albeit in the relatively less competitive realm of Canadian university athletics), I left it knowing little more about myself as an athlete, or how to approach the problem of long term development, than I had going in. Our coach, a volunteer with little or no top level experience as either an athlete or a coach, certainly meant well; but, I would ultimately fail to receive the kind of guidance required to make the transition from talented junior middle distance runner to elite long distance performer, remaining instead stranded in a kind of no-man’s land between my old specialities and what any experienced and knowledgeable observer would have known should have been the focus of my training.


When, as a coach, I observe and evaluate young runners today, I sometimes think about how I would have approached my younger self. By 19, I had run national-level times over 800m, but had also been one of the top cross country runners in my cohort. And, before my 20th birthday, and in my first serious road race, I had run shoulder-shoulder with some of the best Canadian senior athletes, recording a P.B. of 29:15 for 10k. Were I to encounter an athlete like this today, I would be at extreme pains, first, to make him aware of his great long term potential, and second, to impress upon him the importance of shifting his long term training and racing focus to the 5,000m and 10,000m, and perhaps even the marathon, well before the age of 23, and certainly before the age of 25, which is when I finally began to train properly for these events. Good middle distance speed combined with the early ability to run well at distances beyond 3km is the most straightforward indicator of long term elite potential in teenage athletes. As it happened, there was no experienced eye within range to make this simple assessment, and to impress upon me its significance. So, I continued on as I did-- in love with the sport, and capable of prodigious efforts in training, but never entirely sure how to channel that passion and fortitude.

A good part of my motivation to work with younger athletes today originates from my desire to become what was lacking at this crucial stage in my own development: an experienced identifier and motivator of potential senior elite athletes. I was to eventually persevere in the sport and achieve some “late bloomer” success; but, I have often wondered how much more of my potential I could have realized had I, at the age of 19 or 20, had the luck of discovering close at hand someone like my experienced adult self. And, I have wondered, if I had survived to achieve a degree of success by dint of stubbornness alone, how many young runners of my generation with similar potential were deprived completely of satisfying adult careers for lack of expert guidance? Indeed, how many more today are falling by the wayside for lack of a discerning eye and proper encouragement? No coach can be everywhere, or help everyone; but, I have pledged to do my best to ensure that the young athletes with whom I come into contact have the help they need to realize their full potential in the sport, even if this amounts merely to letting an athlete know that he/she has genuine long term potential, and offering an emphatic word of encouragement. My philosophy is, as I state briefly on the main page, that the sport of distance running is more satisfying the more vigorously it is pursued for its own sake. An athlete of great or very good competitive potential, however, is doubly deprived of a satisfying athletic experience when he or she is left without the basic means, in the form of expert guidance and encouragement, for the full realization of his or her inherent abilities. Distance running is a relatively cheap and therefore accessible sport; expert guidance and encouragement, therefore, is often all that is required to make the difference between failure and disillusionment on one hand and success at the highest levels on the other. I coach in part because I am haunted by the thought of young athletes of promise leaving the sport before they can reap its greatest rewards, and all for lack of this one simple but crucial ingredient for long term success.

\bigskip
\textbf{Beginning to Exhale and a Shot of Jack Daniels:}
\bigskip

As for my post university years, these can be divided into two phases: Before and after asthma treatment. Although my knowledge of my own physical aptitudes, and of training and racing in general, was to increase exponentially throughout my mid-20s, mainly through trial and bitter, confusing, error, it was not until my asthma diagnosis at the age of 26 that I was able to establish the high-level consistency that would become my trademark as an athlete in later years. The years until that time were mark by flashes of brilliance followed, sometimes in the same week, by inexplicable losses of form. Although I would never attain complete mastery of my condition (I struggle to this day with seasonal breathing difficulties), I was able, through the use of inhaled medications of increasing effectiveness, to find a stability in my training that would enable me to approach long, hard workouts and races with much greater confidence than before. Thus I moved into the latter phase of my career— that from my mid 20s to my mid-30s—with a confidence born of struggling with and overcoming two formidable obstacles—my late start in shifting to training for the long distances, and my often demoralizing airway reactivity. The first highlight of this phase was, I would say, my televised victory against a very strong field at the 1993 Timex National 10k Road Championships in Ottawa, the first of my now 5 Canadian titles (3 open and two masters). This is a moment I will always cherish (and, luckily, will be able to cherish for years to come, thanks to my copy of the broadcast, which, as a fantastic added bonus, was commentated by legendary Canadian track and field broadcaster Geoff Gowan!).


This long second phase of my post-collegiate career was characterized by increasingly consistent racing success, and also by the deeper study of training science and method. Since the age of 25, I had been advising (“coaching” being probably too formal a term at this early stage) some of my training partners, who would go on to achieve performance breakthroughs and racing consistency similar to my own. It was not until my early to mid 30s, however, that I began to explore in a critical and systematic way the literature on training for distance running—a loose-knit body of laboratory-based research on exercise physiology and anecdotal studies on training authored by leading athletes and coaches—and to think about offering my own thoughts and experiences to other athletes in more formal way. Of the studies I discovered during this period the one I found to be the most exciting and potentially useful was Dr. Jack Daniels compendium of basic training science and experiential knowledge entitled The Daniels Running Formula. Here, I found what I considered to be a concise distillation of the most important and well established principles of distance training—those related to the specific adaptations gained from training at different speeds-- combined with insights derived from years of direct, high-level coaching experience. The voice of Daniels Running Formula was/is that of both the trained scientist and the experienced practitioner: professorial and precise, yet humble, circumspect, and open to novelty. In Daniels, I saw the perfect embodiment of the art/science hybrid that characterizes the state of knowledge in field of endurance training itself. But, for me personally, the training philosophy and principals elucidated by Daniels also seemed to precisely validate, and thereby bring to full realization, the approach to training that I had been groping towards in my own practice as self-coached athlete. The essence of this approach was the importance of very long term, stable, incremental development through the use of strictly controlled effort levels in training-- strictly controlled, that is, in terms of the known physiology of stress adaptation in distance running. In contrast to Daniels’ clearly elucidated principles, too much of what I had read and seen in practice seemed, while perhaps interesting and sometimes even entertaining in itself, excessively narrow and idiosyncratic—more a reflection of the particular experience and charismatic personality of the author than the bases of a broadly applicable program of training. In Daniels, I saw the outlines of a framework for making training more systematic, but which also allowed, even required, a range of creative, day-to-day adaptation on the part of the coach “on the ground”.

\bigskip
\textbf{The Long Run:}
\bigskip

My career as an open elite runner was to end in 2000, just before my 37th birthday, with an attempt to qualify to represent Canada in the marathon for the Sydney Games. I brought all of my accumulated fitness, experience, and knowledge to bear on the problem, but was to come up just short. Scuppered by a last minute virus in my fall attempt at the distance, for which I was well prepared, I had to resort to plan B, which was to train through the winter of 2000 and make an attempt in the spring of the year. Training on the pavement—a necessity in the Canadian winter-- was to take its toll, exacerbating a low-back problem that had been developing since the late fall, and I would struggle to a still respectable 2nd place at the National Championships, losing by 18 seconds to defending Olympian (and long time friend) Bruce Deacon. My Olympic shot was actually my second foray into marathon training—I had tried without success five years earlier—and I was to gain valuable knowledge, if not my actual competitive goals, in the process. In the end, I think I went a good distance in “solving” the event from a training perspective, at least in the sense of knowing how to avoid the more obvious pitfalls, and would perhaps have achieved my goals had I only been a few years younger when I started in the event. Since 2000, I have applied my knowledge and experience to the design of marathon programs for other athletes, and with very satisfying results. I have now steered athletes of both genders, a wide range of ages, and varying degrees of running and racing experience, to very successful and minimally painful(!) marathon completions. The marathon, when raced intensely, is the most challenging and potentially physically destructive of all running events. It is precisely because it is so difficult and risky, however, that its mastery is so thrilling. Because the event presents such an extreme coaching challenge—performance must be maximised on single day, with injury and illness an ever-present threat-- the creation and supervision of marathon programs has become my favourite of projects, whatever the level of performance or experience of the athlete in question.


My running life following retirement has been largely unplanned. It would take a year or so to realize that the back troubles I had developed in the course of my marathon training would prevent me from competing at my accustomed level for what remained of my 30s (although I would manage a 3rd place finish at the National Cross Country Championships in the fall of 2000, a full 18 years after my first trip to the X-C national medals podium). After a brief spell of “retirement” from serious running (which ended up consisting of one day off per week and a reduction of my daily quotient to 45mins from my typical 60-90mins). This brief respite allowed me to make a complete psychological break from the notion of competing at the open elite level again, and gave my body time to heal. It was during this interval that I considered returning to somewhat serious training in preparation for masters (post-40) competition. As my 40th birthday approached, the decision became an easy one: I had missed the thrill of the chase and would return to serious competition, beginning with a kind of “farewell” tour to all of my favourite road racing events from years gone by, and trips to some of the venerable North American races I had read about throughout my career but had not found occasion to run as an open elite—The Bloomsday Run in Spokane, The Peachtree Road Race in Atlanta, The Falmouth Road Race in Cape Cod, and the Boilermaker Road Race in Utica NY. I completed my “tour” in the summer of my 45th year, with a master’s win at the Boilermaker 15k.


In general, my competitive record in the first master’s age class, the 40-44, was a highly successful one. In five years, I would lose only five times to a total of just 6 athletes, four of them Kenyan by birth. I won over 20 races in the age category, most of them in regional or national level events across the U.S. and Canada, including 2 national cross country titles. Although I never equate my master’s success to what I achieved as an open runner, I have found my master’s racing to be immensely enjoyable. In master’s racing, there is the same thrill of competition with little or none of the old self-doubt and fear of failure. There is also a spirit of camaraderie and humour that pervades masters competition. Because of shared respect for the difficulty of the activity itself, competitive distance runners are famously friendlier and more supportive of one another than are competitors in most other sports, particularly team sports. In masters running, the humility that comes from the common experience of declining physical capacities only deepens this sense of community born of shared adversity.


Finally, it is during these post-retirement years that I founded and gradually expanded my running group-- Physi-Kult. The group began with just 4 randomly self-selected athletes—all women in their 30s or 40s, and only one with any real prior experience with distance running. The depth of satisfaction I felt in seeing these women become “real” runners, and come to love serious, competitive running (in spite of their sometimes overwhelming nervousness!) surprised me, as did their rates of improvement. From nearly scratch, and within less than two years, these four women were easily among the best local road racers, and had even begun to win their age categories in big city regional races throughout Ontario and Northern New York State. The youngest, a 32 year old, even found herself on the edge of the elite category before she left the sport to have her first child. It was thus an easy decision to expand the group to its current size in 2003. With the recent addition of the senior elite and high school components, Physi-Kult has become perhaps the premier small, all ages club in Canada. In 2008, with fewer than 20 athletes competing, the club managed to finish in the top 5 among some 30 clubs in the largest provincial cross country championship in country. The master’s men’s team, consisting of 3 runners over 45 and one 50 year old, then went on to win the title at the national championships. The rapid success of this small group in this one corner of the country has convinced me both that there is more distance running ability lurking within the ranks of teenage soccer and hockey players, and of casual fitness runners, than perhaps any one imagines, and that, with the right program and sufficient encouragement, that talent can be mobilized to the great benefit and enjoyment of all concerned.


My plan for the foreseeable future is the continue masters racing, both for the fun of it and to stay in touch with the process of training to race, which I think will enable me to continue to grow and develop as a coach. My thirty-plus years of almost total immersion in the culture and science of this sport has turned me, for better or worse, into a living storehouse of knowledge concerning how to optimize one’s lifetime performance-- and, I think, enjoyment-- through this inherently simple activity. For what remains of my career, it is my intention to make this body of knowledge and experience as widely available as my time will allow. I see physi-kultrunning.com as a potentially very useful vehicle in that endeavour.

\bigskip

Enjoy the rest of the blog!

