\chapter{Daniels Primer \#3 -- The Uses and Misuses of the VDOT Tables}
\chaptermark{Daniels \#3}
\textit{Monday, 1 June 2009}
\bigskip

Stepping away for a moment from the spring racing frenzy, I thought I'd offer another Jack Daniels primer-- this one on the uses and misuses of his revolutionary "VDOT" performance and workout tables.

Runners tend to find Daniels VDOT tables to be the most interesting part of his book, Daniels Running Formula because runners tend to like numbers-- in particular, ones by which they can begin to compare their own performances across different distances, and against those of others. But, what is a the real value of Daniels' famous tables, how were they developed, and how might they be used (or misused) in the training process?

The value of Daniels' VDOT tables (I'll leave it to interested readers to explore the precise origin of the name "VDOT", which is explained on p. 51 of the the second edition of the book) is that they enable an athlete and coach to more precisely regulate effort levels in workouts based on current fitness, so that the athlete may perform the optimal amount of work in a given session. In short, VDOT tables enable us to support or replace our existing, and usually less precise, terminology for describing effort levels-- terms like "easy", "hard", "comfortably quick", etc. Daniels developed the VDOT tables during the course of his lab research measuring oxygen consumption among trained runners at maximal and sub-maximal speeds. For each of his subjects, he determined the speed (velocity) traveled at maximum oxygen consumption (a vV02, as he labeled it). He also discovered that the average athlete could, under lab conditions, travel at this speed for 10-12 minutes. From here, he was able to measure how much an athlete would have to reduce his/her velocity in order to run for longer than 10-12mins. Daniels was then able to work backwards from actual race performances (since most athletes complete given distances within a fairly narrow range of time) to determine the approximate percentage of an athlete's maximum "aerobic speed" he/she would typically operate at in races of different distances, regardless of his/her actual max V02 relative to others. This enabled him to establish a set of real world training speeds (i.e. measured in seconds per km or mile) corresponding to an athlete's maximum effort as determined by his/her actual racing performances. It also enabled athletes and coaches to predict, with surprising accuracy, an athlete's probable performance at one distance from his/her proven performance at another (a big part of the allure of the tables for runners and coaches). Daniels' VDOT tables were revolutionary because they enabled coaches and runners to more precisely regulate training efforts without recourse to expensive lab tests to determine max V02 and running economy; whatever their actual laboratory determined levels, athletes could be reasonably certain over time that their race pace for distances of 3k to 5k (10-12mins) corresponded to their maximum aerobic effort (vV02, in Daniels' terms) and could set their various workout paces accordingly. Thus, armed with Daniels' VDOT tables and his (and others') research on the different physiological adaptations developed from training at difference percentages of maximum effort (from "easy" running to running at MV02), athletes and their coaches could more accurately determine the optimal paces for their various daily training efforts, thereby avoiding the pitfalls of going either too slow or (more likely) too fast in a given session.

Used properly, Daniels' VDOT tables are a huge boon to the training process. In enabling athletes to more precisely regulate training speeds based on actual current fitness-- rather than on, say, an athlete's personal bests, or on an athlete's goal performances-- the VDOT tables helped athlete and coach to set sustainable effort levels for training, thereby avoiding blown workouts, over-training, injury, and subsequent failure to improve. But what, then, are the possible misuses of the VDOT system?

As Daniels' himself points out, the VDOT measurement is only effective if the conditions under which a given performance was achieved closely approximate those of the average training environment. In other words, a performance achieved in cool and windless conditions will not be of much use in setting speeds for a workout to be done in hot and windy conditions. Here, the VDOT may be a reasonable guide, and is certainly better than nothing; but, if training conditions on the ground do not match the racing conditions in which a VDOT performance was achieved, strict adherence to VDOT speeds will lead in the long run to over-training and failure.

Another possible misuse of the VDOT system is to assume that all athletes should be able to perform every kind of workout according to their best VDOT racing performance. The VDOT is an excellent basis on which to set workout paces for more experienced and well trained athletes. I have found, however, that VDOT paces for longer sessions are often not consistently attainable for younger or more inexperienced athletes. Younger athletes in particular often find it fairly easy to work at their VDOT-prescribed paces for MVO2 pace or faster, but are often completely unable to do the corresponding paces for longer "threshold" or "tempo" sessions. In fact, I've found that the less experience and the lower the overall weekly training volume for a given athlete, the more unreliable are the VDOT paces for these longer sessions. Less experienced and/or lower volume runners often simply can't approach the prescribed speeds for these sessions. Sticking to the VDOT paces in these instances turns what should be a manageable session into a long and unduly burdensome weekly time trial.

Finally, it is generally not advisable to use speeds attained in workouts to predict racing performance (the "reverse VDOT" approach). Race performances are a great guide to proper workout efforts, but workout "performances" should rarely if ever be used as a guide to future race performances. In other words, athletes should not assume that because they can complete a workout at VDOT level corresponding to a particular race performance that they are ready to race at that level. A particularly good workout may well be an indicator of a great performance to come, but things are rarely that simple. Oftentimes, athletes will feel they are able to handle their VDOT-prescribed workout paces without undue stress, and may be able to push well below these paces if called upon in a one-off session. The demands of workouts, however, are somewhat different from those of races, and many athletes are able to do things in workouts that they are not quite ready for in races. Likewise, some athletes find their prescribed VDOT paces quite difficult on a day-to-day basis, yet are able to routinely reproduce the corresponding performance in races. Simply put, some athletes train better than they race, and vice-versa.

To use the VDOT tables effectively, we have to bear in mind that they are only guides; and that, as such, they are not a substitute for the informed judgment of athlete and coach on the ground. We also need to understand that the VDOT tables are passive measures of current fitness and not guides for how to achieve a particular corresponding performance level. In other words, we don't attain a higher level of performance by attempting to train at the speeds corresponding to that VDOT. If we do this, the result is much more likely to be over-training and injury than better performance. The VDOT system is meant to help us determine the optimal speeds for training at our current level of fitness; improvement comes not from pushing beyond these speeds but from working consistently at them, and from increasing total training volume carefully over time. Thus, as Daniels is at pains to emphasize, we should not attempt to train at a new VDOT level until our race performances have confirmed our readiness to do so, or until our workouts have become very controlled and easy (a period typically between 3 and 6 weeks, according to Daniels).


