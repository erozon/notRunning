\chapter{Raising Runners}
\chaptermark{Raising Runners}
\textit{Tuesday, 2 December 2014}
\bigskip

Like most other sports these days, competitive running has become, overwhelmingly, a sport for children. While a significant (and perhaps growing) number of the adults who make up today's massive road race fields still approach the activity with finish time and place in mind, truly earnest competition in the sport today is mostly found at the age class levels, and in school races in particular (at least here in North America).

And where kids are involved, parents are never far from the scene. This was not always so. The often hilarious lack of understanding or involvement on the part of our parents is a staple of conversation among retired elites of my vintage. Like an athletic version of the famous Monty Python "Four Yorkshiremen" sketch, former elites of a certain age will inevitably attempt to one-up each other with tails of how little their 70s era parents knew or cared about their beloved activity, or how infrequently they ever actually observed them doing it (my own father watched me compete perhaps 3 times in his life, including a high school cross country race, which he proclaimed bored him, because I had won by such a large margin, and because he had only gotten to see me start and finish!). It isn't that our parents were necessarily unsupportive (although, truth be told, many weren't keen in running, some of them vocally so); it's that they did not feel they had to be involved, or to follow our exploits, beyond perhaps buying us the odd pair of shoes, paying our club fee, or asking us how we did when we got home from a meet. For better or worse, parents of my own generation take a wholly different view of their responsibilities where their kids' sporting activities are concerned. Today, if they don't actually coach their offspring, they feel compelled to be present every time they toe the line, from the earliest age group competitions to, in some cases, the end of their collegiate careers. I have been no different in this respect, and not just because I happen to be a runner and coach.

But, because I am a runner and coach, and one with a history of involvement that stretches back to primary school, I've tried to reflect on, and to critically evaluate, the effect of my involvement in my children's running-- both on their personal experience of the sport and on the sport itself. Below are some of my reflections and critical insights.

First, a confession: In spite of my own heavy involvement in it, I never intended to have my kids take up running, and I became a coach of kids by accident (when another primary school parent guilted me into taking over her duties). Being intimately familiar with both the rewards and trials of the sport-- the great sense of satisfaction, well being, and camaradarie on one hand, and the frustration, heartbreak, and deep, sometimes unrewarded, sacrifice on the other-- the prospect of my own offspring becoming serious runners disturbed the parental over-protectiveness that is the burden of my generation-- better if they did decide to pursue something as totalizing in its demands as running (and parents my age all talk about wanting out kids to have a "passion" for something), that it be a thing whose dangers were unknown to me. Unsurprisingly, given the wall-to-wall status of running and fitness in our family (our kids grew up watching my wife and I train daily in any weather, on holidays, on the road, etc., and they had met a couple of dozen elite and former elite runners, including a half dozen Olympians, by the age of 10-- Dylan Wykes being like an older sibling to them), both of my kids would eventually try their hand at competitive running.

And what did I learn from raising runners (and watching others raise the ones I would come to coach)?

\begin{enumerate}
    \item To understand that the sport looks far different to them in the 2000s than it did to me discovering it in the 1970s. To me, the sport was... a sport. The earliest images of competitive running for me were of thin, stern-faced men (and a few women) assembled in tight packs, circling tracks or navigating turf, in pursuit of some championship or other. My bedroom walls were covered in photocopied newspaper clippings and magazine photos of Olympians and top Canadian athletes, all of which were as readily available to me as material on pro sport athletes. Even my own kids, who saw their share of serious runners, and understood that running entailed competition, grew up seeing running primarily as a thing done either by school children, or by adults after their physical prime. By the late 1990s, and thanks to the eclipse of serious running by "fitness" running, kids were more likely to associate running with images of middle-aged people festooned with reflective jackets and ammo-belt water bottles than with that of Reid Coolsaet in his "short shorts" breasting the tape in some road race. Largely because of this, it was, and is, hard to make even the most athletically precocious kid see running as something to be pursued beyond the age of 22, at the oldest. The vast majority can't conceive of running as a thing anyone would do for life, even if only as a serious hobby, the way many younger adults approach hockey or golf. Serious, senior running (elite or recreational) is all but invisible to kids today.

    \item That the rigours of even casual, entry-level training (and, in particular, going out alone for easy runs) are usually far beyond the daily experience of today's kids-- much further than they were to kids of my generation, who often had to make their own fun outside, on foot. Even those most keen on racing will find it strange and daunting to navigate long distances (i.e. 20mins or so) alone and on foot through their neighborhoods, since they rarely spend any time alone outside for any reason, and since most never walk anywhere more than a few hundred meters. Today's kids will gladly spend any amount of time at a formal "team practice", but they tend not to like to simply "go for a run" alone-- the basic component of any running program.

    \item That innate talent does not drive interest among kids. Many kids with real aptitude for the sport would and do choose to warm the bench in a team sport rather than train for distance running. Even kids who have high levels of early success, or show great potential to improve, can't be expected to be any more likely to embrace running than other kids. Because they have a hard time envisioning what making the top levels of the sport might actually look like (and see point 1 above), the prospect of having above average aptitude for it does always excite them-- at least not in the way success in team sports seems to (and this is not just a matter of fame and money, because most kids realize at an early age that these things are not in the cards for them in any sport). I am no longer surprised when very talented kids quit early and without explanation, often to play sports for which they have less natural aptitude. Genuine enthusiasm for the sport (include knowledge of its stars and finer points) is rare among kids, regardless of their physical aptitude for it.

    \item That very early age class success is usually a problem to be overcome rather than an advantage to be enjoyed for those kids who experience it. This was not a revelation to me, having for years watched these sorts of kids come and go, and be replaced at the top ranks by later-starting athletes, often with backgrounds in other sports. But, my experience with my own offspring provided me with an intimate picture of just how early success can create obstacles to be overcome, rather than opening doors. My son-- a confirmed non-jock with no real sports experience of any kind-- somehow found himself winning a provincial high school age-class cross country championship (and many other races besides) after only a smattering of training. Instead of feeling proud of his accomplishments and looking forward to more, he took them entirely for granted, yet also began to feel the burden of expectations that his demonstration of talent seemed to impose on him. Knowing that he might have to work much harder to continue winning, and that anything short of winning might be seen as failure, he initially chose to abandon the sport, only to return (very tentatively) two years later. Meanwhile, his sister, having won nothing of note in the early age class ranks, and having run only slightly above average times in high school, fell in love with the challenge of the sport, and went on to make two U19 national teams . There were and are the inevitable differences in personality between these two young athletes, but there are also, of course, a great many similarities (including the same coaching and parenting). The main difference between them would seem to be their early experience of the sport-- immediate success on one hand, and a degree of early failure and frustration on the other. Whether it is from heavy, systematic training at early ages, or the luck of the genetic draw, early success in running does not seem, as it might in sports involving technical skill, to foretell future success, or even who will go on to love the sport for its own sake; or, if it does, it might well do so negatively. In any case, deliberate attempts to have kids win in the early years-- whether it be from actually targeting success through training ahead of the curve, or from simply introducing precocious kids to very high level competition that they might not otherwise even know exists-- is generally to be avoided. It rarely increases kids' enjoyment of the sport in the short term, and it certainly doesn't increase their chances of success in, or enjoyment of, the sport in the long term (as much as it might increase the enjoyment of adults-- parents and coaches-- in the short term).

    \item That, when it comes to kids sticking with and succeeding at the sport, what you do as a parent of runners is more important than what you say. In short, if you're the sort of parent who makes serious physical activity a part of your daily life, regardless of your level of competitive success, or if you're the kind of parent who has shown the courage to take on difficult personal challenges in any area of life, whether this be quitting smoking, losing weight, or learning another language, your children are more likely to make competitive running a life long habit than if you only verbally encourage them to do it. Parents who perhaps used to be avid athletes as children or teenagers, but who now seek the "comfort zone" in every facet of their lives, should not be surprised when their kids, no matter how physically talented, choose to abandon running as soon as success requires real effort, or as soon as the more passive adult pleasures (socializing/drinking, shopping, looking at screens) become available to them without restriction. We teach kids what is most meaningful and rewarding in life by embracing it ourselves, regardless of our age, rather than by lecturing or rule-setting. This applies to running as to any activity that requires sustained effort, patience, and deferred gratification.
\end{enumerate}




