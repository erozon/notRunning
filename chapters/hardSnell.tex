\chapter{Hard Snell: Peter the Great on The Fundamentals}
\chaptermark{Hard Snell}
\textit{Thursday, 24 May 2011}
\bigskip

It would seem that some lessons, though ultimately quite simple, must be relearned ad nauseam. This is often because, as in the case of the revolutionary mid-20th Century Kiwi coach Arthur Lydiard, and what he taught the running world would through the exploits of his champion athletes, their implications can be hard to face. In a recent interview , Lydiard's greatest exemplar, three time Olympic champion Peter Snell, who would later embark on a career as a exercise physiologist, offers us a refreshingly straightforward reminder of that which is fundamental to success in running, from 800m to the marathon-- aerobic conditioning.
http://www.garycohenrunning.com/Interviews/Snell.aspx

For those how have never heard of them, New Zealanders Lydiard and Snell (along with Murray Halberg) took the distance running world by storm in the 1960s by returning a total of 4 Olympic gold medals (Snell's 3 at 800 and 1500 in 1960 and '64, and Halberg's in the 5000 in '60) to the island nation, population 2.5 million. Snell remains a candidate for best middle distance runner of all time. And the revolution wrought by the Lydiard group would continue into the 1970s and 80s, with the Finns adopting his methods with great international success, and a new crop of New Zealanders, led by the likes of John Walker, Dick Quax, and Rod Dixon, claiming world records and Olympic hardware.

At the core of Lydiard's revolutionary approach to distance training was the counterintuitive insight that high volumes of sub-maximal running volume could improve performances at maximal running efforts in the middle and long distances (with the difference in the two speeds being more than 2mins per mile!). This insight was conveyed to a young Snell, then only a national level performer at 800m, in a conversation with Lydiard about how the former might go about reaching the international level. Like many young athletes, Snell instinctively believed that the solution to matching strides with the best in the world would be to improve his maximum speed, creating a "speed reserve" that would make his current race pace feel more manageable, and his finishing drive more potent (Snell's instincts were in part the product of being repeatedly told that he lacked the "top end" speed to win beyond the national level-- a message driven home practically by the experience of being regularly and soundly out-sprinted in the final 200m of races). Instead of attempting to increase his maximum sprint speed through regular drilling at maximal efforts, Lydiard advised Snell to add more long, easier "aerobic" running to his regimen in the months leading up to his competitive season. As Snell relates it, Lydiard explained to him that his inability to run under 1 min 50 in the 800m (the then threshold for international success in the event) had nothing to do with his inability to reach the requisite speed (55 secs per 400, which Snell could do with relative ease), but with his inability to hold that speed beyond 600m. The latter, insisted Lydiard, was function of aerobic endurance and not the difference between Snell's full-out sprint speed and the desired pace-- his so-called "speed reserve". The catch, of course, was that Snell would have to commit himself to a program of daily running of up to 15 miles per day during the non-competitive months-- a difficult sell, when the standard approach for middle distance runners was a staple of shorter track sessions at race speeds or faster. Snell decision to trust Lydiard's then unconventional approach, and to begin to train like almost like a marathon runner for large stretches of the year, would prove history-making. His newfound endurance would very rapidly propel him to the front ranks of his event nationally, then to an upset victory at the Rome Olympics, with a time more than four seconds faster than his pre-Lydiard best. He would go on to defend his Rome title at 800m four years later in Japan, and add a gold at 1500m for good measure. In between, he would claim the world record in track's then glamour event, the mile. In top form, Snell would prove all but unbeatable in the latter event.

Years later, as exercise physiologist Dr. Peter Snell, the legendary Kiwi would offer a theoretical explanation for the revolutionary success of the Lydiard approach, variations of which would go on to become standard practice within the most successful distance running nations, with the metaphor of "aerobic base" (months of easy aerobic running) and anaerobic superstructure (weeks of race-pace interval training) becoming common currency among coaches the world over. Dr. Snell believed that Lydiard's keen intuition (born of his own experiences as a club-level athlete) had led him to discover that so-called "fast twitch" muscle fibers-- those responsible for producing the more explosive contractions that produce top-end speed-- can be coaxed into responding to a low intensity training stimulus once the "slow twitch" fibres that are first recruited at easier training paces had become depleted and exhausted. In other words, what appeared counterintuitive about the Lydiard approach could be explained in terms of the two-for-one training benefit that easy aerobic running could produce, provided the athlete was willing to run far longer than conventional wisdom would suggest. The athlete would still have to spend a period of time practicing race paces on the track; but, with the benefit of the endurance gained from the "aerobic base" training, he/she would be able to attain and hold these speeds for longer periods, and even be able to attain higher speeds when required at the end of races, as a result of fast twitch fibres that had been conditioned to perform better when fatigued.

Whatever its precise physiologic basis, the practical success of the Lydiard paradigm over the past 50 years remains undeniable. (Although confusion is sometimes generated by a failure to distinguish between the Lydiard paradigm and Lydiard's own sometimes idiosyncratic programming-- a distinction that would become one of Snell's own life lessons. See Snell's blunt words of advice on how to adapt the Lydiard method in the linked interview). From East Africa, to Japan, and now to the USA, where a distance running renaissance has saved American athletes from the ignominy now being suffered by the once great British and Europeans, high volume programs of the Lydiard type are the foundation of global success from 1500m to the marathon. Yet, in some quarters, the lesson that to run fast one must first run far remains a difficult sell. Indeed, in the USA itself, where the prevalence of high mileage training in universities and post-collegiate enclaves put American runners in the front pack internationally from the 1960s to the 1980s, this wisdom was temporarily set aside in favour of an approach emphasizing basic speed over endurance (the prevailing wisdom at the nadir of the sport in the USA was that "long, slow distance" makes for "long, slow runners"). The truth that, in distance running, more is indeed more can be difficult for some to accept, both because it is counterintuitive, and because to sustain a higher volume training program is a hard and tricky enterprise.

It continues to seem obvious to the casual observer that, because most races are won with superior finishing speed, the secret to winning is to increase one's sprint speed, rather than trying to become aerobically stronger. Thus, many high school and club coaches will continue to drill their distance runners, season-in and season-out, on "leg speed", "turnover", and running form, while cautioning them against the dangers of trying to run too far. The result is often that young athletes, by the time they reach their university years, have become intimidated by the prospect of running more than 30 or 40 minutes per day; or, they worry if they become too tired to attain personal-best middle distance speeds on demand, 12 months per year. And, in truth, even when approached correctly (i.e. at genuinely easy paces, on soft surfaces, and in the company of partners where possible), completing the kinds of daily aerobic volumes required to promote steady improvement is psychologically difficult for younger athletes (and age 15-16 is probably the minimum age for following a serious training program in distance running), particularly in places where the daily routine of children requires nothing more strenuous than walking back and forth from the car, or playing a ball sport. To engage in this form of activity simply goes against the cultural grain in many ways in contemporary society: it requires a degree of tolerance to discomfort; great patience; inner quiet; and the courage to take short term risks (however minor) in pursuit of greater long term rewards. In the end, it can take years to acquire a taste for the pleasures of long, easy running for its own sake-- a pleasure that mature runners universally enjoy, and that gets them out the door every day, often well past the end of their competitive careers. The general bias towards hard interval work over easy aerobic running today for younger runners (i.e. between the formative ages of 15 and 18) would thus seem to be the result both of the basic misconception about finishing speed that Lydiard sought to dispel, and a concession to the cultural reality that contemporary young people are generally averse to running for up to an hour a day at gentle paces (which I'm convinced they are inclined to see as an activity for older people trying to "stay in shape"!) The line of least resistance in building a youth club or high school team is thus to offer a program modeled after the team sports with which young people are most familiar-- i.e one that meets frequently in large groups for relatively short and intense "practices". Unfortunately, the result of this approach is that, while rates of participation may climb, and while age class success may abound in the bigger clubs and schools (because hard, intense training actually can produce very dramatic short term results), young athletes receive a miseducation in the sport, both in terms of what's required to reach the highest levels, and in terms of their own longer term potential. By the time they reach their university years, many young athletes have become wary of or averse to running even moderate amounts of easy volume, let alone the amounts required to reach their full potential, and often stuck in an event range that does not suit their basic physiological makeup (hence, in Canada, the relatively large volume of 18-24 year old middle distance runners and the almost complete absence of serious long distance runners, and marathoners in particular).

How can we cut against this general tendency and ensure that Lydiard's basic innovation becomes and remains the centre of our sport practice here in Canada? We can make a start by challenging the persistent myths and misconceptions that threaten to undermine an aerobically based approach to training for distance running, chief among which are:

\begin{enumerate}
    \item That lots of easy running volume leads to "burnout" and injury. If, by "burnout", we mean lack of enthusiasm to train and improve, then I would submit that the opposite is true. Lots of high intensity running, which leads to rapid improvement followed by flattened or declining performance, requiring several periods of complete rest during a year, and producing little year-over-year improvement, is more likely to dampen a young athlete's enthusiasm to train over the long haul. (It is no coincidence, BTW, that such an approach is the norm within seasonal high school programs, whose goal is to whip untrained athletes into shape in the shortest possible time.) Those who know anything about the history of the sport will know that this high intensity approach was standard practice in the pre-Lydiard days, when the conventional wisdom was that athletes would reach their lifetime peak performances within about five years of commencing serious training. Since the advent of the Lydiard paradigm, it is not uncommon for athletes to continue to improve as much as ten years into their mature careers, and at a range of different distances. Furthermore, as even the average recreational runner knows, long easy running is a psychological tonic rather than source of stress; it is the kind of training that still attracts us long after we have ceased to improve. As for the question of injury, there is no doubt that running longer creates a greater risk of injury; but, I would argue this risk is no greater than that associated with high intensity training, and probably less than that associated with hard or long training of any kind performed intermittently. Greater injury risk is an aspect of all serious training in any sport. In running, however, the injuries incurred are very rarely serious or debilitating in the long term, and the risk can be mitigated very easily through good daily management, including the timely use of cross-training. In any case, for athletes committed to realizing their full athletic potential, there is really no choice but to court a degree of risk. Is it better, after all, to have played it safe and avoided injury than to have strived for more and suffered the occasional setback?

    \item That too much easy running will make you slow. As Snell suggests, and as the experience of the world's best runners clearly indicates, lots of easy running is more inclined to make you faster, including in a finishing kick. The greater endurance it promotes enables athletes to draw on more of the basic speed they possess at the end of races-- and most good distance runners have more than enough basic speed to produce fast finishes, provided they still have the legs to do it when the time comes. Even at the highest levels of the sport, most distance races are won with top speeds that even a very good primary school athlete could muster when fresh, and athletes with less top-end finishing speed frequently out-sprint athletes who are faster-- that is, when the former are not already safely ahead, due to the faster pace they can sustain over the vast majority of the race!
\end{enumerate}

In the end, we have to be persistent in pointing out to doubters and the uninformed that high aerobic volume running is the global norm at the top levels of the sport. And, we have to introduce young athletes to this reality from the beginning, even as we scale their programs to their age and level of experience. The actual amount of easy running that young athletes do is secondary to the general message that simply getting out the door every day to run at an easy pace is the basis of training for this sport. It's what distance runners do the vast majority of the time, and it's what young runners need to become accustomed to if they want to reach their full potential, whatever that may be. We need to teach young runners (and their parents) that it is this, and not so much what happens twice a week at "track practice", that is the principle basis of training to be a runner. The vast majority of young runners will not go on to become serious runners, of course, but this is no excuse for not giving every runner the best chance of maximizing her inherent potential, should she so choose. After all, while we may know that, statistically, most runners who try the sport will not pursue it beyond school age, we do not know who among them just might! There is thus no excuse for foreclosing the options of young runners by taking the line of least resistance in the development of their training programs.
