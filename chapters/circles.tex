\chapter{Of Circles, Vicious and Virtuous}
\chaptermark{Circles}
\textit{Tuesday, 17 February 2009}
\bigskip

Going in circles is, of course, a big part of our lives as runners, whether it’s going around the track or moving through the cycles of our yearly training routines. There are, however, some kinds of circles that we would like to reinforce and others we’d like to break.

As any experienced runner will tell you, successful training—training to which the body seems to respond and adapt—seems to have a momentum of its own. There are times when the elements of our training—from our strength routines, to our long runs, to our sleep patterns-- seem mutually reinforcing, making training seem almost effortless. Borrowing a term from psychology used to describe the self-begetting nature of positive behavior, I describe this blissful state of ascending cyclicality as the “virtuous training circle”. That same experienced runner will also tell you, however, that the virtuous training circle is sometimes followed by its dismal opposite—the “vicious training circle”, in which misfortune feeds on itself in a downward spiral of fatigue, injury, despair, and de-conditioning.

Common sense would dictate that the secret to maintaining consistent improvement and maximizing our enjoyment of running is simply to prolong our periods of virtuous circularity and reduce those of vicious circularity, or perhaps to eliminate these bad cycles altogether. But how to do this? To begin, we need to understand both what are the constituent elements of the virtuous circle and why it so often collapses, turning into its terrible opposite. We then need to understand how to interrupt the vicious circle before it becomes entrenched.

The first thing we need to be aware of when it comes to cycles is that variability in our body’s response to any training stimulus is inevitable; some degree of cyclicality is, in other words, to be expected even in the most sophisticated training plan. Our bodies simply cannot continue to respond indefinitely to the same kinds of training, whether it be long, slower running or very intense interval sessions. The slip from virtuous to vicious circularity thus often begins when the athlete or coach fails to vary the training stimulus within a single year, or even year-to-year in the case of younger athletes. When diminishing returns from a particular type of training inevitably begin to manifest themselves, the response is often to increase the stimulus, or else discontinue training altogether. The first response often leads to frustration and injury, while the second leads to unnecessary losses of gross training adaptations. The solution, therefore, is to alter or vary the training stimulus at the point of maximum response. For example, we would want to discontinue or drastically reduce our emphasis on interval training-- i.e. workouts that entail reaching 95-100\% of max Vo2 and heart rate for a total of 10-15mins per session-- in favour of other kinds of training—say, threshold-pace running, which entails running for 20-40mins per session at 88-92\% of max Vo2 and heart rate—when we feel we have reached maximum fitness for the season (ideally, just before we approach our seasonal goal races). We might also want to reduce our total volume of running in favour of some faster, more intense running at the point where all those extra easy miles are beginning to feel more burdensome than stimulating. The bottom line is that a properly designed training program allows for a normal amount of cyclicality, and is sensitive to the individual athlete’s response to training.

Second, we need to understand that the secret of the virtuous circle is the rational regulation of daily training effort. What does this entail? In short: the careful relating of training speed to current fitness, which can be easily accomplished using the famous “Vdot” tables found in Jack Daniels Running Formula (of which more below$^*$). Jack Daniels’ single greatest contribution to endurance training science-- and the main ingredient in his famous “formula”—is his discovery of the optimal training speeds required to promote the key physiological adaptations involved in increasing distance running performance. For every athlete at a given level of fitness, Daniels showed, there is a set of training speeds that, when blended within a training cycle, will stimulate optimal physiological responses; optimal, that is, in terms of maximizing potential gains and minimizing the potential risks of training, all other things being equal. Before Daniels (and still post-Daniels in some quarters), conventional wisdom among coaches and athletes was that harder was always better; that, in other words, the fastest speeds attainable in a given bout of training would provide the most vigorous physiological response (with the added benefit, perhaps, of making the athlete “tougher”). The genius of Daniels was to show that in the longer run it was best to train at speeds which, while perhaps slower than an athlete could attain within a given training session if he or she were to go all-out, were optimal in terms of providing the sought-after stimulus-- whether increases in max V02, the improvement of running economy, or the promotion of beneficial changes in the cells of the blood and muscles-- while reducing the inherent risk of injury, illness and diminishing returns associated with intense training. Daniels was able to clearly isolate the percentages of maximum effort multiplied by the length of the training session that would supply the desired training stimulus, and beyond which diminishing returns and increase risks would set in. He was able to show, in short, that there was such a thing as “surplus effort” where training was concerned. Now, it is not true that Daniels’ prescribed paces can never be safely exceeded. Daniels point is simply that doing so on a regular basis will provide no extra benefit, some increased risk, and quite probably a reduction of training consistency, perhaps ending in the dreaded vicious circle of injury and illness. So, when you encounter an athlete enjoying the pleasures of a prolonged virtuous-training-circle you will no doubt have encountered an athlete who has learned how to control his or her effort levels in training. The consistent athlete—the athlete who manages to avoid as far as is possible the vicious circle of injury, illness and stagnation—will be the athlete who does not race her easy training runs just because she feels good on a particular day, or who does not treat every hard training session as an all-out test of her fitness.

Unfortunately, simply understanding Daniels is not all we need to control our training paces and improve our over-all consistency. Determining our correct paces in the real world—the world in which weather conditions may vary, life stresses may intervene, and our conditioning may (and hopefully will) be improving—is an inexact science that requires the judgment of an experienced coach and/or an athlete who has refined his intuition where effort management is concerned. In this latter respect, we can all do well by adapting Daniels advice about racing— i.e. run the first 2/3s using the head and the only last 1/3 using the heart (in the figurative sense) —to our daily training. Our emotions must inevitably become invested in our training, but they can’t be allowed to regularly override what we know and can feel is the best effort level for us in the longer term. Good daily effort management, combined with other basic elements such as proper rest and nutrition, will take us a considerable distance in establishing the bases for the virtuous training circle.

But what happens when, in spite of our best efforts, things break down and injury or illness stop the momentum of our virtuous circle? In addition to the natural cyclicality of the trained body, the necessity of adding greater volumes of running on a year-to-year basis in order to continue improving will guarantee that many of our good cycles, no matter how carefully managed, will eventually end, sometimes in injury and illness. And, for masters runners, we must add the variable of an aging body, which makes optimal training a perpetually moving target. In these instances, the trick is to prevent the entrenchment of vicious circularity and promote the reestablishment of virtuous circularity as soon as possible. Here, the best approach is to actually plan for injury and illness by developing a clear routine of cross training which can be taken up as soon as an interruption of more than a few days occurs. This removes the temptation to push through injury and illness-- or, the opposite, to become suddenly completely inactive-- and perhaps to become depressed and desperate when our pleasurable run of successful training and racing is suddenly ended. It may seem counter-intuitive to think about and plan for failure when things are going well; but, developing and maintaining a cross-training routine during good times is the best way to ensure we’ll get back to them as soon as possible when they inevitably end. Another piece of Danielsian wisdom is useful here: Daniels advises to look upon our “down” cycles as opportunities to rest psychologically, to learn about ourselves as athletes, and to remedy any long term deficiencies that may have led to our downfall. The down cycle can, in other words, be a productive time in its own right—provided that it is not allowed to develop a momentum of its own that prevents us from returning to our normal training when we want and need to. In the end, the solution to preventing— or breaking, if need be—the vicious circle of injury and illness is knowledge, (“self” and otherwise) combined with good planning and disciplined attention to detail.

Maintaining or reestablishing the virtuous training circle—that nirvana of the distance athlete—is not a matter of luck. Successful runners of all ages—runners who continue to improve and to enjoy their running from year-to-year—are successful not because they have won some genetic lottery; rather, they tend to be successful because they are open to learning and have developed the ability to temper their instincts and emotions with the formal and experiential knowledge they have gained from year-to- year. Less successful athletes—those who remain mired in repeated cycles of injury, illness and stagnation, and who may feel forced to abandon the sport—tend to be those who, to borrow a phrase from Albert Einstein, endlessly repeat the same failed action expecting, somehow, a more favourable outcome.

$^*$For those without handy access to a copy of Daniels’ Running Formula, a simple and relatively accurate way to calculate your proper training paces is to use his “six second rule”, which involves adding six seconds per 400m to your training pace as you move from “repetition pace” (the pace you can run for 4-5mins all-out), through “interval”, “threshold” and “marathon” pace (the paces you can run for 3/5k, 10miles/Half Marathon and marathon respectively) and finally to “easy” pace (the fastest pace you should run for most of your recovery days). For example, this would mean someone with a best mile time of 5 minutes would run her “repetition” workouts at 75 secs per 400m and her easy runs no faster than 1:39 per 400m, or in the range of 6:40 per mile/4:30 per km..





