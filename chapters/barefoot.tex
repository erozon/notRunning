\chapter{Down to Earth on the Barefoot/Minimalist Vogue}
\chaptermark{Down to Earth}
\textit{Monday, 11 May 2009}
\bigskip

Over the last few weeks, I've noticed a sudden increase in the volume of on-going buzz on the topic of barefoot running and "minimalist" footwear. Although I have some strong opinions on this topic, I've resisted joining the debate as it has been percolating on the running message boards and listservs-- until today, when the topic found its way into the mainstream, or at least my mainstream, in the form of an extended segment on the CBC current affairs program The Current. (http://www.cbc.ca/thecurrent/2009/200905/20090511.html)

In the segment, host Anna Maria Tremonti interviews guests-- including Chris McDougall, minimalist advocate and author of the book Born to Run, and John Stanton, business man and self-proclaimed expert on running-- on the question of whether the advent of modern running footwear circa the 1970s has been a bane or a boon to runners' overall biomechanical health. McDougall fulminated against the modern shoe, with its array of Madison Avenue inspired "motion control" and cushioning technologies, citing the high injury rate among participants in the post-1970s running boom (somewhere around 80\% per year) as compared with the remarkable example of the indigenous Terra Humara people of the Southwestern U.S., who reportedly run distances of over 100 miles in simple rubber-soled sandals, and who do so well into their advanced years, with very low rates of "overuse" injuries. He also cited his own example-- that of a heavy (200+ pounds), middle-aged, formerly injury-plagued urban runner who has been able to leave his woes behind by eschewing modern footwear in favour of the Terra Humara's footwear of choice. Stanton, meanwhile, offered the view that modern footwear has actually enabled more people to run by offering an array of different shoes to meet the needs of a greatly expanded and highly biomechanically varied running public. My own view is closer to that of McDougall, although I think he and many other minimalist advocates base their actual claims on some questionable epidemiology, suspect biomechanics, and dubious anthropology. As for Stanton, his claim that the big shoe companies, with their endless variety of new, technologically enhanced models and styles of shoes, have enabled more people with unconventional and/or sub-optimal biomechanics to enjoy the sport is a disingenuous half-truth typical of someone who poses, strictly for marketing purposes, as a simple running enthusiast and "coach," but whose real and abiding interest is increased sales and outdoing the retail competition by any legal means necessary. (Were Stanton's purported ideals as a runner and coach ever to come into conflict with his interests as a business man, there is little doubt which set would emerge victorious!)

I'm an advocate of "minimalism," and even barefoot running; but, I think the claims of minimalist and barefoot evangelists like McDougall need to be brought down to earth.

I'll start with the epidemiology that purportedly connects the advent of more padded and "stabilizing" running shoes with the reported 80\% yearly rate of injuries among runners. This is, first of all, a simple correlation of the most general sort. One might equally blame the current rate of injuries on the advent of super light- weight running clothing, which has been just as big a feature of the running scene since the 1970s as more padded shoes. More to the point, however, the running boom has been rooted in the growth of the sport among non-traditional sporting populations, and populations whose average age, weight, and number of hours worked per week have all increased during the period in question; in short, running has grown among a population one could reasonably have expected to incur higher rates of injury no matter what their choice of footwear-- as compared, that is, with the old 1960s demographic of men in their 20s and 30s with university and high school racing experience. The other simple problem is that, with the comparative paucity of rehab services in pre-boom days, we have little way of assessing the real rate of injury during those years. It is quite likely that large numbers of would-be life-long runners "back in the day" packed it in completely after one or two bad injuries, and without registering the fact through, say, a visit to the Dr. or physio. It's therefore hard to meaningfully compare the rate of injury among runners of these different eras, let alone assess the contribution of changes in running footwear to any trend in either direction.

But what of the actual hypothesis that more padded and "stabilized" running shoe mid-soles could cause wearers to incur more injuries than the older, simpler and far less supportive models of the 1960s and early 70s? The bases of this claim concern the biomechanics of the running foot: specifically, the suggestion is that an an over-supported foot will lose its strength, flexibility, and thereby its natural shock-absorbing characteristics; and, that the extra padding and support of the modern shoe will tend to encourage runners to adopt a more "biomechanically inefficient", and therefore more injury-inducing, gait-- a "heal-striking" gait that sends impact forces willy-nilly through the body, reducing forward motion to boot. I'm inclined to agree that modern "motion-control" and heel-elevated footwear could have the effect, over time, of reducing foot strength and flexibility in the lower leg and foot. I'm not convinced, however, that the solution for today's average runner is to opt for minimally supportive shoes, or no shoes at all. The average runner today, after all, has many other things with which to contend when it comes to avoiding injury. Along with the above mentioned fact that the average runner today is a little older, heavier and busier in general, is the fact that he or she has grown up wearing shoes and must by necessity do the majority of his/her running on hard surfaces. For this reason a strict minimalism when it comes to footwear has very little to offer today's middle-of-the-pack runner. To the extent that minimalism has become like a kind of anti-modernist religion or ideology-- and it clearly has in some quarters-- I think it needs to be reacquainted with the concrete (no pun intended) reality of running today. It's probably far more useful to instruct the average runner today about proper nutrition and training than it is to harp about how their shoes may be injuring them.

As for theories about the bad biomechanics of shod versus un- or lightly-shod running, I'm not at all convinced by claims about so-called "heel-striking", mainly because I'm not convinced that there is any such thing as heal striking, defined as catching the majority of one's weight on one's heel while running. If it were really possible to land on one's heel as suggested in this phrase, it's not clear to me how any forward motion at all could be maintained, such would be the extent of the breaking action on each footfall. While it is clear that the heel contacts the ground while running in shoes and normally does not while running barefoot, it's never been shown that making contact with the heel actually slows one down, particularly on harder surfaces. In fact, African athletes, many of whom spent their youth barefoot, and who did many of their early races sans shoes, almost universally opt for shoes when racing and training when given the opportunity (and this includes those without shoe contracts-- the vast majority of African runners). In all my travels, I've met only a few athletes who did not ever make heel contact with the ground (one being a very long time friend, who has had bilateral surgery on his achilles tendons). Even runners who appear to be keeping their heel off the ground on landing have, if one cares to check, at least some wear on the lateral heels of their shoes. (I, for instance, have always been described as a biomechanically efficient mid-foot striker, yet I have extensive wear on the outer heel area of both my shoes, trainers, racers, and even spikes). I think that almost all runners tend to make initial ground contact with their heel area, but bear maximum weight during the "stance" phase of their stride-- the point where the centre of gravity is directly over the foot, and the knee is at full flex. In fact, I think this is where maximum weight bearing occurs even when running barefoot. It stands to reason that this would be the case; if it were not, if would be very difficult to sustain any forward momentum, shod or un-shod.

To clarify my position in this debate, I would say that elements of the minimalist message have great merit. It is generally true, for instance, that shoe companies really don't care as much about large scale injury rates from using their product as they do about providing a soft and secure-feeling ride for the average runner. If you are an athlete with a light frame and strong, flexible feet, you may well be much better off opting for as little shoe as possible. But, if you fit this description, you are not representative of the average runner today-- who, let's remember, did not grow up barefoot, is probably older and above optimal running weight, and who is compelled to run on pavement every day. Minimalists make much of the fact that the human body was evolved for running; however, it was not evolved for running on hard surfaces, and it was probably not even evolved specifically for running at modern racing speeds over very long distances (primitive hunts probably having been very drawn-out affairs, involving moderate speeds over great distances, in addition to the occasional sprint for cover to avoid predation oneself). The average runner today is a long way from this "state of nature", and the sport of running is likewise far removed from anything like the primitive hunt on the grassland for which our bare foot was evolved. So, while the average shoe has undoubtedly become more padded and supportive than it needs to be, shoes in general, and even shoes with more than a minimal mid-sole, represent both an inevitable accommodation to modern living conditions and a technical advance that has enabled us to run faster and probably further than ever before-- which is why even the world's fastest runners tend to opt for them, even when their sponsors are not looking.

Challenged on its faulty logic and stripped of it anti-modernist zeal then, the minimalist message is a useful but far from earth-shattering one. All runners should attempt to preserve and enhance their foot strength and flexibility in ways that do not expose them to greater risk of injury. For a very small minority of runners, this may involve some barefoot running. For others it may involve training exclusively in what are conventionally marketed as "racing" shoes. And, for the majority, it should involve attempting to wear the simplest and least supportive shoe they can manage. Before that, however, it may have to involve dropping excess body weight and doing exercises to help build foot strength and flexibility. It may even involve something as simple as spending more time walking around barefoot in and around the house. In the end, modern footwear is neither the savior it is marketed as, nor the devil it is purported to be in the minimalist fable of our fall from running grace.




