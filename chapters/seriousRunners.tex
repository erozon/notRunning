\chapter{Where have all the (serious) runners gone?}
\chaptermark{(Serious) Runners}
\textit{Tuesday, 3 March 2009}
\bigskip

I ended last week’s entry with a promise to expand a little on my claim that runners like Dylan Wykes—former age-class middle distance standouts attempting to become world class senior distance runners—have become increasingly rare in North America over the past 20-odd years, and to offer an explanation as to why. As it happens, this discussion provides the perfect segue to the second thing I want to do in the blog this week, which is to formally welcome the new athletes who have signed on since the launch of the site. The new members are a mix of male masters athletes still interested in striving for new or age-graded personal bests, and much younger-- again male-- athletes attempting to launch careers at the senior elite level. This self-selected sample offers, I think, an interesting snap-shot of the state of the sport in this country at the moment.

The state of distance running in North America (note: Canada and the U.S. can be treated together here, although with allowances for the recent resurgence of American running) presents something of a paradox. While the sport is seeing unprecedented numbers of participants in the kids competitive and adult recreational categories, an erosion of depth in the senior elite ranks (particularly pronounced among women) that began in the late 1980s/early 90s from a post-war peak in the mid-1980s has continued more or less unabated through this first decade of the new century. There are tentative signs that this decline may just now have begun to rebound, particularly in the U.S.; but, the contrast between the elite scene of today and that of the early to mid-80s is still clearly marked in almost every conceivable measure. Whether in terms of the numbers of sub-2:20 male marathoners/sub-2:35 women marathoners, or in terms of the performance gap measured against world standards, high performance distance running in North America is simply not what it used to be. While North American athletes sometimes appear near the top of championship results lists—such as when Deena Drossin and Mebretom Keflezighi claimed medals for the U.S. in the 2004 Athens Games—the depth of performance at the national level in both Canada and the U.S. reveals a distinct shortage of younger athletes poised to take these athlete’s place when their careers inevitably end. North American distance running seems to be perpetually 2 or 3 athletes away from complete oblivion at the international level, due to an extreme shortage of serious national level athletes capable of forcing one another to improve to the next level.

This discrepancy between the numbers of recreational adult runners and the number of serious racers, including elites, is particularly acute, and is clearly reflected in the dramatic slowing of average finishing times in road races over the years. It would seem that , among adults, including those of prime sporting age, distance running is in the process of being “de-sportified”. For adults, running is, in other words, rapidly becoming an activity pursued for largely extrinsic, non-performance related reasons, such as weight loss, social bonding, tourism, and the enhancement of psychological well being. Some of this is, of course, the inevitable consequence of such factors as an aging population and the active promotion for profit of the activity to the broader public, and to formerly inactive women in particular, through advertising and mass culture (see, e.g., the “Oprah Effect” following her foray into marathon running). Much of this shift is, of course, entirely to be welcomed, since it has been associated with a popular turn to increased physical activity during an epoch of mass obesity. There is, moreover, no reason that such a trend should have been associated with a decline in the pursuit of running as a serious sport, particularly when we consider the fact that it has been accompanied by a wave of increasing exposure to the sport among school-age children. (As I said in an earlier post, I come from an era when very few kids tried distance running, and the sport had virtually NO public profile.) On the face of it, one might just as easily have predicted a boom in serious competition as a result of this combination of running’s increased public profile and the growth in the number of kids trying the sport.

Yet, such a boom has not occurred—quite the opposite, as I have suggested. As a sport (rather than as a “fitness” recreation), distance running in North America has become increasingly like soccer— a children’s activity that is largely adult-initiated and adult-organized (notwithstanding the increased interest in over-40 competition, of which more below). As with soccer, millions of North American kids may run in races every year, but few know anything about, watch, or are generally “fans” of, the sport at the elite level— the way they are with, for instance, basketball, football, baseball, hockey, or even triathlon.

There are, I think, some clear reasons for the failure of this twin boom of adult recreational and kid’s competitive running to produce a growth in serious adult participation, and elite development in particular. As it happens, the aggressive marketing of running as a primarily “lifestyle” activity for adults, and the mass enrollment of children in club and school competitions, have worked, in their own particular ways, to undermine rather than promote running as a serious sport for adults.

The de-emphasis of road races as sporting contests in favour of their promotion as fitness or fund-raising “events” has removed the element of intimidation for the average person, and so swelled the total number of adult participants; but, this marketing strategy has also made running increasingly less compelling to younger athletes-- less exciting and interesting. Young runners can be forgiven if they grow up thinking of road racing as something only people their parents age do to keep in shape, or to raise money for charity! Far less available today are the narratives and images that inspired my generation of school-age runners to want to become long distance runners, and marathoners in particular. Media coverage of running as a sport has all but disappeared, or else has been shunted to the margins—to the back pages of mainstream running magazines, or to websites for the self-selected “hard core” (the latter having probably saved the sport from complete eclipse in the eyes of younger athletes.)

Meanwhile, the much greater and earlier formalization of participation for young runners these days— which, when both primary schools and clubs get into the act, can lead to year-round serious training and competition for the most precocious—contributes to a situation in which, by the end of high school or university, many of the top performers feel “tapped-out” , generally uninterested in pursuing the sport at the senior level, and/or simply anxious to “get on with their lives” (even if this simply means getting a routine job, going to graduate school and/or socializing more with friends). At the same time, by the age of 17 or 18, many others with longer term potential will have long since abandoned any idea of becoming elite runners, believing their failure to achieve immediate, age-class stardom to be the final verdict on their potential.

Clearly, the relative decline of distance running as a serious sport in North America has many other potential explanations, such as the mass entry of East African runners into the sport and the prevalence of doping, both of which have contributed to the vastly increased gap between national level performances and truly world class standards for almost every country on earth, and which have made the pursuit of elite level performance seem perhaps futile for the even the most manifestly gifted young North American athletes. And, for Canada in particular, we might add declining funding for sport development and national team travel. However, since the decline in question has been more pronounced in certain areas than others—i.e. within the longer distance disciplines, and in road racing in particular, as compared with middle distance running, which is also subject to African dominance and systematic doping, yet continues to attract a fair number of participants at the national level—I think the variables I suggest remain central to the discussion.

Turning to the example of Dylan, what makes him relatively unique in the North American context today, and a kind of throw-back to an earlier period in the history of this sport—a period in which North American distance runners stacked-up, on average, much better globally—is that, instead of deciding to either give up the sport at the end of his age-class and school-based career, or to continue to compete within his accustomed event range, he has decided to deepen his involvement in the sport and try his hand at the longer distances, and on the roads, in an effort to realize his full athletic potential. He has decided, in other words, to train harder than he ever has at the very moment when top runners his age typically decide they have had enough, either because they “don’t have the time”, or, at age the age of 21 or 22, and having trained and raced hard since sometimes as young as 13, because they feel they are as good as they can ever be. Today, runners like Dylan are the product of an willful overcoming of the kinds of obstacles I have discussed. The appearance of runners like him occurs in spite of the current state of things in the sport, rather than because of any system for encouraging their development. This means that we must now rely on our young athletes to be remarkable in some respects simply in order to take the first step towards elite status— that is, actually sticking with the sport and training hard for 8-10 years—never mind to become world class.

The first of the problems I identify—the “de-sportification” of road racing—is the most difficult to reverse. I sense, however, that the by-the-bootstraps resurgence of American distance running—a process spearheaded by the advent of privately sponsored and expertly coached “training enclaves” for post-collegiate athletes—is helping to turn the tide, aided in a very important way by the growth of the internet as an alternative medium for coverage of competitive running. Internet-driven awareness of the U.S. resurgence among young Canadian athletes is, I believe, helping to inspire a resurgence of interest in serious distance running among post-collegians here.

This resurgence will remain stunted, however, if steps are not also taken to deal with the second and more easily addressed problem—that of too early specialization and year-round training and racing among age-class athletes, which continues to be a problem across North American. Two years ago, Athletics Canada released an exhaustive study detailing the optimal development path for young track and field athletes—its Long Term Athlete Development Guidelines. Yet, this very important and potentially useful document has not been backed by any effective enforcement measures at the national and provincial levels—measures that might include, for instance, limiting by age the number of competitions young athletes may enter in a season or year. Many road races have wisely taken it upon themselves to restrict entry to athletes under a certain age; yet, age-group track and field and cross-country in Canada remain entirely laissez-faire, with the exception of limits on the distances younger athletes may race in championship events. If we are to expect more young athletes to reach their 20s with an interest in pursuing the sport at the senior elite level—or, at the very least, with a desire to pursue competitive running for fun and fitness, the way thousands continue to enjoy recreational hockey or golf into their middle age—then we have to ensure that their involvement in the sport remains relatively casual and seasonal up to the age of at least 16. In running, it has been shown by literally hundreds of examples world-wide, it is possible to begin serious training as late as age 19 or 20 and still achieve world class results. Yet, many North American parents and coaches remain under the illusion that early, intense training and racing is necessary to reach the highest levels in the sport. In the interests of promoting greater participation rates and levels of enjoyment among kids, as well creating the conditions for the production of more potential Olympians, we need a development model better suited to the realities of running as a sport—one that encourages incremental, age appropriate participation, and that eschews the parent-centred, hyper-competitive culture of sports like minor hockey, gymnastics, or competitive swimming. It’s only by controlling the extent and intensity of kids involvement in distance running before the age of physical and psycho-social maturity that we can expect more of them to be interested in pursuing the sport in their prime years.

It would help, obviously, if those post-collegian runners who do decide to try for the next level could look forward to some support in the form of qualified coaching and modest funding opportunities (routine in many other countries); but, that’s an issue for another post. One of my main objectives in launching Physi-Kultrunning.com, however, is to make the first of these factors—decent coaching— as widely available to young runners as possible. In the absence of American-style “training enclaves” in Canada (with the very welcome exception of the Brooks Marathon Project and the Speed River group in Guelph), internet-based coaching may be the next best thing.

It remains, then, to welcome the new group of runners to the larger P-K running group! As I said at the top, these recent members present a telling picture of the state of the sport in Canada. They are almost all male and are either established and formerly self-coached masters, or much younger athletes looking for help in breaking through to the next level. The masters athletes, being older (of course), come from a bygone era in the sport—and era when serious recreational competition was much more common than today—and don’t need it explained to them that running is more rewarding if pursued seriously, and for life. (Incidentally, why, I’ve often wondered, do serious middle-aged runners these days frequently have to explain themselves to friends and family, whilst middle-aged golfers and hockey players, many of whom spend even MORE time and money pursuing their sporting passion than do runners, pass unnoticed, or are applauded?) As I alluded to above, master’s running is in very healthy shape in North America, with 40+ age runners now routinely dominating the tops spots in local road and multi-sport competitions.

As for the new younger runners in the group, they share Dylan’s generationally non-conformist desire to explore the limits of their athletic potential, and, in contacting me, have demonstrated the extent of their determination to do all they can to get there. As for the absence of new women, that may say something about women’s different use of the internet(!), but it also, unfortunately, speaks yet again of the crisis in serious participation among younger women runners in Canada. In far greater percentages than boys, girls are abandoning the sport at the end of their age-class and school careers; and yet, the data tells us, are no more likely to do so in pursuit of serious career ambitions, or to raise a family. Young people today, male and female, continue to start careers and families much later in life than in past decades, when the numbers of serious competitors, both male and female, were comparably greater. Girls have taken to running in truly remarkable numbers in the past 30 years, and post-collegiate athletes have as much or more spare time as they ever did to pursue senior elite careers; yet, they would appear to distinctly lack the inclination to do so. My strong suspicion is that they have been even more susceptible than boys to the pressures that have turned running increasingly into either a kids sport or a casual fitness activity for middle-aged adults. Girls tend to start training and competing very early and intensely, and likely see little of particular interest to them in road racing as appears in North America today. In any case, this is a topic for another post.




