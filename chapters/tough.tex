\chapter{So, You Think You're Tough?}
\chaptermark{Tough}
\textit{Wednesday, 9 December 2009}
\bigskip

If you're a runner, you're very likely to answer this question with something like: "Maybe, but not enough, and not all the time"; or, "No". When our race performances fall below expectations, we runners are more likely to blame our lack of mental focus, resolve, or what we like to call our "toughness". Loathe to make excuses for a poor performance, most runners prefer to blame what they see as their own psychological weakness rather than other factors, such as their conditioning, or choice of race tactics. Why do runners so often think that they fail themselves in difficult race situations, what exactly is "mental toughness", and how important is it, anyway?

Runners often feel like they're mentally weaker than they want to be, or should be, for a couple of reasons. The first reason is that, because runners tend to conceal their mental states from one another, and yet are intimately familiar their own own doubts and frailties around racing, they are inclined to assume that their friends and competitors must have it together mentally better than they themselves do. As a coach of many runners-- and thus privy to the inner lives of more runners than most-- I know that all runners, even those who appear most in control, are inclined to feel that they are being held back by their own inability to "go deep enough" in difficult race situations. In fact, I am myself a runner with a reputation for being able to get the most out of himself in training and competition; yet, I know that I'm not as "tough" as I might appear in the heat of battle (I am actually a runner who can consistently deliver according to expectation in races, but I also know that so-called "toughness" is not as important in racing well as runners often think it is, of which more below). The second reason that runners often feel they lack it "upstairs" in race situations is that we are so quick to forget how physically difficult racing actually is; in fact, our ability to recall the distress of racing is so poor that we often forget it within minutes of finishing! (How many times have you heard a runner fresh from the finish chute, and still out of breath, say something like: "If I'd only made the decision to go with so-and-so, or push harder at such-and-such point in the race, I would have been 10 seconds faster". Granted, sometimes this is true; but, far more often the runner has simply forgotten how tired he/she actually was at the "key moment" in question. In my role as coach, I've made a point of paying very close attention to the signs of physical fatigue that individual runners are apt to show when the going is becoming very difficult in a workout or race situation; and, as such, I'm often able to tell people that, regardless of what they might think they remember, they were more likely far too physically spent to react meaningfully at the alleged "key moment" than too weak mentally-- that, if there was any failure involved, it was of a purely physical nature.

What we typically think of as mental toughness is, furthermore, somewhat overrated when it comes to it contribution to successful race performance. In my view, the ability to ignore discomfort and increase effort in the final stages of a race, when compared simply with being very fit, properly tapered, and properly paced, is marginal in its importance. I think so-called mental toughness can make a difference in our ability to "go to the whip" (to use a great analogy, courtesy of one of Canada's best ever marathoners, Art Boileau) in the very late stages of a race-- and so can often make the difference between winning and losing, for those at that level. However, the other above mentioned factors are far more important in their contribution to performance, and can make the difference between a great race and a disastrous race, rather than just between a good race and a very good race, as is most often the case with so-called toughness. Ultimately, of course, we all want to be tough, fit, and properly paced all at the same time; but, on days when we're not feeling as tough as others, we can always rely on being physically prepared and smart in order to avoid a complete flame-out. In fact, for those racers unprepared in these vital respects, being tough can actually be something of a liability, in that it can induce us to attempt things we have no business attempting on race day.

But, taking a step back, what is this quality called "mental toughness" and, more to the point, how is it acquired? Here, I think it is necessary to consider the problems of mental toughness and physical preparation as a unity.

Mental toughness, I would argue, is the ability to cope effectively with the anxiety naturally associated with the feeling of hypoxia, compounded by the stress of athletic competition. As with the quality we call courage-- which does not entail the absence of fear, but the ability to cope and perform in the face of fear-- mental toughness does not mean the absence of feelings of anxiety or stress in race situations; it's best thought of as referring to the ability to get the most out of our bodies in spite of the inevitable anxiety and stress that we feel when we race. (As a matter of fact, some of the "toughest" runners I know manifest a great deal of anxiety and stress both before and during competition, and yet race very consistently, and rarely fail themselves in the later going.)

And how does one come to possess such an ability to cope thus in the cauldron of a long distance race? In a word: familiarity-- familiarity, that is, with the actual stresses, physical and psychological, associated with racing. Very few people, I would argue, are born with the ability to push their bodies to full capacity in an extreme test of physical limits, such as a distance race represents. The vast majority of us must learn it, and learn to become better and better at it, through exposure to racing itself, and through training to race. Mental toughness is, as I sometimes like to put it, more a habit than a personal attribute. We develop it through workouts that are structured to reproduce the stresses of racing without being races themselves, and through the drill of simply getting out the door every day. (Speaking of the latter, in a conversation the other day with my daughter, who was having a hard time getting out the door to train in the damp gloom of a late November afternoon, I reminded her that moments like this should be greeted as opportunities, not obstacles-- opportunities to demonstrate our resolve to ourselves and to steel our mental armour for the races ahead. I told her that many times I have thought, mid-race: "I didn't run all those miles, in all that lousy weather, just to give up when it counts most"!). Any athlete who has prepared to the best of his or her physical ability is, by definition, a "tough" person. Racing, by comparison, ought to be thought of as the easy part-- and it is, if all the routine, but vital, physical work has been done. To quote another great marathon champion-- Juma Ikanga of Tanzania-- "the will to race is nothing without the will to prepare". If a runner has already demonstrated the "will to prepare", she will have already developed most of what she needs, psychologically as well as physically, to compete well-- and all the more so if she is able to recognize this fact and resist dwelling on her own feelings of "mental weakness". In the end, therefore, our concern with our "mental toughness" ought better to be directed at our will to get ourselves out the door to train day-in and day-out, than at our ability simply to endure stress in a race situation; for, without the first kind of toughness, the latter type is meaningless.




