\chapter{Waging the Winter Campaign: Some Retreat, No Surrender!}
\chaptermark{The Winter Campaign}
\textit{Wednesday, 6 January 2010}
\bigskip

Reading the many front-line reports from my athletes in the trenches of this still very young winter training campaign, I would conclude that, for the moment, winter has the upper hand. Conditions have varied across the country-- from the bitter cold and snows of Northern Ontario and Alberta, to the unseasonably mild conditions of Quebec City and, to my astonishment, St. John's NFLD-- but no region has been completely spared by winter's assault (except Victoria B.C., but that hardly counts). Despite the promise of a milder El Nino winter north of the 49th, it's been weather business as usual-- perhaps even a little worse than usual in some places-- for Canadian runners. And the casualties are already beginning to mount.

Attempting to train through the Canadian winter is not really a war, of course (we have REAL wars ongoing to remind us of that!); but, attempting to do the sport we love in this climate does present some real challenges in terms of strategy and tactics, and it requires establishing a very clear set of objectives. In other words, it requires an understanding of the special nature of the problem, good planning, and a degree of flexibility on the ground. After more than 30 consecutive years of battling it, the vast majority of them without access to today's arsenal of technologies-- your lightweight, wicking fabrics, your treadmills and your elliptical trainers-- I have determined that, for the most part, winter cannot be beaten; it can only be neutralized via an intelligent strategy and a series of well timed tactical retreats. If we want to exit the winter season no worse off than when we entered it-- and that is really the only realistic goal-- then we have to accept that it will challenge us, and be prepared to retreat from its worst advances when necessary.

Here are my hard-won training tips for battling even the worst winter to a standstill:

\begin{enumerate}
    \item Always assume the worst. Forget environment Canada's long term forecasts and accept the fact that winter will attack, and that it will require you to respond in the form of adapting your training. If it turns out that we all get lucky, great; but, being psychologically prepared costs us nothing, and enables us to respond quickly when the all but inevitable comes to pass.

    \item Accept that you will quite probably lose some overall fitness over the winter, and that you will likely be in no position to challenge any P.B.s in March and April. But, at the same time, don't panic. It's quite possible to regain lost shape, and then some, very quickly once the better weather returns-- provided you're not already injured when that time comes. Again, winter is usually for surviving, not conquering.

    \item Don't attempt to run your highest mileage totals of the year during the winter months. In the old days, it was commonplace for runners to attempt to do their "base" training during the non-competitive winter months. This did not apply well to the Canadian context then and it still doesn't, climate change notwithstanding. Runners in more northerly climes are better advised to do their highest mileage in March and April, and again in August, September and October, leaving the worst winter months for their speed and power work. Faster running, hill running, and plyometrics are actually easier to do in winter, with the help of treadmills and indoor tracks. And, there is good reason to believe that doing this kind of work as immediate preparation for longer, harder training is the best way to proceed in any case (see, e.g. Daniels' Running Formula, Ch. 4). Fast hill repeats with longer recoveries and intervals at mile race pace improve balance, strength and overall biomechanical efficiency, thus reducing our risk of injury when it comes time to do do our longest and hardest training of the year. In any case, those who attempt to hit high mileage targets during the winter will likely become frustrated, risk-prone and, in all probability, injured before it's all over. (Been there, done that, as they say.)

    \item Don't be a hero! Real runners should be interested in building their aerobic capacity and not their character through their training. There are no awards for eschewing the indoor track, elliptical, or treadmill and running outside in all conditions as a matter of principle, or for wearing shorts in sub-zero temperatures-- no awards worth winning, at least. No one will care how "tough" you were in facing down winter in these ways if you end up running slower than you should in the spring and summer, due to an injury sustained in attempting to make your dubious stand.

    \item In choosing when to retreat to the treadmill, elliptical, or indoor track, consider the footing outside rather than the temperature. With some simple precautions, it is possible to train safely outside in very cold temperatures. (And no, you will not "freeze your lungs" if you run in the cold. If this were possible, x-country skiers would be in big trouble!). Generally, only the extremities (and in the coldest temperatures, exposed facial skin) are in any danger in the deep cold. Running on slippery surfaces, however, is a different matter. Without proper traction, we run the risk of de-optimizing the relationship between the benefits and risks of training, and this negative relationship intensifies as our attempted speed increases. When we train, ordinarily injury risk increases with the length and intensity of our sessions; but, so does potential benefit, creating a trade-off. On loose or slippery surfaces, the risk of injury to the hips, groin, hamstrings, achilles, and plantar fascia (not to mention trauma from falling down) increases and is not matched by the potential benefits of the training, since our speeds will be slower relative to the effort applied. Simply put, when we attempt to run on poor surfaces, we increase our risk of injury while reducing the potential benefits of our training relative to other training options.

    \item Learn to use the various modalities of indoor training effectively. Making the best use of your indoor training options entails, first, understanding which of them is best match for runners. There is some debate about this, but, in my experience, the available options ranks as follows in terms of their suitability for replacing outdoor running:
        \begin{itemize}
            \item Running on a treadmill (a no-brainer, really). In spite of some minor differences in our running strides on and off the mill, treadmill running is as close to a one-to-one with outdoor running as you can get, and is useful for replacing the full range of running workouts, from long runs to hill reps. But, remember, treadmill speedometers are not always a reliable guide to actual pace. And, even when calibrated, treadmill speeds feel about 10 secs/km easier than their outdoor equivalents, due to the absence of atmospheric resistance. Always add 1\% of elevation or .2 MPH to equal your outdoor running paces. Finally, be aware that treadmill running forces a faster stride rate than outdoor running; so, when assessing effort, tune in to respiration rate and muscle fatigue rather than to how fast your legs seem to be moving.

            \item Running on the indoor track. Circling the indoor track is perhaps better than t-mill running, but for the greater risk of injury from repeated cornering.

            \item Tie: elliptical training and traditional-style x-country skiing. Both are great modalities for replacing running's aerobic stimulus, but their significantly different limb actions and resistance to gravity make them secondary options (i.e. unlike running, both involve mainly milder concentric muscle contractions rather than sharp, ballistic, eccentric ones). Runners who replace running with these activities will find that their lower legs and feet are somewhat de-conditioned when it comes time to hit the road, trail and track again full-time.

            \item Deep water running. Here, the limb action is very similar to running, but the buoyancy factor reduces the aerobic demands below that of the other options. It's possible to recoup this loss through higher intensity sessions, but most runners without a lot of experience can't manage the kind of intensity required (think: hard interval sessions once or twice a day just to maintain basic conditioning!). The other drawback of deep water running is that the initiation period required to do it effectively (about 2 weeks of everyday sessions) reduces its usefulness as an emergency substitute for running for all but those with prior experience doing it.

            \item Swimming. Great for the few who really know how to do it, and who have the extra time involved, but too technically tricky and time-consuming (about 2 hours of swimming is required to replace 1 hour or running) for the average runner. And, of course, the limb action is significantly different from running.
                
            \item Stationary cycling. Good for multi-sport athletes who have worked out proper bike set up, and who have developed the leg power to reach the aerobic intensity necessary to replace running, but risky to the low back and probably next to useless for the average weak-legged runner.


        \end{itemize}
\end{enumerate}



