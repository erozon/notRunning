\chapter{When to Pull The Plug?}
\chaptermark{Pulling the Plug}
\textit{Monday, 5 October 2009}
\bigskip

No, not \textit{that} plug! However, while I've no intention of getting up on my hind legs to talk about medical ethics, the question of whether and when to put a bad workout or race out of its misery does have something of a personal moral dimension for some runners. My own recent DNF at the Syracuse Festival of Races 5k-- which, at it turned out, occurred at about the same moment as one of my athletes' steadfast refusal to abandon his marathon race, despite being hobbled by a preexisting calf problem that began slowing him down as early as the 8k mark-- gave me pause to consider this, for some, sensitive issue. What, I wondered, made it easier for me to bail out of a 5k race (and it was a pretty easy decision by the time I made it) than it was for my athlete to let go of his marathon, particularly when, it seemed to me, he was putting so much more at risk than I was?

I should start by saying that I have actually dropped out of more races and workouts than many runners have finished! It's not that I do it very often; this recent DNF was only my second in a decade, although I have ditched many workouts in that time. It's just that I've started such a vast number of workouts and entered so many different races, and under such a wide variety of circumstances, that even 5 percent DNF rate equals something like 50 races and 500 workouts! I'm never been happy about abandoning a race or workout, but I will do it in an instant, subject to certain conditions; these are also the conditions under which I tell my athletes that it's O.K. to pull the plug. And, I've relaxed my rules a bit since turning 40, since getting older has rather drastically reduced the number of races I can safely attempt in a year. My "DNF rules" are informed by a basic calculus concerning the probable net effect of struggling to finish a race on my ability to maximize my racing performance in that season. Basically, if I am obviously sick or injured (and particularly if I'm bothered by a a problem I suspected beforehand might flare up), and my condition is clearly going to both negatively effect my performance and quite probably going to reduce my ability to train and race in the near future, I will abandon the effort. In the case of my DNF on Sunday, I started the race feeling under the weather with a nagging cold for the entire week prior, but hoping I would come around just in time (which I often have in similar circumstances). When it was obvious, both in the way I felt and the performance I was putting together, that this was distinctly not my day, I shut it down without hesitation. (As I mentioned in an earlier post, racing or training hard whilst unwell has, particularly since turning 40, been the chief cause of sudden injury for me). My only other DNF of the past few years actually happened earlier this year, when I pulled out of the national track 10,000 at 6k in hot and humid conditions. In this instance, it was because I had no chance of meeting my time goal-- the only goal I had going in-- and fighting to the finish would have meant squandering one of my precious few race efforts of the year, and costing me a week or more of recovery time. In both instances, I made a calculated decision to abandon at a very particular stage in the race and left the field without regrets.

By why, one might ask, do I have rules at all? Why not just allow myself to abandon any race or workout that I simply don't feel like finishing? Although I don't see the decision to finish or not finish a race as in any way a moral one (and I find it a little odd when people take pride in having finished every race they've ever started no matter what), I do think that runners who want to be their best should not get in the habit of abandoning races or workouts simply because they're not going according to plan. Very difficult or unsuccessful workouts and races have a very important role to play in the development of a strong racing mind. Finishing when it would be both psychologically and physically easier to let it all go is very important in building the kind of mental focus required to get the most out of one's body on the days when it is ready to deliver. As my DNF rules suggest, I certainly think this can be taken to counter-productive extremes; but, I do think that under all but the above circumstances runners should attempt to complete races and workouts to the best of their ability on the day. An uncompromising attitude is a powerful tool in realizing one's full athletic potential. Besides, I've often found that interesting and surprising things happen in the midst of what seem to be failed workouts and races. Sticking it out has often given me the opportunity to salvage something of value in an otherwise dismal outing-- a stronger than expected final repeat or a few places gained unexpectedly in the late stages-- something that becomes a springboard to a much improved performance next time out.

Finally, as a coach, I think it's important to let the athlete make the final decision when it comes to finishing or not finishing a race or workout. For some athletes, the decision to finish a workout or race, even when the risk of injury, illness and lost training time is great, has a deeper personal significance. Likewise, the decision to abandon a particular race may be related to factors beyond the scope of the coach athlete relationship. In either case, while I may offer my own point of view, my policy is to respect the autonomy of the athlete when it comes to the decision to "pull the plug". So, while I certainly found it ironic that, at the very moment I was deeming it unwise to run but two more hard kilometers in my 5k race, one of my own athletes was deciding to push on for another 34(!)kms on a gimpy calf, I realized that what was ultimately at issue was our respective relationships to the sport itself. My decision to stop and his decision to persevere, while polar opposites in one sense, were equally expressive of our own uniquely personal reasons for running and racing in the first place, with neither being right or wrong-- another reminder that running is always about much more than simply running.

