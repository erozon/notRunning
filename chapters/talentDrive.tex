\chapter{Talent or Drive: That is the Question (or is it?)}
\chaptermark{Talent or Drive}
\textit{Wednesday, 7 May 2014}
\bigskip

We in the coaching business are often asked: What is more important for success in the sport, "natural" talent, or a willingness to go the extra mile in training (aka "drive")? After all these years of training, observing, and coaching, I think I'm ready to furnish some kind of answer. As I'll try to explain, it's really neither, once you've addressed the more important meta-question of why we run at all.

But, first, some definitions are in order. I define running "talent" as the package of physical predispositions that an athlete receives gratis from nature, a.k.a. genetics (and I leave aside the fascinating question of precisely how we come to receive these gifts-- e.g. how genetic cues can be activated by the lifestyle habits and other experiences of our parents and grandparents, as detailed in the study of epi-genetics, a good intro to which can be found in archives of Alex Hutchinson's always illuminating blog, Sweat Science.) For distance runners, the two most basic features of our genetic endowment are our baseline aerobic capacity (i.e. our untrained aerobic capacity) and the robustness of our response to aerobic training stimuli once it's systematically applied. (Interestingly, the science now suggests that these are indeed discreet genetic endowments that can be easily isolated, and, for the curious, measured using a simple test-- for a fee, of course). Among other complex "natural" variables are our basic body-types and neurological/bio-mechanical make-ups, which, it stands to reason, can affect our basic aptitude for the sport, including our ability to avoid, and heal, common running-related injuries. (These variables are complex because it appears that we're born with some and can acquire others through our early lifestyles. In any case, we can't do much about these things by the the time we're old enough to be serious athletes, so we might as well call them natural gifts.) As with everything to do with our physical natures, our predisposition to run is bound to be a very mixed bag; and, as in the story of Achilles, all it takes is one weak link-- if it is weak enough-- to do us in, or at least make long term success much more of a challenge. We all know at least one example of an Achilles-like athlete-- one who is plagued, often to the point of despair, by a few centimeters of poorly designed anatomy (often, not coincidentally, that which felled the great Achilles himself-- the Greeks knew their running)! However, it's usually fairly easy to plot individual athletes along a rough "talent" continuum. Young runners who easily beat their peers on very little or no training, who dominate the early age-class ranks simply by showing up, and who possess both superior endurance and sprint speed, are the examples par excellence of what we mean when we refer to talent, or natural talent. But, we can't forget those older runners, who, never having had the opportunity to really apply themselves to running as children and youths, improve dramatically, seemingly at the mere mention of the word "training". Casual observation would suggest that the majority of the world's best athletes appear to have been those possessed of most acute responsiveness to training stimuli (most were very good, but not particularly distinguished as youth and junior athletes, and some were really not very good at all until they began serious training), while a tiny few seem to have been double winners of the aerobic-potential lottery-- i.e. at the front of the pack from step-one, yet able to continue improving at the average rate well into adulthood.

The concept of "drive", a.k.a determination to improve regardless of the sacrifices entailed, is a lot easier to define, but a lot more difficult to explain in terms of origin or specific character. The desire to succeed in running, or in anything else for that matter, doesn't seem to have any clear heritable dimension. It seems to derive from an admixture of personality and biography that is as unique as individuals themselves. It can also be a very generalized or highly focused trait (some runners are driven only when it comes to their running, while others bring the same high level intensity to everything they do). Most interestingly, the determination to improve and succeed does not seem to bear any clear relationship to "natural talent" in distance running. In my experience, the most talented runners tend to possess only the average amount of drive to get better; in other words, their surplus of talent tends to make them neither particularly complacent about improving nor particularly driven to reach the highest level possible. And this seems to apply regardless of age. In running, "wanting it", whether the "it" in question is a spot on the Olympic team or an age class Boston Marathon qualifying standard, seems to be a fairly randomly assigned character trait, and seems unrelated to the naturally assigned traits that make up raw physical talent.

"Talent" and "Drive" are, of course, nothing but ideal-types, useful for conceptual purposes only. In the chaotic flow we call practical reality, the two variables are hard to isolate and almost impossible to quantify with any kind of precision. Everyone, it seems, has a complex mixture of both physical aptitude and drive; and, to make matters still more complicated, only the most basic elements of the "natural talent" variable (aerobic baseline, responsiveness to training stimuli, and basic body-type) seem constant over time. Both bio-mechanical aptitude and "drive" can vary considerably over time in the case of an individual athlete. Both physical and psycho-social maturation can and often do transform individual athletes beyond recognition, for better or worse, athletically speaking.

The answer to the question "what's more important for success in the sport, talent or drive?" thus ends up being a pretty complicated one (and dependent on what we mean by "success", of which more below). What we coaches can say with some degree of certainty, however, is the basic type of athlete with whom we most prefer to work. In my case, since I get to work with many of both types, it's a very difficult call. On the one hand, it can be awe-inspiring, and deeply gratifying, to see at close hand, and be asked to help, very talented athletes do what they do. What many of the less talented among us often fail to appreciate is the intense psychological pressure that very talented athletes must face, starting at an early age, in order to be able to compete. Few notice the efforts of the less naturally talented but hard-working athlete as they attempt to make their way up the lower and middle ranks of the sport; but, many watch the very talented-- most, supportively, but some, jealously, in secret hope that they might fail, and thus be brought down to our more pedestrian level. This is true in relation to most human endeavours, and is the lifeblood of the always thriving schadenfreude industry. The very talented athlete who can successfully cope with this often intense scrutiny, whether or not he/she ultimately succeeds in fulfilling expectations, is remarkable to watch and a pleasure to know. And the athlete who does succeed in reaching the highest levels is, psychologically speaking, a unique human specimen, possessed of abilities beyond normal comprehension. Here I recall an anecdote related by the famous British coach, Harry Wilson, whose prize athlete was multiple world record-breaker and Olympic Champion Steve Ovett, one of the greatest middle distance runners of all time. From the age of 16 until retirement, said Wilson, Ovett would insist on speaking with his old coach on the eve of most races. By the time he had reached the pinnacle of the sport, Wilson said that he felt at a loss for anything of value to offer Ovett in these pre-race calls. What, really, could I tell him? said Wilson. He was, after all, Steve Ovett!

On the other hand, few experiences are more inspiring and humbling than watching an athlete unearth his/her unseen potential through hours and hours of difficult, and largely anonymous, labour. Often this kind of athlete will also have had to persist through serious injury, and in the face of the doubts (and sometimes ridicule) of those around them, including friends and relatives. My favourite coaching stories involve athletes like this, both because they are often the most moving, and because they are the most relevant to athletes of more modest talent-- i.e. almost all of us. It is not that the very naturally talented among us do not often work very hard to realize their potential; it's that, when they do, their efforts are much more likely to be recognized and validated in conventional ways. Winning, it seems, justifies sacrifices that seem otherwise absurd. By contrast, the athlete with less natural ability than sheer drive must believe, at least for while, in the value of what they're doing when perhaps few others do or would, and often must wait much longer before receiving validation of their efforts, if, indeed, validation ever comes. And when these kinds of athletes are young-- i.e. at a time when they're most insecure, and when status-consciousness is almost cult-like-- their examples are all the more powerful and exciting to me. If, as a coach, I were ever forced to choose between these two basic types-- the extravagantly gifted or the highly driven--, and all other things were somehow equal, I would ultimately choose the latter.

Luckily, however, I'm NOT forced to choose! And, in fact, the talent versus drive question is not the fundamental one. What's more important is the meta-question of why, as athletes or coaches, we should care one way or another about who has talent and who has drive, and in what proportions, when the more important question is why any of us bothers at all? Isn't it, after all, possible to be either miserable or fulfilled in the sport whether we're naturally and immediately great at it, or instead have to work very hard to do well? In fact, doing "well" at running must ultimately be measured in terms of the intrinsic fulfillment we derive from the activity, regardless of how well we stack up against the best (but, my bias is that I think these intrinsic rewards are greatest when we all strive to stack up as best we can, both against our own potential best, and against the ultimate best that human beings have achieved in the sport). An example of how the talent vs. drive question may not be relevant to the deeper one of how much it all matters, and why, can be found in the great American Steve Scott's admission that he had felt his career-- one that included over 100 sub-4 minute miles, a near world record, and a World Championship silver medal-- had been a "failure", because he had not reached the absolute pinnacle of the sport! In other words, the brilliantly talented and successful Steve Scott had found a way to be less than satisfied with a career of which not one in one million of us could ever dream. And I imagine it would be possible to find a reason for disappointment in the face of even greater tangible success (did Scott's more success rivals-- Steve Cram, Said Aouita, and Sebastian Coe-- worry about their status as all-time greats, and perhaps feel less than satisfied by the assessment?). From every starting point, and at every level of competitive achievement, it is ultimately the same: we create our own joy, or fashion our own sorrow and torment, based on our individual capacity to learn from, and experience as richly as possible, every facet of this very old and elegantly simple activity.




