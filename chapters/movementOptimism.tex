\chapter{Running and "Movement Optimism"}
\chaptermark{Movement Optimism}
\textit{Friday, 1 February 2019}
\bigskip

Personal acquaintances and regular visitors to this space have probably clicked (or not, as per their predilections) thinking they are about to read (or avoid reading) something about capital P politics and some new, running-oriented take I have on them. It turns out they are only half right: The "Movement" in the phrase "Movement Optimism" does not refer to the mobilization of politically activated people in pursuit of a better/different world; it refers to the activation of the limbs of individual human bodies at work and play. But, there is a kind of politics-- a scientific kind-- secreted in the concept. In short, "Movement Optimism" is a kind of "new paradigm" movement within the broader field of kinesiology, and of physiotherapy in particular-- one that seeks to challenge established assumptions about the link between movement patterns, postures, and pain/dysfunction in human beings. "Movement optimism" is shorthand for the scientific conviction (one wants to avoid the contentious term "belief" in this context) that the moving human body is much more adaptable to its own so-called "imperfections" than the dominant paradigm of diagnosis and treatment-- the "kinesiopathological" paradigm-- would allow. "Movement optimism" seeks in particular to challenge the orthodox and dominant understanding of ordinary pain (i.e. pain not related to an ongoing disease process) as rooted in deviations from a movement-ideal whose "correction" must be the focus of treatment. The "optimism" in the phrase thus refers to the ability of individual bodies to, over time, achieve a more or less pain-free equilibrium in harmony with its own signature quirks. As you'll gather from the sample of Greg's Lehman's work linked below, none of this is to suggest that there is no role for therapists in the diagnosis and treatment of chronic pain; indeed, "movement optimism" is meant to offer a new perspective on the causes of chronic pain that will guide the practice(s) of treating or avoiding it.

My familiarity with "Movement Optimism" is via the cogent, and at times polemical, explications of Greg Lehman, a clinical educator in pain and injury rehabilitation with formal training in chiropractic and physiotherapy. Here is a sample of Greg's work, and a nice summation of how he situates his approach within the broader field of physiotherapy and pain-management. I want to apologize in advance to Greg and any of his colleagues and better informed acolytes for any errors or misrepresentations that appear here.

I encountered Greg first as a young therapist, fresh out of chiropractic college. Suffering from a low back pain that was taking the fun out of serious training-- and that had effectively ended my open elite running career in my late 30s-- I was working my way through different treatment options. Like most athletes of my vintage, I had seen chiropractors about my various training-induced pains and injuries over the years, with widely varying degrees of success. Because it was back pain in particular this time, I thought chiropractic might be my deliverance. I was referred to Greg by a friend after seeing-- and roundly dissing-- a older local chiro who, after hooking me up to a spinal scanner that proceeded to spit out a colour-coded picture (red for "bad" and green for "good") of my vertebrae that left me wondering how I was able to stand upright, let alone run around, offered to sign me up for 20-odd treatments to sort out my obviously grave backbone dysfunction, after which I would be prescribed a 5mm lift in my shoe (and, no, I couldn't just put the lift in my shoe and see what happened-- all the red in my scan must first be eliminated through spinal adjustments before something so drastic as a 5mm lift could be inserted into my running shoe). When I told Greg my story and asked if perhaps I did still need some chiropractic adjustments (those pleasant-feeling spinal cracks that DCs are famous for) he said two things that stuck with me: 1. Spinal adjustments are scientifically baseless and probably b.s.; and 2. Bodies are smart; they can easily learn to work around so-called imperfections, such as minor leg length discrepancies. My body in particular, with its then 100,000-odd miles inscribed in its fibres, was probably especially intelligent wrt circumventing minor anatomical asymmetries. What Greg did instead of cracking my back was administer some active-release (ART) massage and prescribe some, to me, newfangled exercises (the now staple routine of "core-strengthening" moves). A third interesting and unlikely thing he offered was that he did not actually know how these things worked to alleviate back pain, but that he knew from clinical practice that they did-- and they did for me, eventually. (I've had the usual run of injuries as I've aged those almost 20 years since I first saw Greg, but I have not had that debilitating back pain since-- a pain that, at 38 year old, was preventing me from lying on my stomach for more than 30 seconds at a time, and that was turning my running into a cruel parody of its former self.)

In addition to becoming a serious and successful masters competitor in the decade after sorting out my back pain, I deepened my erstwhile involvement in coaching, eventually working with athletes ages 8 (through my neighborhood primary school's XC and track programs) to 50, ultimately founding Physi-Kult Kingston. As coach of athletes of all ages, male and female, and of every speed on the continuum, I was able, even compelled, for the first time in my life in the sport to make a study of runners in motion, and to help them work through their different injuries. Greg's advice to think of the human body as "smart"-- as an adaptive problem-solver-- occurred to me often during these years. Coaching also taught me that runners' ways of moving are like hardwired kinetic signatures whose basic elements, while they may become less pronounced with repetition, strength training, and maturation (then again sometimes more pronounced again with aging and physical decline) are always identifiable, so much so that they sometimes even run in families.

When I read Greg's work on what he was calling the "biopsychosocial model of pain and injury" I both recognized the concept from his earlier, simple insight and felt an immediate intuitive resonance across our different but cognate professional practices (I was also pleased to learn that Greg had become a pretty handy age-class distance runner!). As I continued to read the samples of his work that he posted on Facebook, and follow the discussions among "pain professionals" they often provoked, my sense of the affinities between our shared skepticism about the biomechanical paradigm of "correct/dysfunctional" human movement-- his in the form of doubts about the assumed straightforward link between "bad" movement patterns/postures and pain, and mine in the form of questions about the notion of "correct/incorrect" ways of running and their supposed links between success/failure and pain-free/painful running-- only grew. It seemed that we had both seen too many examples, respectively, of people who "should" be in pain but weren't, and vice-versa, and people who "should" be fast and pain-free but weren't, and vice-versa. We had seen, in other words, too many dramatic breaks in the supposed link between biomechanical/kinetic "correctness/incorrectness" and its predicted results that we had begun to doubt the very existence of any such paradigm of human movement. It turns out that we had also been developing similar insights about the psycho-social aspects of pain itself-- that it was perhaps as much a perceptual as a physical phenomenon. I don't know enough about Greg's work to say exactly how he arrived at his insights regarding pain, but I arrived at mine in part through personal experience (examinations of my own experience of chronic pain as an aging, lifelong runner) and in part through witnessing my athletes, young and old, deal with the pain of running injuries. I learned, in general, that factors such as a runner's understanding of the origin of their pain and of the severity of its alleged underlying physical causes; their commitment to running as a sport; and even just their age, tended to determine the expressed severity of their pain and, in some cases, the time it took them to overcome it and return to running. In my own case, I learned that simply believing, rightly or wrongly, that my chronic pain was likely now a permanent feature of running for me made it somehow more bearable. I knew I was often running through pain that that would have sent me straight to the pool or elliptical only a few years before, but I was able to manage it, sometimes for months or years at a time. I also felt pain come and go with no obvious physical cause, and had imaging reveal areas of my body that "should" be producing terrific pain but simply weren't, at least not straightforwardly. After a while I resolved to talk less with my athletes about the physical causes of their pain (and, of course, there were sometimes obvious physical causes of pain) and never about their body's supposed deviations from the ideal, but instead about their body's, and the body's in general, intelligence, resilience, and adaptability. Whatever the case, believing that one's body was "wrong" for running, I figured, would be counterproductive in dealing with the chronic pain that serious distance running often seemed to cause. This didn't mean I stopped having athletes strength-train or get treatment for their pain; it just meant that I encouraged them not to blame supposed "imperfections" in their body for their pain or under-performance, and to understand that all bodies failed at times, yet all could heal and thrive in the sport, often flying in the face of the paradigm of biomechanical optimality. Without even having ever heard the phrase, I had become a "movement optimist".

Like Greg, I'm cautious not to over-draw the operative distinction. I don't want imply that anyone who disagrees with me about "good" and "bad" running mechanics and their connection to pain and injury is somehow a "movement pessimist". My plea as a running coach is to alert people to the psychological risks of telling athletes that there are ideal biomechanical/kinetic templates in relation to which they should be measuring their own bodies, ways of moving, and patterns of injury. In fact, as Greg explains in the piece linked above, biomechanists themselves have been discovering that certain very basic loading patterns of the knees and spine thought to lead to injury are perhaps not as dysfunctional as once understood to be. Likewise, certain kinds of loading patterns involving working directly into chronic pain (e.g. eccentric loading of a sore achilles tendon) have been shown to have therapeutic affects for reasons that are not well understood. When I look at runners and their myriad different ways of moving-- their kinetic signatures-- I see not "flaws", but the ingenious adaptability of the human body, performing an activity second only to autonomic functions like breathing and digestion when it comes to evolutionary refinement. When I speak to runners about their bodies and ways of moving, I encourage them to trust the adaptive capacities contained in this gift of evolution, and to persist in their efforts to achieve relatively pain-free motion. The way I see it, simply getting out the door every day is difficult enough; believing that you also have to wrangle your "flawed" body into some ideal of form and movement-- an ideal against which you are bound to fall short, then perhaps use this falling short as a reason to stop trying-- is both wrong and a sure recipe for driving people from the sport. "Movement optimism" for me, as a running coach, is not some facile "power of positive thinking", or "mind over matter". I do not tell runners that their pain and failed performance is "all in their head". I simply discourage them from thinking that they are doomed from the start by their so-called limitations; that their body is smart about solving basic problems, if given the right stimulus and the time to adapt to it. A part of this message a reminder that there will be inevitable differences in the ease with which people run and the speeds they can attain. Nature, after all, is not fair as regards our social and cultural ambitions. The rest of the message, however, is that almost anyone can run-- and that anyone who can run can get better at running, usually beyond their expectations.

