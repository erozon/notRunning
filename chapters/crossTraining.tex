\chapter{The ABCs of X-Training}
\chaptermark{X-Training}
\textit{Tuesday, 4 March 2014}
\bigskip

No matter how old I get, this sport, and my body, never tire of schooling me. These past few months, I have been returned for some remedial work to my least favourite course-- X-Training 101. As a coach, I have had to teach this particular course on many occasions; but, if there's one thing I've learned from teaching in general it's that the occasional "self-administration" of an exam or two really helps ensure you know what you're talking about. Unlike the other courses I've taught, however, I didn't choose to self-test my own protocols; it was the sport itself that forced it on me. In any case, here I am, revisiting subjects like "Elliptical workouts: Do they suck less than stationary cycling and why?"; or, "Pool running; Belted or beltless?". And since I'm stuck in class for the foreseeable future, I thought I might as well take my exam review public. (For the sake of brevity, I'm going to confine myself to the section of the course entitled "x-training when injured", leaving the section on x-training as a supplement for another installment.)

\bigskip

\textbf{What is x-training?}

\bigskip

For distance runners, it's any non-running activity that stresses the aerobic system in ways similar to running, but without loading the problem area (in my case, this time, the right forefoot). Examples of common x-training modalities are, in no particular order: swimming, cycling, hiking/snowshoeing, x-country skiing, deep water running/water tread-milling, arm-biking, ellipticaling, and walking (yes, walking). Really, only strength training (including so-called "crossfit") is NOT x-training for runners. Building muscular strength and power when unable to run may be called for for rehab purposes, but injured runners should not kid themselves that extra time in the weight room is going to preserve their running gains in any meaningful way. Strength-building activities like X-fit may even add muscular weight gain to the inevitable (but it is hoped minor) fat gain associated with running cessation, leading to reduced relative aerobic capacity.

\bigskip

\textbf{To x-train or not to x-train?}

\bigskip

It may seem counter-intuitive, but there are some knowledgeable people (perhaps most notably, emeritus coach Jack Daniels) who suggest that perhaps injured athletes should consider not attempting to retain their aerobic fitness through x-training when injured. Not least because x-training can be tedious and time consuming in the extreme, this theory is worth examining. It's core logic is pretty simple: in order to reduce the potential for future injury, it may be best to keep the musculo-skeletal system in some kind of sync with the aerobic system. Cross training, the theory goes, preserves, and may even enhance, basic aerobic capacity in injured runners, creating the potential for overwhelming de-conditioned musculo-skeletal structures upon returning to running, initiating the dreaded "cycle" of injury. In other words, the logic goes, the transition back to running may be smoother if the runner's aerobic capacities have been eroded proportionately with his/her musculo-skeletal atrophy.

As exciting as the possibility that there may be some scientific justification for our essentially laziness-based desire to sit on the couch and wait for the pain to go away, I would argue that the benefits of x-training when injured far outweigh the slim, and probably dubious, benefits of letting the aerobic and musculo-skeletal systems remain in sync. There may still be a case for not going crazy on the bike or elliptical, but I can see no valid justification for not x-training at all when hurt. At the very least, aerobic x-training keeps us in a familiar routine, keeps our weight under control, and generally reminds us what kind of athletes we are. Its unpleasantness may even serve to remind us of the things we really loved about running in first place, and make us take the privilege of pain-free running less for granted. (After all, getting injured in the first place usually results from inattention of some kind, and probably should be punished, in order to encourage greater vigilance in the future!) It is not uncommon for runners coming off long stints of x-training to declare that they will never again complain about having to go for a run or do a workout.

My one concession to the "no x-training" position, however, is in the case of injuries expected to last less than one week (but only if the athlete has not x-trained at all in the previous year). For injuries expected to entail less than a week away from running, the initiation period entailed in x-training (of which more below) often makes it more trouble-- and risk-- than it's worth.


\bigskip

\textbf{Choosing one's punishment:}

\bigskip

While we never choose when and how to become injured-- at least not consciously or deliberately-- we do get to exercise some choice when it comes to our x-training. This choice, however, is not entirely unconstrained-- that is, if we want to both reduce our down-time and achieve maximum benefits.

The first consideration when choosing an x-training modality is the type an location of one's injury. All types of x-training can work the body hard in various ways, and none do it in exactly the same way as running. The trick in choosing the best course of x-training action is trying to maximize the aerobic bang-for-buck while minimizing the load on the problem area. For instance, gravity/impact force is your enemy when dealing with problems below the knee (and the further below the knee, the greater the threat). For metatarsal problems, including stress fractures and neuromas, the only real x-training option is deep water running (and, later, water t-milling, for those lucky enough to have access to such technology). For calf and shin problems (including achilles tendinitis and tib. post pain) the elliptical, or some combination of the elliptical, water running, and cycling, may suffice. For problems at the knee or above, however, the hip/glute dominance of water running, plus the greater range of motion entailed, may present problems. Here the elliptical, or other more moderate weight bearing activities such as hiking, may work. As a very general rule, if you get persistent pain in the problem area from any type of x-training, it is probably delaying your return to running, and may not be worth your while in the long run. Things can get tricky, however, if a particular type of training causes only very moderate pain and produces a great aerobic stimulus. In this case, some minimal forestalling of recovery may be a beneficial trade-off, particularly if the injury is going to require a very long period of recovery in anyway.

The second consideration is how much aerobic stimulus you're likely to get from each available$^*$ alternative. Since there is little good science on this question, and likely too many variables to control in any case, I rely primarily on my own hard won experience. Generally speaking, x-training that produces the highest heart rates is to be preferred. In particular, however, it is important to consider how much experience one has with the kind of training in question (i.e. how long and risky the initiation period to the new activity is likely to be), and how much the activity mimics running (in ways that do not stress the problem area, as aforementioned). Swimming, for instance, is a great, full body, near-zero gravity kind of training that is capable of producing very high aerobic loads. Swimming is next to useless, however, if you never learned to do it properly; and, swimming doesn't mimic the running posture or muscular loading at all. Chances are that your injury will be gone before you have mastered swimming sufficiently enough to gain any real benefit. Stationary cycling, on the other hand, while simple to master and great for eliminating ground contact and stimulating the aerobic system, is pretty useless if you lack the leg power to drive the pedals, and thus your heart rate (getting out of the saddle helps remedy this, but its not possible to stay in this position for very long). The same can be said for x-country skiing (skating, not traditional), which has much to recommend it (including the tremendous psychological bonus of being an outdoor activity), but is usually not worth the time and effort to learn (to say nothing of the impracticality of accessing places to do it consistently, for most of us living south of the Ottawa Valley). My own default choice over the years since its invention and perfection has been the elliptical trainer. The elliptical is easy to master (taking only about a week to become comfortable); is safe for almost all injuries (except for my current one, unfortunately for me); produces a reasonable facsimile of the running motion (it's like cycling out of the saddle, except with the arm/upper body involvement that cycling more or less lacks); and, once mastered, can produce truly prodigious aerobic workloads. On the elliptical, it is also very easy to measure one's workload, and thus potential aerobic gains. As such, it appeals to the numbers-based obsession of the average runner, providing a self-competitive stimulus that makes the x-training experience a little more focused and a little less soul-crushing. When using the elliptical, the only really important considerations are the resistance load and rpms (always low enough to spin at 90 rpms or greater while producing a heart rate corresponding to the various levels one would see during different types of run training), and the time equivalence with running (I convert the elliptical to running at a rate of .75/1.00).

$^*$And physical access to different modalities is an obvious constraint at all times. These days, most of us have reasonable access to gyms, and many of us have ellipticals and/or treadmills at home; but, for some, convenient access to x-training equipment and facilities remains a problem. Sometimes the optimal x-training modality is simply the one you can actually fit into your schedule.


\bigskip

\textbf{How to Proceed:}

\bigskip


Once you've decided on a modality, the question becomes how to begin-- and even the most straightforward of x-training activities (e.g. hiking/walking) requires a break-in period. You may think your running fitness has prepared you for any old kind of aerobic activity. If you do, you are wrong. Your running has provided you with a great circulatory/respiratory engine, but it has conditioned your limbs and neuromuscular pathways for running only. You will feel awkward, and end up with some unfamiliar delayed-onset soreness, from ANY unfamiliar pattern of loading, regardless of your general level of fitness. And, in your desperation to get going, you may even injure yourself from your x-training activity. And how embarrassing would that be?

In general, the x-training that involves the least amount of gravity/ground impact is the least risky on start-up. In particular, however, some low-impact activities-- i.e. those that entail radically different postures from that of running-- can require much longer initiation period. The special skills and particular patterns of muscular loading involved can make swimming and x-c skiing, for instance, more trouble than they're worth for many runners, particularly older and less athletic ones (and you know how you are! For those with some recent background in these sports, however, start-up time can be very minimal). The easiest activity where transition time is concerned is probably stationary cycling (but it is also perhaps the worst for runners when it comes to posture and capacity to produce requisite aerobic loads). Most runners can manage up to an hour of stationary cycling at an easy spin tempo (90-100rpms) on the first day. For the most common and accessible types of x-training-- water running and elliptical training-- I have found that 30-35mins at easy to moderate effort is best on the first day, followed by a build-up of 5-10mins per day for a week, with a full day off before proceeding with moderate to harder efforts. The typical moderate to hard elliptical or water running session should last 50-75mins and entail a total of 25-40 minutes of higher effort. And it's important to note that these low to no-impact activities require less recovery time between bouts of higher intensity effort, because of the lack of hard, eccentric loading involved (ellipticaling) and the greater ease of blood circulation (water running). For harder elliptical sessions, I recommend a rest/work ratio of no more than .5 to 1. For water running, .30 will usually suffice (which should come as no surprise to those with serious swimming backgrounds).

How often and how long one should x-train depends on the expected duration of the problem. For problems expected to last 1 to 3 weeks, 50-75mins per day is more than sufficient. For longer term injuries (stress fractures, surgical repairs), it's probably advisable to plan for 60-100mins per day in one or two sessions, with 2-3 moderate to high intensity sessions per week, once the break-in period is over.

\bigskip

\textbf{Lessons Learned(?):}

Besides preserving our aerobic gains, keeping our weight under control, and continuing to feel like an athlete in training, are there any lasting benefits to x-training? There are indeed, and chief among them is the reminder we receive about just how tedious, inconvenient, and difficult x-training really is, after we may have forgotten. The reminder can induce caution in the future, when confronted with the problem of how to respond to an injury threat. Athletes experiencing the first warnings of trouble in a known problem area, and wondering whether or not to roll the dice and continue running, will do well to have a clear recollection of precisely what is a stake in coming up snake-eyes-- potentially weeks of soul-crushing indoor penance, spinning a wheel, or inelegantly plying the waters of some dimly lit and smelly natatorium, usually at some un-chosen hour of the day, and in the often hostile company of the facility's intended users-- actual swimmers, and parents with small children.

Runners consigned to x-training might also discover, or rediscover, the value of maintaining some x-training in their regular routine. Along with providing a little variety and low risk aerobic stimulus, a little supplementary x-training can ensure that, when injury strikes suddenly, we are ready to deploy to the gym or pool immediately. The hassle of starting or restarting an x-training routine from scratch can lead many to idle in self-pity for a few days following the onset of injury, costing them potentially useful training time at what might be a crucial point in their season (and not all injuries strike during our non-racing seasons); or worse, reluctance to hit the elliptical or pool again can lead runners to court risk that they otherwise might not. If a little x-training is a part of our regular routine, we are more likely to do it to ward off injury before it's too late, and get after it immediately following the onset of a sudden or unexpected problem, like an acute injury.

Finally, runners coming off a few weeks of vigorous x-training can learn just how well it can work in preserving, and sometimes even enhanncing, aerobic capacity. Regular readers will be familiar with my accounts of PK runners who have emerged from the pool or dismounted the elliptical to produce solid results, including personal bests, within 2-3 weeks of hitting the ground (which is about how long it takes to readjust to running after coming off of the elliptical; readjustment after water running typically takes a little longer, mainly due to lower leg atrophy).

\bigskip

Class dismissed!

\bigskip

P.S. Stationary cycling and belted (unless you're unusually buoyant).

