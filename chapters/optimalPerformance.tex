\chapter{Determining Your Optimal Performance "Bang"/Training "Buck" Ratio}
\chaptermark{Bang and Buck}
\textit{Tuesday, 13 July 2010}
\bigskip

Now that I've successfully (I think) navigated my way through the Perfect Storm of coaching demands that formed when my acceptance of the coaching position at Queen's met the demands of high school season (to say nothing of the onset of summer yard work!), I've finally found a moment to update the blog. I do so with a post I've been planning for a while. With the personal training/fitness business exploding, and producing a shower of new ideas and modalities for improving our basic strength and endurance, I ponder the question: How are runners to determine the best use of their available training time? I also name the P-K POM for June.

The question of how to determine the best allotment of one's training time and energy is a perennial one, born of the fact that no one, not even the rich professional athlete, has unlimited time and energy. But the continual advent of new types of potentially beneficial and accessible forms of training, some through slick marketing and others through natural cross-fertilization between sports disciplines, or, occasionally, through the modification of injury rehab modalities, make this problem ever more potentially tricky. And for the recreational athlete in general, and time and energy-challenged older athlete in particular, the problem of optimizing the training mix is that much more vexing.

I begin with the following disclaimer: Ultimately, the optimum allotment of training time and energy between the available options is highly specific to the individual athlete, and requires consideration of a vast range of personal variables. It is thus something to be determined between coach and athlete, and negotiated over time through trial and error. What I offer here are therefore only general guidelines based on what I understand to be the best approach to producing performance improvements in runners falling into range of very broad categories, including: \textbf{younger developing runners} (i.e. ages 13 to 22); \textbf{experienced "peak-age" runners}; \textbf{older beginning runners}; and \textbf{experienced older runners}. The forms of training I consider can be placed in similarly broad categories, including: \textbf{strength-based} (i.e. various kinds of mostly low-speed resistance training, most typically involving weights); \textbf{aerobic power-based} (i.e. running at speeds and for durations requiring maximum possible oxygen uptake); \textbf{endurance-based} (i.e. sustained running at speeds that are known to over time produce gross anatomical and physiological adaptations, such as increased cardiac output and blood volume, and improved oxygen and energy usage through cellular development within the relevant muscle fibres, typically associated with high performance in long distance disciplines); and finally, new and older forms of \textbf{alactic/ballistic/plyometric} (ABP) training heretofore more typically used by sprinters rather than distance runners (i.e. jumping, bounding, very short sprinting, and even barefoot running). (Note: I exclude the category of flexibility training because I have already discussed this in a previous post. At best, I think this form of training belongs in the strictly optional, or "bet-hedging", category for all but the most leisured distance runners.)

For younger developing runners-- and, in particular, those who have come to the sport at what I consider to be the proper age, and with a background that includes plenty of formal and informal play-based physical activity-- the vast majority of the training stimulus should come from aerobic power and endurance-based training; and between these types, 80-90 percent of training time should be allotted to the latter. For those without an extensive background in play-based activities, some form of APB-based training may be beneficial (but, this can often be achieved by young athletes "playing" at a range of other track and field disciplines in the context of an all-events track club). Where younger developing runners are concerned, however, the main consideration when it comes to their involvement in the sport of distance running must always be psychological/sociological rather than physical. To ensure the best chance at both short term enjoyment and long term performance maximization, training of any kind should be very moderate before ages 15-17 (i.e. from 2 to 5 hours per week only). Training for distance running, because it is overwhelmingly based on the very long term and intensive promotion of gross bodily adaptations, is acutely work-like. And play and self-discovery, not work, should be the foundation of childhood as far as possible, especially in light of the fact that children already "work" 6 or more hours per day at their educational development. As younger runners reach the typical age of psycho-social maturity (typically ages 15-17), their volumes of training can increase fairly significantly, with the emphasis, as I say, on developing the kinds of capacities most central to success in the sport, and most susceptible to improvement through training-- aerobic power and endurance. At this stage, however, the role of a coach becomes more important than at any stage of athletic development, and crucial for the young athlete's longer term development. Younger athletes, even while they may be physically and psychologically ready to train very hard, still lack much feel for which forms and quantities of training that work optimally for them at their particular stage of development. And most runners before the age of 21 or 22 remain somewhat lacking in the patience required for long term development. They are also easily discouraged by the occasional and inevitable poor performance.

For peak-age athletes (those in their senior post-secondary years to about age 35), the vast bulk of the training stimulus should continue to come from the two staples-- aerobic power and endurance-based training; but, there is an increasing premium on strength and ABP-type training in these years. Luckily, athletes in this age range will typically have more disposable time and greater energy to train than at any stage of their lives, making it relatively easy to accommodate these new training demands. The greater necessity for these other forms of training arises as training capacity in the two main forms begins to peak, and as acute specialization begins to erode the basic support structures (like core-strength, balance, and power) that were formed in childhood through play-based activities. On the most basic level, high-level distance training is somewhat catabolic as well as neurologically narrow in demands, meaning that it tends to reduce overall muscle mass and retard basic athleticism. Some of this is the inevitable and beneficial result of intense specialization, but the line between functional specialization and over-specialization can be a fine one. This tendency towards over-specialization can be countered to some extent through strength and ABP-type training. I say this, however, with the following caution: New types of ABP training, especially plyometrics and alactic running, are currently in great vogue among runners with the time to fit them in, and many are quite alluring in their novelty and apparent sophistication, creating the strong impression that they must be useful and effective (and certain specific forms may yet prove to have little or no real benefit for distance runners). Such is the attractiveness-- and even fun-- of these forms of training that athletes may be tempted to substitute them for additional running (i.e. in cases where athletes have still have room beneath their capacity for more running volume). And I have a more specific fear that the vogue of strength and power work may feed into the stubborn misconception that, because distance races are frequently won with fast finishes, training that improves speed and power is of equal importance to aerobic power and endurance work. (While understandable, this is misconceived: first, because, in distance races, speed is only really effective once maximal aerobic and endurance capacities are reached; and, second, because the finishing speeds required to win even the highest level distance races are attainable by even the average high school sprinter, and by most good high school distance runners. Finishing speed-- i.e. for finishing long distance races-- is thus much more a function of aerobic power and endurance (to say nothing of emotional drive) than it is of the kind of speed and power that ABP-type training promotes. The proper place of this kind of training is, therefore, as one of the supports of a high-volume running program, whose benefit is mainly prophylaxis against injury rather than direct performance improvement. But even here, however, runners-- even strong, peak-age athletes-- must be aware of the injury risks as well as the benefits of ABP-type training.

And it is with risk/reward ratios of various training modalities that discussion of the next category of runners-- that of the older developing runner-- must begin. The middle-aged beginner athlete (and here I mean not someone who has never run at all, but someone who has never trained seriously for running) will typically lack the the accumulated benefits of any kind of training-- either the gross physiological adaptations associated with years of serious aerobic and endurance training, or the strength, power, speed and balance associated with resistance and ABP-type training. This category of athlete, in other words, will be able to make significant gains (subject, eventually, to some age-graded loss, of course) in all of these basic areas; yet, because he/she is older, and therefore at somewhat greater risk of injury, due to his/her inevitably slower recovery rates and, quite likely, reduced leisure/recovery time. For this category of athlete the question therefore becomes one of where to place the bulk of the training emphasis across a range of almost equally beneficial options. In my view, the older developing athlete must have greater balance in his/her program between strength, aerobic power, and endurance-based forms of training, using ABP training only very sparingly, if at all. As with younger developing athletes, the older beginner must focus the bulk of his/her training on improving endurance and aerobic power. The older athlete, however, will typically be lacking in muscular strength, and in the core areas in particular (i.e. the stabilizing muscles of the low/deep abdomen, and of the low back, glutes and hips). Older beginners will also typically have weaker feet and lower legs. All of this atrophy is the product both of age and the reduced daily, dynamic physical activity associated with modern adulthood. A program of strength training that focuses on typically weak areas is therefore not only beneficial, but frequently crucial for the success of the older newcomer to the sport. The combination of a soft core, thin, inflexible feet, and weak calves is an injury time-bomb, and doubly so when the slower rates of recovery and repair associated with the aging process are factored in. As for ABP training, one might assume that it would come highly recommended for the older athlete. These kinds of training, after all, are designed to produce highly movement-specific forms of strength and power. The trouble with this form of training when it comes to the older athlete, however, is that it produces these gains in dynamic strength, power, and balance through the sometimes extreme and repetitive eccentric loading of the muscle (i.e. the rapid simultaneous stretching and loading of the muscle fibres). Older muscles, being usually a little shorter, less elastic, and slower to recover from micro-trauma, are much more susceptible to injury when worked in this way. There is no question that ABP training can be highly effective for older runners; the question is whether the performance gain is equal to the injury risk, especially considering the fact that the greatest net gains for the older runner, as for the younger athlete, are likely to come from simply running more and running a little faster a couple of times per week. As a coach of older beginners, I never rule any kind of training completely out for all runners; my general caution to new runners over 35, however, is that they spend their scarce training hours and energy running more, and doing non-eccentric forms of strength training, focusing mostly on the core and often on the lower legs, leaving the fancy ballistic/plyometric drills to the school-age and professional runners.

The final category of athlete is that of the older, experienced runner. (In other words, people like me!). The older more experienced athlete, and the ex-elite runner in particular, will typically have reaped all, or nearly all, of the basic gains they are every going to get from the staple forms of training-- aerobic power and endurance-based forms. This category of runner will typically have run tens of thousands of easy miles, and have completed hundreds of intense, max V02-building sessions on the track, road, and trail, to say nothing of hundreds of races over a variety of distances. As a result of all this training and racing, the older, experienced athlete, if he or she has followed a sound training plan all those years, has nothing to look forward to in terms of increased performance; after about age 35, it is a slow but steady downhill slide (although the speed of the decline will typically vary greatly from person to person, and can be greatly mitigated by continued long and hard training). My advice to the older, experienced athlete who still retains a taste for training and racing is to concern his/herself more with remaining injury-free, which means following a reduced and balanced program in which the training focus is varied regularly throughout the year. The older, experienced athlete, because he/she has trained throughout adulthood, will have retained a great deal of the basic athleticism required to effectively use the full range of training modalities, including ABP-type exercises. The older experienced athlete will gain less from an additional mile logged or hard workout completed than will the older beginner, and can thus afford to spend a few weeks each year making, say, strength or plyometric power, rather than aerobic and endurance training, the focus of his/her training. As a general rule, however, this category of athlete-- the rarest of them all, for obvious reasons-- is advised to recall that he/she cannot safely manage the same total training volumes as during peak years. Older experienced athletes who cannot adjust to this reality are eventually frustrated, and find themselves relegated to the sidelines, physio table, and cross training regime.

I conclude with a reminder to runners of all ages that the basis for success in this sports remains running itself, and longer, aerobic running in particular. The runner who has not explored the limits of his/her capacity for daily aerobic running is a runner who has not yet entered the real world of the sport. It is frequently said that the beauty of running is its absolutely simplicity; that the challenge for the Olympian--to push his/her body to the limit in search of the greatest possible level of adaptation to the basic demands of the sport-- is no different than for the average age-classer. The difference between the best and the merely average resides in part in the sheer capacity of the former to survive the grind of training required to reach the most extreme levels of adaptation; but, the struggle of all athletes to realize their maximum personal potential is a very much a struggle to run as much as possible without breaking down. Other forms of training may help shore up the embattled body of the serious runner, but they are never a substitute for the activity itself. Many have chosen, for one reason or another, to believe otherwise-- to believe that additional speed and power, for instance, can produce greater performance gains in long distance runners than more long distance running itself; but, when one observes the training of the sport's best at every level, the verdict is in, and has been in for decades: To run fast over a long distance, one has first to run slowly over a much longer distance, and many times over. It is only on this foundation that the other forms of training discussed can have a meaningful impact on racing performance.
