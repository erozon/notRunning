\chapter{Allowing Ourselves to Learn}
\chaptermark{Allowing Ourselves to Learn}
\textit{Monday, 19 October 2009}
\bigskip



Many years ago, in the hour before what was to be my fastest ever 10k, I made a pledge to myself that I have honoured ever since. If I run well today, I promised, I would never again worry about how my body felt while warming up for a race. As I had a hundred times before, I jogged through my pre-race warm-up that evening feeling so tired and sluggish that I wondered how it would be possible to reach and sustain my goal race pace. If four and half minute kilometer pace was making me feel uncomfortable, how would I ever manage the two minute and fifty-two second ks that my training had told me I was capable of!? I had, of course, felt weak and sluggish many times before what would turn out to be strong races, but I had never before promised myself to remember precisely how I felt in order that I might stop worrying about it for good. Each time in the past, all of my pre-race anxieties would be forgotten in the rush of excitement of the race and the great wash of relief when it was all over. And, when I logged my race report, I would rarely mention how I felt beforehand, only how I felt during, and the race result itself. What I had decided to do differently this time was to set my emotions aside and allow myself to consciously learn something from my racing that would benefit me next time out. We're always learning from our races and workouts, of course, but the knowledge that most of us accumulates is unconscious and intuitive, such that, over time, it is difficult to catalogue precisely what we know, how we came to learn it, and what it felt like not to know it.

Much of what I do as a coach these days, both during routine weeks and on race days, consists of spotting and recording patterns in the way my athletes respond to their training stimuli and racing experiences. Through my communications with athletes, I then try to pass on what I discern so that the athlete will be better able to participate in the process of his or her own coaching, both through providing more meaningful feedback, and through developing a deeper, more intuitive understanding of their own training process. My job is made much easier, and the coaching process much smoother and more productive, however, when athletes allow themselves to accumulate their own body of experiential and intuitive knowledge, and when they begin to record it, both mentally and in their actual training logs.

Over the years, I've discovered that some athletes are far better at both learning from me and from their own experience than are others, and that this has nothing to do with basic intelligence or innate running ability. The difference, I think, can be explained in terms of the ability of some athletes to set their emotions aside long enough to allow their rational faculties to fully apprehend the training and racing process. Emotional drive is, of course, absolutely crucial for success in this and any other sport, and those with more of it tend to enjoy greater success than those with less of it, when all other things are equal. Athletes with a greater emotional investment in what they're doing also, I think, have a richer experience of sport than those who manage to do it completely dispassionately. The best athletes, however, are better able to confine their emotions to the moments when they are useful-- such as in the difficult sections of a hard race or workout, or when they are forced by injury into a tedious cross-training regime-- such that they have the mental space to learn from what is going on around and within them. With many athletes, and younger ones in particular, I find myself having the same conversations, and trying to impart the same information, over and over again at workouts and before races. With these athletes, I'm always on the alert for the ideal "teachable moment"; but, the lessons are often slow to stick. These are the athletes who constantly worry about how they feel and doubt their fitness beforehand; who, in blaze of emotion, ignore carefully plotted pre-race plans in favour of "how they feel"; and, who forget everything that happened before and during a race or workout almost immediately after. Athletes like this are frequently very talented, and their passion and free-spiritedness often produce spectacular performance breakthroughs; but, more often, their fire and spontaneity lead to failure, disappointment, and confusion. And then there is the flip-side of the emotional coin: those athletes whose fear and anxiety repeatedly prevent them from taking risks and taking full advantage of the physical adaptations they have earned through their training. In the end, athletes who habitually put passion over reason tend to have shorter and less fulfilling career than those who strive to manage their feelings long enough to learn from their experiences.

Unfortunately, there is no simple secret to setting our emotions aside long enough to begin to learn from our own training and racing experiences. As athletes mature and gain more experience, learning to learn becomes easier; the ability, however, will always comes more easily to some than to others. The best way to become a better student of our own training and racing, however, is simply to keep a good training log. The next step is to learn what information is most useful to record; and, more important than things like daily training heart rates, body weights, and calories consumed, which rarely vary much, is subjective information, such as our thoughts and feelings before, during, and after our races, and at different stages in our training year. Among the things I've learned from recording subjective information in my training logs is that I will tend to feel in a race almost exactly the same on average as I did in my final two workouts before, regardless of my basic conditioning, or how I felt immediately prior to the race, whether good or bad. This information has both greatly calmed my nerves before races that my warm-up has suggested might go badly, and prepared me to face racing situations in which, in spite of feeling normal in the warm-up, things might not go my way. I've also come to learn from my training logs that periods of training that I remember as having gone uniformly well-- simply because they preceded a very good race, or included a particularly memorable workout-- often contained many more sub-par workouts and anxieties over possible injuries and other physical problems than my gilded memories suggested. This information has many times worked to allay fears that perhaps my training was not going well enough to prepare me for an upcoming race.

Short of having a good training log (which, of course, takes years to compile), the especially nervous or emotional athlete can begin to create some space in which to learn simply by spending some time post-race revisiting his/her feelings beforehand, both immediately prior to the race and in the key workouts leading in, with an eye towards better understanding and mastering any negative tendencies they might have. Often all it takes to prevent our emotional drives, fears and anxieties from interfering with our performance (to say nothing of spoiling the whole experience of racing itself) is a little self-knowledge gleaned from the study of our own basic emotional tendencies. This way, we are better able to get out of the way, so to speak, of our own well trained bodies long enough to let them do their thing, and confine our emotional drives to moment when they are most useful. Legendary American coach Jack summed-up this problem nicely when explaining how best to approach racing the marathon: run the the first 3/4 with the head, he said, and the last 1/4 with the heart. I would add that this works all the better when we have first used our heads, meaning our rational minds, to understand not just how our personal bodies work, but how our sub-conscious mind and emotions behave during the process of training and racing.
