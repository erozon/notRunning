\chapter{No Great Coaching, Only Great Re-Coaching}
\chaptermark{No Great Coaching}
\textit{Wednesday, 17 April 2019}
\bigskip

The late Justice Louis Brandeis' famous advice to authors everywhere about importance of assiduous revision and editing over first-draft inspiration-- that there are, ultimately, "no great writers, only great re-writers"-- probably applies to many things, but it most certainly applies to coaching runners.

I refer here to the difference between "the plan" and it's actual, day-to-day, on the ground, execution. I was reminded of this when reading Rachel Cliff's illuminating account of her recent Canadian Record marathon build. In hindsight, Rachel's success in Japan appears to have been of the "accidental" variety we hear so much about in our sport. Starting preparation for a race to which she would eventually be denied an invitation (the prestigious Tokyo Marathon), and encountering a series of small problems typical for runners operating at the very edge of their training capacity, she would shift focus to the shorter and more familiar (for her) half marathon distance. Then, as luck would have it, she received an invitation to the lesser-known but still top-notch Nagoya Women's Marathon. With training going a little more smoothly now, she reverted to her plan A and signed up. The rest of the story is now a matter of sports history, but it ends with her discovering, to her delight, that she was actually far better prepared than she thought, in spite of the radical non-linearity of her road to the start line.

The more we know about what happens behind the scenes of any successful runner's results the more we come to understand that Rachel Cliff's story is more the norm than the exception. Runners always seem to be surprising themselves by producing personal best results following long periods of injury, or off of "broken play" preparation, like Rachel's Nagoya build. And even when training seems to have gone "according to plan", closer inspection of training notes after the fact often reveals dozens of instances of altering or adjusting said plan on a day-to-day basis. In fact, the training leading up to any successful result was, on closer inspection, almost never the inevitable march to triumph that it sometimes is in the memory of coach and athlete. Yet, even knowing as we do that our plans are going to be subject to revision, sometimes of the radical sort, coaches and athletes are likely to feel panic the next time lived reality forces us back to the drawing board, often with the added stress of deadline pressure. It seems that we always begin with the expectation that this time the plan will proceed exactly as written!

Mine is not a plea to embrace contingency and chaos and to stop making plans altogether. This would be impossible (on what basis would anyone decide how much or how to run on a given day?), and likely ineffective even if it weren't. We need to continue making plans, not least because distance runners compete far less than other athletes relative to the time they spend training, but also because our major goal competitions are often months, even years, in the future. Instead, I propose that athletes and coaches author their plans in the full knowledge that they will need to be re-written on the fly, and with a sense that having to re-write a plan is not "failure" so much as a concession to reality. Simply eliminating the anxiety around having to alter a master plan would be a benefit in itself. Instead of feeling that the original plan-- the "first draft" if you will-- was the best and only way toward the goal in question, and panicking when following it to the letter is no longer an option, athlete and coach would both be better to see temporary setbacks that force deviations from a plan as opportunities to adapt, and perhaps even improve, said plan in light of new information. Such an attuning to the reality that all plans are made with insufficient knowledge of the future-- because the future, being "the future", is always somewhat unknowable-- would also help prevent athlete and coach from doubling down on obvious errors, or, what amounts to the same thing, sticking with a plan for its own sake, out of fear of the unknown.

And what applies to making plans for a single athlete in pursuit of a single seasonal goal applies to coaching in general. Our personal "coaching philosophies" are really nothing but overarching frameworks comprised of accumulated knowledge and intuition about the sport and how best to approach it. They are, in other words, master plans-- which ought to, like all plans, be subject to critical scrutiny and, if need be, revision. And there is not a coach alive who would deny this basic reality. Most of us, however, rather than being open to rewriting our master plans from time to time in light of new information, are inclined simply to add new elements to them as we discover potentially valuable interventions (the overwhelming majority of which turn out to be wastes of our attention and our athlete's precious energy, but that's for another time). What we are less willing to do is to examine and re-examine the basic practices that form the starting points of our approach to training (e.g. number or workouts per cycle, total volume, use of resistance training, amount and placement of competition, etc. etc), particularly as they apply to individual athletes. Some of these basic elements can become like phrases we have grown to love and are thus loathe to consider revising, let alone cutting entirely from our manuscripts. It is often the case that the practices we consider to be foundational to success in the sport withstand our critical scrutiny and re-earn their place in the larger opus; but, no element, new or old, should ever be permanently exempt from scrutiny.

Finally, to the importance of re-writing/coaching I would add the indispensability of regular reading. Just as writers are constantly exposing themselves to the work of other writers of maximally diverse variety (from children's authors, to journalists, to TV screenwriters), running coaches need to "read" the works of other coaches, including those in other sports. Indeed, since coaching is largely a form of educational discourse, running coaches need to pay attention to the works of anyone who is good at thinking systematically, critically, and originally, and at communicating the fruits of that thinking. When we do, we find that, as with authors of all types, the best at what they do display a resistance to stale formula and clichè that reveals an abiding willingness to critically examine and revise-- to re-write-- anything and everything.
