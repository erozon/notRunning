\chapter{"Teaching" Running}
\chaptermark{Teaching Running}
\textit{Thursday, 6 October 2011}
\bigskip

What is the difference between being a running "coach" and a running "teacher"? An article last month in the Life section of the Globe and Mail$^*$ on the proliferation of organizations and individuals offering expert services for recreational runners-- from old-fashioned coaching, in which an experienced hand designs and oversees a regime of running at different paces, all with the goal of achieving better racing performances, to instruction in "how to run", focusing on the simple act of putting one foot in front of the other as quickly as possible-- set me to thinking about this distinction.

To begin, there is clearly a teaching element to all coaching, and running is no exception. As I suggested at length in an earlier post, effective coaching engages an athlete's creative, intellectual, and emotional capacities, and, over time, equips him/her to become a more active participant in the training process. On the most basic level, good coaches encourage athletes to make a study of their own body's response to different training stimuli, while providing a framework of experience (both their own, if they are runners or former runners, and that of others they have known or coached) within which to situate this growing knowledge. The best coaches are also well versed in the lore of the sport, and are able to inspire and instruct athletes with stories about the success and failure of those who have gone before them. As a result, runners frequently recall their favourite coaches as not simply "programmers", but as teachers and mentors. There are some strict limits, however, to what a running coach can actually "teach" his/her athletes.

I have always been very skeptical, for instance, of the notion that coaching extends to the level of teaching athletes literally "how to run", and nothing in the new spate of commercial offerings to this effect has altered my reaction. Today's running "teachers" are often sincere and well meaning, but what they offer is based on an entirely faulty premise, and they survive by exploiting for commercial purposes the ignorance of new runners concerning the basic elements of successful running performance. In particular, they exploit the widely held misconception among non- and beginning runners that fast runners are fast, and often look different ("better") than slower runners, because they "learned" how to properly move themselves across the ground, and that this "knowledge" can be made available to the uninitiated through detailed instruction. Often the first question I'm asked by neophyte runners, or the parents of young runners, is whether I might be able to "fix" their running form, or tell them how to run "properly". There is, indeed, money to be made in plying this particular area of the trade, but only at the cost of indulging the consumer in the basic misconception that good runners are good because at some point they learned to "self-monitor" their way to more fluid, aesthetically pleasing, and therefore more efficient, running mechanics.

It is a misconception, and, I think, a form of coaching malpractice, to suggest either that physiologically "efficient" running mechanics can be seen by the naked eye, or that, to the extent that efficient runners do move in some broadly similar ways, this way of moving can be simply imitated by anyone, regardless of their particular configuration of body parts, angles, tensions, and muscular strength. The question of our ability to "see" physiological efficiency or inefficiency (i.e. the relative energy costs of running) through an athlete's pattern of movement was definitively answered when veteran coach and exercise physiologist Jack Daniels asked a selection of experienced coaches to identify the most physiologically efficient runners within a sample simply by watching them run. The result was that there was no correlation between the coaches perception of "efficient" form and actual physiological efficiency as measured through the use of oxygen at sub-maximal speeds on treadmill. As it happened, the least classical looking-- or, if you will, the most awkward looking-- runners in the sample were often the most physiologically efficient-- and physiological efficiency in the only kind that really counts when the object is to cover long distances as quickly as possible. The question of whether, to the extent that faster runners do tend to move in similar ways in comparison to slower runners, and beginning runners in particular, is a more difficult one to settle. Packs of fast-moving elite runners really do look broadly similar to one another, and different from groups of recreational joggers. The lessons we can take from this fact, however, are not nearly as straightforward as today's running "teachers" would have their clientele believe.

To begin with, faster runners have generally been "selected" based on optimal running mechanics. Those with gross skeletal asymmetries or malformations, muscle imbalances, etc., are generally weeded out of the top ranks of the sport, as harsh as that may sound. Furthermore, the front ranks of any race are made up of runners who have trained longer and harder than beginning runners, in spite of being generally younger than the average recreational and beginning runner. In other words, elites, by definition, are rare specimens in almost every respect, and their running form, no less than their physiology, is a product of this uniqueness-- a uniqueness that is not the product of conscious learning, but of simply being and doing. No amount of analysis and conscious imitation of the way fast runners move is going to enable a runner with a physiognomy that is not optimally suited for running, a runner without many miles of running to help refine his/her neuro-pathways, or an older runner, move with the same power and economy as a younger runner who has reached the top of his/her game, and who has done so in large part by running prodigious numbers of kilometers. The best distance runners in the world did not themselves "learn" to run the way they do through conscious self-monitoring or instruction; they either moved more or less that that way from an early age, and thus tended to excel at the sport, or else developed as refined a pattern of movement as they could for their particular bodily configuration through simple repetition, in precisely the way that all physical movements become more automatic and economical the more they are repeated (think, for instance, about typing, or tying a shoelace). This is why, in spite of the general similarities among elite runners, there remains a subtle range of different postures and ways of moving that constitute the biomechanical “signature” of each athlete.

But whether beginning runners can be "taught" to run "properly", or like faster runners (which I'm completely convinced that they cannot) is not the point. Every runner, whether fast or slow, young or old, must adapt a way of moving that is in accord with his/her particular body. Since running is as innate and basic a form of human movement as walking, the most immediate way to find the line of least resistance in doing it is simply to do it very often. This is precisely how young runners go from gangly foot-stompers to compact and light-striding speedsters over a matter of a few years, without a single minute of "technical" instruction. And, if nature or the civilized life has left one's body in a poor state for beginning to run, or too much of one kind of running has created weak points and asymmetries, then some care must be taken in starting a running program, and in addressing basic muscle weaknesses and imbalances through targeted strength training (something that must remain a standard feature of any athlete's training regime as its demands grow). The goal of every runner must be to adapt the most efficient way of moving for his/her unique body. Distance runners can certainly benefit from general “postural/form cues”, such as “run tall”, or “drop the shoulders”; but, they should look askance at any claim that there is one optimal running form for everyone, and that it can be acquired through technical instruction. (Matters are quite different for sprinters, who are concerned not with physiological efficiency, but with sudden, maximum power output. Here, technique is crucial, and is subject to conscious adjustment in racing and training).

\bigskip

$^*$The Globe’s Life section has done an excellent job lately of covering our sport. We’d preferred to be covered as a sport, of course, but we’ll take whatever can get at this stage!
