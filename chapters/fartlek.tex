\chapter{Fartlek: The Workout That Dare Not Speak Its Name}
\chaptermark{Fartlek}
\textit{Tuesday, 17 November 2009}
\bigskip

If you're looking to feel silly and slightly uncomfortable, try saying the word "Fartlek" to a group of primary school students. Even if you're talking to a group of keen young runners, it will do no good to follow this up by explaining that the word is actually Swedish, meaning "speed-play". The point is, you will have said the word "fart" without any comic intent, and they will find it hilarious.

In its original form, fartlek running was defined by the use of spontaneous changes in speed introduced within the course of an otherwise easy, aerobic training run. Its Nordic inventors intended it as a kind of hybrid of easy, recovery-pace running and formal interval training; and, like a lot of training techniques, it was given rise to by a combination of necessity and opportunity. Fartlek training was born in the forests of Scandinavia as a means of taking advantage of the opportunity that the natural environment afforded, and of making do without easy access to running tracks or stopwatch-bearing coaches (there was, recall, a time before convenient, affordable and easily portable hand-held timing devices). Its pioneers also intuited that it was perhaps a more accurate simulation of the precise demands of actual long distance races, and of off-track races in particular, than the then standard track interval session run at faster than race speeds and with more passive recovery periods. Early fartlek sessions would have athletes running freely and picking their own landmarks between which to do pick-ups of varying speed and length. Later, with its broader international dissemination, fartlek would become more formalized in terms of the length and intensity of the accelerations, and more tailored to the needs of athletes in specific event ranges. It remains, however, an ideal way to combine the volume of a longer, easy run with the intensity of a track interval session, as well as an occasional alternative to the grind of standard interval and tempo sessions.

It it important to understand, however, that fartlek training is not a form of compromise between two more ideal forms of running-- easy recovery running and hard interval training, or simply a psychological respite; it is its own form of training, and it offers its own unique psychological and (I think) physical benefits. Fartlek training is ideal preparation for longer, off-track races in particular. What makes it ideal in this respect is the imperative to recover on the fly, to accelerate when already running at a fairly high heart and respiration rate, and to focus throughout a longer, continuous bout of running. And the top speeds in fartlek workouts are typically no faster than those reached in a race of 5kms or longer, with the average pace in a good session frequently matching exactly the athlete's proper tempo run pace. The active recoveries and the typically longer duration of the fartlek session tend to prevent athletes from ever approaching their 1500 or 3k paces, forcing them to spend more time at their actual long distance race paces rather than above or below them, which frequently happens when the training plan includes only easy runs, interval sessions and tempo runs. So, while the foundation of any correct training plan remains the MV02 interval session and the tempo run, punctuated by the easy, aerobic run, the fartlek session remains a vital adjunct. It provides both psychological respite from these other kinds of sessions; but, more importantly, it offers a useful simulation of the physical and mental demands of long distance racing, which include the ability to respond to mid and late race changes in effort and speed, and the mental discipline to maintain focus under prolonged stress.

As most of my athletes will have learned, I have a few favourite fartlek sessions, including:

\begin{itemize}
    \item the 60/40, in which athlete runs for 60 secs @ 5k race pace and recovers for 40 seconds at or slightly faster than typical easy run pace; and
    \item the 6 to 1 "hybrid" tempo and interval-pace session, in which the athlete completes a series of runs descending from 6 minutes mins down to 1 minute, taking 1 minute recoveries @ typical easy run pace between each segment, and attempting to increase his/her pace from tempo speed to down to interval speed in the final 3 segments of the session.
\end{itemize}

The fartlek form, however, allows for infinite variations, and I never tire of inventing and self-testing new combinations of speed, recovery and total volume. And the introduction of different terrain expands the possibilities that much more.

So ingenious is the idea contained in fartlek running that it was bound to be invented at some point. It's just a little unfortunate for us anglos that the Swedes, whose word for speed happens to be "fart", got there first!

