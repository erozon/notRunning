\chapter{Jack Daniels Primer \#1 -- Daniels for the Ages}
\chaptermark{Daniels \#1}
\textit{16 March 2009}
\bigskip

As it happens, I had been re-reading sections of senior U.S. coach Jack Daniels’ 1998 classic Daniels’ Running Formula (2nd Edition in 2005) and thinking about starting a regular section here in the blog dedicated to explicating some of the key elements of his approach to training, when someone sent me a link to a series of video interviews with the man himself that have been running for the past month on the running website Flotrack, entitled Thirsty Thursdays with Jack Daniels. This was all the extra impetus I needed to get down to it. I want to begin by saying a little about Daniels’ unique approach to periodization and conclude with a comment about his personal style as revealed in the charming little Thirsty Thursday segments.

Approaches to periodization (i.e. the yearly cycling of training emphases used by coaches to promote the continuous, all-round development and timely peaking of athletes) have been changing somewhat over the past few years, but there remains a fairly widespread and long-standing consensus among coaches—one that probably has its roots in the ground-breaking theories of the great New Zealand coach, Arthur Lydiard—concerning when to introduce faster paced training (i.e. at mile race pace or faster). In most programs, particularly school-based ones, faster running is typically introduced immediately preceding and even during the main racing season. The reasoning here is fairly intuitive: You run your fastest speeds in training around the time you want to run your fastest speeds in competition. As it happens, the only coach I’ve ever discovered who has systematically challenged this common sense is Jack Daniels. Daniels programs typically recommend that majority of the faster running be done during the earlier training phases (during the winter months in most of North America), with longer and slower tempo runs and less intense “cruise” sessions during the actual racing season. The time in between is taken up with the most intense kinds of training-- longer intervals. But, why would a runner what to run faster in training at a time of year when his important competitions—the ones he wants to be most “sharp” for—are still months away; and, why would he choose to train less intensely at a time of year when intensity would seem to be at a premium?

To understand Daniels’ answer to these questions one needs to understand his famous “formula” itself. The genius of Daniels is that he has been able to, with considerable accuracy, isolate the physiological adaptations associated with training at particular speeds for given durations. For practical purposes, he isolated four different running speeds that, in general, provoke four different types of adaptations in the trained body. (In the interest of brevity, I’ll leave it to the reader to pick up his/her copy of the book or go on-line to get the details here.) According to Daniels, the benefits of faster running are that it encourages the development of optimal biomechanical “running economy” through the promotion of strength, balance and relaxation. For Daniels, then, faster running—which must necessarily be done with longer recoveries, in order to prevent fatigue from reducing one’s speed— is best used as a basis for the most intense and race-specific kinds of training that runners will do—i.e. “interval” training, which involves running at around 95\% of maximum aerobic capacity for up to 15 minutes total in a single session. Ideally, he reasoned, a coach would want to position a cycle of faster running before a period of more intense interval training in the yearly scheme, so that an athlete enters this most difficult and potentially risky phase of his training with optimal bio-mechanical economy; thus, faster running should seasonally precede interval training, and less intense "cruise" training should be the mainstay of the competitive phase.

To this imminently sound bit of reasoning I would add the following argument for keeping faster running out of the racing phase, particularly for school-age runners, who are typically competing at the middle distances: Racing itself involves faster running. During a racing phase, runners don’t really need any additional familiarization with the feeling of trying to run fast, and adding yet more faster running into the program while athletes are already running all-out once or twice per week--and tend as a result to be feeling highly charged and motivated to compete-- risks compromising their racing performances, and may put them at greater risk of injury. My approach is to allow the races themselves to provide any necessary re-familiarization with the feel of faster running (or, “sharpening”), keeping the workouts less intense. Ideally, by the time the racing season arrives, most of the benefits of both hard and fast training will already have been realized (the “hay”, so to speak, will already be safely “in the barn”), and there should be no need to worry about fitness loss through reduced training intensity. While ideal for age-class runners, who typically have a well defined competition phase, I've found this general approach to be highly effective for older runners training for the longer distances too.

It has therefore been my practice for years to break with the conventional approach of assigning longer and slower training in the winter and shorter and faster training in the spring and summer. For younger athletes and middle distance athletes in particular, the period December to the end of February is taken up with faster hill repeats and track sessions at mile/1500 race pace, which immediately precedes the hardest training of the year—the six to eight weeks from early March to late April, which are taken up with longer track interval sessions and tempo/fartlek workouts.

In a rare section in the book on youth development, Daniels even suggest that this “speed before intensity” approach to periodization is perhaps the best basis for the macro-cycling of young athletes. What holds for a single season, he suggests, might also hold for an athlete’s early years as a whole. It is perhaps best, he argues, for young athletes to spend greater periods of their early years trying to run fast than trying to go long and hard. Coaches of young athletes, he suggests, might consider keeping total running volumes and intensities low in the early years and increase them step-wise as an athlete matures.

Anyone who reads Daniels book, or watches him in any of the above mentioned interviews, however, will discover that, with him, there are few hard and fast rules. Daniels has firm and well supported opinions on most running matters; but, at 76, and after more than 40 years of total immersion in the science and culture of distance running, he has developed a gentle and patient touch (perhaps this was always his style). When asked in one of the video interviews about his main goal as a coach, he answers, in effect, that he would simply like to encourage athletes to want to train by creating an appealing environment within which to do so. It would seem that, after all these years, he has come to realize that the genius of coaching resides less in the details of program construction and more in the ability instill an enthusiasm for the training process. As a career exercise physiologist, he clearly would not want to discount the value of sound methods; but, he seems to have realized that a welcoming environment, social and otherwise, is perhaps just as essential for keeping athletes committed to the longer term-- a vital consideration in this most demanding and patience-testing of sports. In these interview segments in particular, Daniels exudes a personal warmth and self-effacing humour that belie his now great authority in the sport. At the very moment when demand for his advice is at its peak, and when he could, if he chose, wield his influence with considerable force, he appears content simply to gently advise, and to speak only when asked. He appears, at his great age, simply to enjoy the privilege of spending time in the presence of runners. To my eye and ear, he offers a model of aging gracefully and well, and presents a living testament to the benefits of a life driven by intellectual curiosity and intense social engagement. He just happens also to be one of the world's foremost authorities on running.




