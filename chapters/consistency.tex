\chapter{How We Get Ahead in Running: The Meaning of "Consistency"}
\chaptermark{The Meaning of Consistency}
\textit{Thursday, 27 December 2012}
\bigskip

You've heard it and read it in the running magazines dozens of times before: the key to getting to the next level in the sport, whether you're young or old, is to train consistently over a long period of time. We know this means, among other things, that doing a moderate amount of running for a few months beats doing a few prodigious bouts of training, followed by little or no running at all (sometimes due to injury or illness, but also due simply to loss of interest in the aftermath of the binge itself). We also know that the longer we can go without significant down time-- several years, if possible-- the better we'll get. My experience as both athlete and coach, however, has taught me that things are not quite so simple. The real truth, once again, resides in the details-- details that are discernible only through closer inspection of what successful athletes actually do, and have done for years. For instance, while it's true that many successful athletes, myself included, have had very low rates of injury over long stretches of their careers, it is also the case that many equally successful athletes have suffered more than one setback lasting 2-4 months in the span of a few years during their careers. And while many of us have kept our total weekly training volumes within a fairly narrow band over a decade or more, with the most successful tending to do slightly more volume on average from one year to the next, plenty of top runner's diaries record quite large fluctuations within these steadily increasing yearly averages. What does this seemingly conflicting evidence tell us about the meaning of "long term consistency" as it relates to training for distance running?

The very short answer is that consistency simply means not giving up (of which more below)! It is possible to have a successful career in this sport on the basis of both a very uniform pattern of training from week to week and year to year, and a pattern characterized by wide fluctuations in total training load and intensity, either by design, or as a result of injury and other setbacks. Where possible, uniformity and careful incrementality is always to be preferred to its opposite, at least in the long term. The trouble is, it is rarely possible, because the variables upon which it depends are often very difficult to control. The trained body can be capricious and therefore hard to read; and, in many places in the world, the seasonal weather can be distinctly uncooperative. All runners must therefore deal to one degree or another with shocks and interruptions to their carefully plotted plans, and some runners must deal with regular and multiple such disturbances. Common sense may dictate that, in the face of challenges to consistency, coaches and athletes must double-down on their attempts at stabilizing the overall training plan. There may be something to be said, however, for accepting, and even attempting to exploit, the pattern of short term inconsistency imposed by life's tribulations.

I have always intuited that it is sometimes best to "strike while the iron is hot" when formulating a training plan-- to, in other words, increase volume and/or intensity when it's obvious that my or my athlete's response to the stimulus is robust. This intuition coalesced into an actual insight the other day during a post-workout chat with a particularly fast-improving young athlete. His just completed workout had been significantly faster at the same perceived effort than a similar session only two weeks earlier. When he suggested that he must have simply pushed a little harder this week-- because he "couldn't actually be any fitter in only two weeks"-- I began to think a little harder about my own experience. When I was improving (and it's been many years since I have actually improved), how did it typically happen? My recollection-- backed up by my old training logs-- indeed showed that, when I improved, it tended to be in sudden bursts of 6-8 weeks, followed by longer periods of relative stasis. There were definitely times when I really did get fitter and faster than I had ever been in a period of just a couple of weeks, often following a period of longer and more intense training than usual. My pattern over many years was therefore not linear at all. It tended to be more a matter of improving in short bursts, then simply hanging onto those gains until the next burst. It was the hanging onto established gains-- the never giving up part-- that really distinguished my career as a whole.

Then I stumbled upon the following post from Alex's Hutchinson's consistently outstanding blog on exercise science in which he reviewed a piece of research on the possible benefits of "killer training weeks"-- the kind typically done during university and club training camps. This caused me to recall the many times I had seen and heard of athletes suddenly reaching new levels of performance in the wake of breakthrough, one-day efforts in half-marathons and marathons (although much more commonly half marathons). Perhaps there really was something to be said for abandoning routine-- and even common sense-- once in a while in pursuit of that big breakthrough!

But then my more prudent, analytical side intervened and offered some needed balance. If we could always know with certainty in every case when the best odds of successfully pushing the envelope in training would occur, then the problem would be simple. Absent such information, however, the idea of introducing "super-weeks" of training into a typical routine seemed almost always ill-advised, and more likely to end in injury and over-stimulation than in a performance breakthrough. A more sensible use of the above insight would seem to be to recognize that improvement typically does happen unevenly, and to discover the points in a individual athlete's various cycles (seasonal, yearly, and career) when such breakthroughs were most likely to occur. Planning heavier (perhaps even much heavier) training during these times might prove beneficial, provided the full understanding and commitment of the athlete could be enlisted (a crucial variable).

Finally-- and here we return to the importance of simply not giving up-- we need to aware that periods of much heavier than normal training do not always produce the intended and desired result, even when very carefully planned. In fact, sometimes they result in the opposite-- injury, illness, and loss of performance. In successful athletes who have endured multiple setbacks related to injury and illness during their careers, we notice a couple of distinctive things. These kinds of athletes-- and here 2:07 marathoner Dathan Ritzenheim of the US is perhaps history's best example-- are typically very diligent and creative cross-trainers who rarely miss a beat when misfortune strikes. Successful but oft-injured athletes are also typically very good at learning from their mistakes. They will often adjust or completely overhaul their approach to training in the aftermath of an injury or period of poor performance. Very successful athletes are therefore not always "consistent" as much as simply determined! And they often learn to use the pattern of "inconsistency" that nature, fortune, or circumstance imposes on them in very productive ways. Years ago, master coach Jack Daniels discovered this fact inadvertently when he conducted follow-up physiological testing on a group of elites he had studied while at their career peaks. To his surprise, he found that those who had suffered the greatest number of setbacks, due to injury or whatever, tended to have retained the greatest percentage of their peak-age fitness. Whatever the reason (Daniels thought perhaps some combination of the down time from training and the extra motivation that might have come from feeling like they had never trained well enough to realize their full potential), these athletes managed to turn a life time of short term "inconsistency" in training into consistently high long term fitness. There are lessons here for a all of us, young and old.

