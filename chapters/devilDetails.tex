\chapter{The Devil in the Details}
\chaptermark{Devil in the Details}
\textit{Monday, 21 September 2009}
\bigskip

Why do some athletes who seem to have everything going for them-- a knowledgeable coach, a solid program, and plenty of apparent natural ability-- nevertheless seem to consistently under-perform, and generally fail to develop their long term potential? And why do others, sometimes using the same coach and program, and with the same, or even sometimes less, apparent basic ability, go on to enjoy long and successful careers? Could it simply be a matter of luck?; the unseen and genetically determined "trainability" of some athletes and not others?; or, perhaps, could be it be some athletes' genetically determined proneness to injury and illness (another type of basic luck)? The truth is, there is no way to know for certain the factors that explain success and failure in athletes who appear to be equally endowed. Such is the complexity of the training process-- which is also, we can't forget, a "living" process, given the complex imbrication of our training with our everyday, non-running routines. Nevertheless, while a fully scientific answer to this question may elude us, a look inside the daily lives of a very large numbers of athletes-- such as I've been afforded in the course of a 30-year involvement in the sport-- offers certain hints. If fact, my experience strongly suggests that some athletes, for whom all other things seem roughly equal, enjoy dramatically different degrees of long term success because of the way that they manage the seemingly minor details of the training process-- what we might refer to as the supports of the training process, as distinct from the actual business of completing runs and workouts.

That the devil of repeated failure seemed to reside more often in the details of an athlete's quotidian routine became apparent to me as I began to look more closely for an explanation of my own periodic failures and set-backs. And my conclusion was gradually confirmed through subsequent observations of the daily habits of the many other athletes with whom I would come into contact over the years, whether as a friend, competitor or coach.

Using my years of detailed training logs, I was able to discover that there was invariably a moment, usually in the form of a single poor decision, such as an ill-timed or too intense workout session, or late night out, that triggered a series of events (often involving further poor decisions) leading to the periodic collapse of my training or racing. When I was younger, some of these initial episodes of bad judgment had simply to do with lack of knowledge or experience (I was, after all, largely self-coached). Later in my career, however, these bad decisions, when they occurred-- and they occurred less frequently the older I got-- had less to do the not knowing than with my occasional impatience, complacency, corner-cutting, or ill-advised risk-taking. I had, in particular, a marked tendency to force my return to hard training following a bout of illness (particularly common when my children were small). To this day, the vast majority (something like 90\%) of all the injuries I've suffered have been sustained within a week of a lay-off caused by viral illness. After something like the 3rd or 4th repetition of this pattern, I began to discern it as, in fact, a pattern; nevertheless, I still occasionally took undue risks following minor illnesses, although, of course, usually without incident. It was the fact that I didn't always get hurt following a cold that lead me to take the risks I did, and the more time that elapsed between my last illness/injury episode, the more I would be willing to take the risk of doing a hard session during or immediately after a viral infection. Of course, every time an injury did occur, and I was forced into the pool or onto the elliptical again, I would realize that I had made the same mistake yet again, and I would remind myself of how stupid it was to risk losing 10 or more future workouts-- to say nothing of the tedium of obligatory cross-training and rehab-- in order to save just one workout this week. All the higher level sophistication and general determination in the world would amount to nothing, I would eventually conclude, without proper attention to the details. Like Achilles with his heal, I surmised, we were no stronger than our weakest link-- which was more likely to reside in some apparently trivial detail than one of our basic training principles.

As it would happen, I would sustain relatively few injuries or other set-backs in my career, and would enjoy long stretches of successful racing right into my 40s. And, the older I got, the more I realized that my long term success had more to do with my ability to recognize and attend to the smaller details of my program than with, say, "good genetics", or some other form of happy chance. In spite of my occasional tendency to take risks around illnesses, I began to realize that I must have been getting most of the details in my training and general preparation right most of the time, and probably more often than many of my equally talented and similarly hard-training, but more oft-injured, competitors. Sure enough, as I got to know some of my competitors as good friends, and saw first hand the many small lapses in good judgment they frequently made-- from pushing through workouts while obviously (to me, anyway) in the early stages of injury, to refusing to re-schedule or abandon workouts when obviously over-tired, to neglecting their strength routines-- I began to realize that, in many cases, their apparent "bad luck" had a much more specific cause: poor day-to-day judgment and lack of attention to the detailed supports of their training processes. And, on the other side of the coin, I began to notice that athletes who were doing better than I was were most often the ones who had developed, and were successfully adhering to, even more sophisticated support systems. Eventually, and finally, my entry into coaching would convince me that the devil of repeated failure was most often in the details of an athlete's training process.

Becoming a coach has entailed developing a familiarity with the personalities and daily habits of people normally reserved for psychologists and immediate family members! In addition to leading to some deep and abiding friendships, it has been an indelible lesson in the importance in athletic success of managing one's day-to-day affairs and controlling one's occasionally counter-productive impulses. Everyone I have ever coached has professed a keen desire to succeed; yet, some have proven much better at attending to the small threads in the fabric of their training programs, which, if allowed to come loose, will lead to the unraveling of the entire cloth. It is these athletes who have tended to enjoy the most long term success. Proper attention to most of the details in question-- from getting good sleep on a consistent basis, to managing illness, to maintaining a minimal core strength routine, to communicating with me immediately about possible injury problems (and following my advice), to simply following the training program as written-- is well within the realm of the practical for the average athlete, no matter how time-pressed. In fact, my busiest athletes are frequently the most diligent in the management of their daily support routines. Those who don't manage these crucial details effectively, I have come to understand, simply do not, deep down, believe they are as important as they are to their over-all success. Some who have not always been good at managing such details have, over time, and as a result of bitter experience, learned to become better at the job (much as I myself did). For others, on the other hand, a tendency to want to "get on with it" and never mind the fuss, is an ingrained trait of personality. In these instances, consistent success and long term talent development remain frustrating up-hill battles. This kind of athlete is, in fact, the most likely to abandon the sport prematurely, blaming "bad luck", in some form or other, for their failure to thrive.

When it comes to negligence in the maintenance of proper training-support routines, special comment must be reserved for the teenage athlete. Proper attention to details like rest and nutrition requires the ability to understand the link between present actions and future consequences; this is an ability that most teenagers lack, simply because they are teenagers. Nonetheless, some young runners take lack of attention to detail and generally bad day-to-day judgment to new heights. Teenage runners generally want to succeed as much as adult runners-- perhaps more-- and their decision to pursue this most difficult of sports marks them as a special breed within their age cohort; nevertheless, the same kid who will complete every run and workout without fail, will also, without warning, decide to stay up half the night partying the week before his most important race of the season, and while already suffering from a cold! The teenage athlete is also remarkably difficult to sell on the merits of proper nutrition, strength training, and even simple injury rehab, such as icing or stretching. They are also sometimes reluctant to obey basic workout instructions, preferring, lack of experience notwithstanding, to do things their own way. And, amazingly, some asthmatic teenagers will repeatedly forget to bring their inhalers along to workouts and races, even when the simple, side-effect-free, administering of said medication means the difference between success and failure. In short, teenage athletes are often remarkable in their ability to confront the bigger challenge of being runners-- the regular completion of workouts and runs; but, just as often, they are reluctant to register the importance of getting the details right. Unfortunately for them, this makes them an excellent negative example for all of us.

In the end, I'm convinced that most unsuccessful runners of any age or basic ability level (from potential elite to age class recreational) are undone by a failure to do what they know, or ought reasonably to know (because the have probably been told!), is correct than by unknown variables, such as their basic genetic inheritance. Failed runners often speak of injury-proneness, or the basic, genetically-determined inability to handle the required training loads, in accounting for their troubles. And there are, of course, rare examples of otherwise genetically blessed athletes whose bodies are in some other way irreparably flawed, causing them to break down under the burden of even modest training. Upon closer examination, however, many more unsuccessful runners have failed because of their own repeated lack of attention to the important details that sustain any training effort, and because of a repeated inability or unwillingness to learn from their mistakes. In my experience, most runners are actually capable of training much longer, harder, and more consistently than they ever have; yet, many cannot progress because of a repeated failure to attend to the seeming minutiae that so often make the difference between success on the one hand and injury or poor race performance on the other. I conclude, then, with my list of the most common neglected areas of detail among runners, young, old, elite and average. These are mostly simple and easy to manage variables that most reasonably experienced runners know are important but nevertheless often neglect:


\begin{enumerate}
    \item The treatment and proper rehab of common injuries, including the timing of return from injury.
    \item The management of effort levels on a daily basis (i.e. failure to reign-in the very common "harder is always better" impulse).
    \item The maintenance of a simple strength program to shore-up known areas of weakness.
    \item Attention to basic nutrition (a fast improving area, it must be said).
    \item Attention to sleep requirements and sleep problems (sometimes more complicated, admittedly).
    \item Deciding when and how much to race (many runners enter races for the wrong reasons, at the wrong times, and generally race too often).
    \item The choice of pacing strategies in races (many people insist on exemplifying Einstein's definition of insanity when it comes to their choice of racing tactics-- to wit: Repeat the same failed strategy over and over again with the expectation of different results).
\end{enumerate}
