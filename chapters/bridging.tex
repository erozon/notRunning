\chapter{Bridging the Seasonal Training Divides}
\chaptermark{Seasonal Training}
\textit{Monday, 23 March 2009}
\bigskip

Any serious runner who trains in a true four-seasons environment will know that different climatic conditions produce their own unique challenges. Summer requires acute attention to proper hydration as well as the risks of heat stress and skin damage, while winter demands some ingenuity when it comes to dress, footwear and workout planning, and may even require complete escape via the treadmill, indoor track or, for the elite athlete, warm-weather training venue. Seasonality, however, offers another challenge that many athletes may not have taken sufficient account of-- the challenge of navigating safely between the major seasons.

In my long experience as an athlete, the most dangerous times in the training year where sudden injuries are concerned are the "transitional" months-- March and December at most northern latitudes. During my peak years, I was lucky to have been almost completely injury-free; however, of more serious injuries I did sustain (those requiring more than a week of down time) all occurred during these two months, with 3 occurring between March 10 and 17! I've been less systematic in keeping track of my athlete's patterns of injury (there being relatively few injuries to deal with in a given year, thankfully), but I think there are reasons for all runners to be cautious during these months. And, paradoxically, there may be even greater reason for athletes who have trained well through the winter to be wary in the months of March and April in particular.

Almost all running injuries are caused my mal-adaptations of one kind or another-- that is, by a failure of the body to respond properly to the introduction of a training stimulus. The most obvious reason our bodies fail to adapt properly to a training stimulus, breaking down in injury instead, is the speed and extent of the introduction of the stimulus in question. For instance, the average runner will become faster with the introduction of higher training volumes and more intense bouts of training; but, if increased volume and intensity are introduced suddenly, the body's capacity to adapt to the new stimulus may fail, usually at the level of our muscles or tendons, resulting in the debilitating pain commonly referred to as an "overuse" injury (but perhaps more accurately referred to a "failed adaptation" injury, as most runners can handle much more "use" than their experience tells them, provided it is introduced gradually enough). The changing seasons, and particularly the transitions into and out of winter conditions, entail some sudden changes that can overwhelm the body's adaptive responses and lead to injury.

The transition to winter running, first of all, entails an often very sudden move from running on softer surfaces to running on pavement and/or ice and snow. The onset of winter for many also occurs at the end of a hard racing season, which is often followed by a break from, and the subsequent reintroduction of, training. These and other factors, such as the greater likelihood of becoming sick with a virus, the stress of preparing for seasonal social gatherings, and even, for some, seasonal mood disorders, make December a minefield of sudden changes for the serious runner.

The transition from winter to spring-- often just as sudden-- means the generally safer move from harder to softer running surfaces, but holds its own set of risks, particularly for the runner who has trained consistently through the winter. During the first blissfully warmer days of spring, it is easy for the fit runner to unintentionally increase his average training speeds, often before access to softer surfaces has become available. The winter-fit athlete, champing at the bit for the start of the spring racing season, may also begin to attack his harder sessions with renewed gusto. Winter-weary runners may also attempt to will the onset of spring weather by dressing for it at the first sign, suddenly exposing working muscles and tendons to single-digit temperatures.

No matter how well prepared for these transitional months, they will always present special risks. At no other time of year do we confront so many basic changes to our training regimes that are not of our own initiating. And, if we are rash enough to introduce sudden additional changes at this time of year, we are doubly at risk. The best defense against the pitfalls of the seasonal transition is added caution and heightened awareness. Runners should take new aches and pains more seriously at these times of year, and be prepared to make a hasty detours onto the elliptical trainer or into the pool. Buying a little extra time to adjust to these rapid changes in our training environment while trying to maintain as much of our hard won winter fitness as possible is the surest way to bridge these seasonal divides and stay on course to meet our spring and summer racing goals.




