\chapter{Hopping Along: On Beer and Running}
\chaptermark{Hopping Along}
\textit{3 April 2019}
\bigskip

I had my first swig of beer-- a thirsty, breathless pull on my dad's pint as he sat in a lawn chair and I ran around the yard under a scorching July sun-- at the age of 8. I can't say I liked it, but it was cold, fizzy, and at hand. A few years later, I discovered competitive running. I can't say I liked it either at first, but it was available, and I was good at it, kind of. Both tastes, I would discover, once acquired, can exert a powerful hold!

By the end of high school, I had drank a few beers of my own and run a lot more miles. I liked the miles a lot more than the beers. Beer was, in those days (and for some today, still), a cheap and very simple alcohol vector-- one whose semiology was, and still is, "fun" and "sociability" rather than either "effeteness" (see wine and single malt scotch) or "solitary alcoholic dissolution" (see gin and blended whiskey). Beer was for younger men, and it was for binging on. No one I knew actually liked the taste of it-- and for good reason. To say that it was tasteless was to compliment it; it was only "tasteless" when it wasn't aggressively awful-- which, due to both poor quality control and complete inattention by the brewer, it often was. Even at its best, it could not be imbibed comfortably unless at palate-numbing temperatures (hence Coors "cold" mantra). It managed to be somehow simultaneously weak-tasting and disgustingly bitter. It also tended to leave a sticky, gag-inducing film on the back of the throat. The styles of the beer could sometimes be found on the bottles, but nobody cared enough to look (it was always just "ale" or "pilsner"). Though most beer drinkers had brand loyalties-- some to the point of self-identifying, hat and t-shirt wearing , fanship-- everyone knew damn well there was no telling them apart by taste. Girls often wouldn't even attempt to drink beer-- because, apparently, they had no need to exhibit their manhood by choking down swill without grimacing. Because I didn't like this "beer" enough to force down a bottle (and liked even less the high school "parties" where binging on it was more or less a requirement-- and how else to cope with the adolescent sexual anxiety that these gatherings were both a cause of and a response to? But that's a whole other story!), I didn't have much to do with beer till I rediscovered it in university. (Queen's University, as it happens, which would become home to the "beer mile" years after I left. Of which more below).

And it was in university that beer and running began their association for me. The beer predominantly on offer was still gross, of course, but interesting alternatives-- the occasional stout or Belgian-- were beginning to appear in the more hop-curious drinking establishments that, now that I was of legal drinking age, I could enter without fear of ejection (and you could still drink underage-- sometimes WAY underage, particularly if you were female-- in those days). The average undergrad party still featured gallons of crap beer guzzled at top speed, sometimes with mechanical aids, or by exploitation of the physics of canning-- then often regurgitated later by means of similar physics. But, if you were a varsity athlete, and a serious one, drinking beer (and, yes, getting drunk on it) offered a new basis for bonding with one's "people". The beer or two with dinner in the pub following a weekend race or hard workout, or the blow-out after-party following a long competitive season, replaced the lonely and awkward choking down of pints at a status-conscious campus party or a desolate, deafening, chrome-and-mirror dance club. Even crap beer was drinkable on such occasions-- and more interesting beer, such as a pint of stout, an amber ale, or a yeasty, boozy Belgian tripel-- could be downright thrilling. I was just beginning to discover that well crafted beer consumed when exhilaratingly tired and/or several hundred calories in deficit could activate dimensions of taste that seemed to transcend the physical.

For me and millions of others, however, the full potential of the running/beer nexus would await the craft beer revolution, which was organizing at the cellular level in parts of the U.S. by the time I left university in the late 1980s, but was still a decade or so from storming the corporate beer Bastille. In Canada, the first whispers of change came in the form of beers that were made without adjuncts-- in other words, with just hops, water, yeast, and grain. Lagers and pale ales and a couple of different stouts were still the only styles widely available, but the idea that beer could be more than a bitter yellow water created for the benefit of young men looking to get drunk while remaining upright, or of middle aged men trying to nurse a respectable, functional alcoholism, had been broached. It was when I was approaching middle age myself, and still training seriously, that true craft beer invaded the mainstream, bringing the range of styles and the intensity of flavour now familiar even to the casual beer drinker. By the turn of the millennium, beer had come full circle, returning to its local, artisanal, and experimental roots. In so doing, it had become as interesting as wine or scotch, but without the price and high ABV. Beer was now something that could be consumed for reasons other than its alcohol content, and in a style similar to that of its effete cousins-- daily, in small quantities, and as a compliment to food and other things, such as running!

\bigskip
Yes, running.
\bigskip

Beer and running had a connection long before the craft beer revolution, but in the form of the post-race beer keg ritual. Many races, big and small, would crack kegs of cut-rate lager for participants, who would swill it from plastic cups while standing around aglow in their sweaty gear, discussing performance versus expectations in minute detail, the way we runners are wont to do. In this context the "beer-ness" of the substance being imbibed was largely irrelevant. Then, a few years later, there came the invention of the "beer mile"-- a contest involving an equal number of beers consumed and laps of a standard running track completed (4) as quickly as possible. This beer/running connection was forged, unsurprisingly, by college-age men-- men, one suspects, becoming a little bored with both running and drinking beer, and deciding to combine the residual thrill of each into a single activity. Beer-miling came into something of a vogue when it attracted the attention of nostalgic post-collegiate and middle-aged men (and a few women) looking to see if they still had it in the zaniness department (some still did, but that's for others to discuss). Despite many invitations, I was never enticed to attempt a "beer mile", and the very thought of combining running and beer in that way became off-putting in direct proportion to my increasing interest in and love of craft beer. At the risk of ending up at odds with my many friends who "beer mile", I maintain that drinking beer competitively and combining it with a foot race is a acute insult to both venerable activities.

I suspect I'm not alone in this view, however, as the worlds of beer-miling and serious beer appreciation tend to be entirely separate ones. The beer in beer-miling is selected for its ease of very rapid swallowing and for its intermediate ABV (above 5\%, as per the now official rules of this "sport"), such that the liquid in question might just as well not even be beer at all-- except for the ironically comedic effect that the whole thing seems to trade on ("runners are supposed to be fit, and here they are guzzling beer!").

The beer mile has indeed worked as popular spectacle to the extent that it has (and it now features a championship circuit, prize money, and sponsored athletes!) by playing on the mainstream N.A.R.P (non-athletic-regular-person) misconception that seriously fit people, and especially distance runners-- that most hair-shirted of athletes-- tend not to drink alcohol at all, and least of all something as plebian as beer. All my adult life, the admission that I quite like beer, and drink it often, has been met with bemusement by non-runners. I have a feeling that if I said I liked wine the reaction would be different (for then I would be some kind of connoisseur and not just someone who likes to get drunk). Even today, however, 20+ years into the craft beer revolution, professing a love of beer often conjures up images of the drunken, Everyman buffoonishness of a Homer Simpson or of Canada's own brothers Mackenzie. Beer, it seems, is still primarily for chasing away one's workday problems, whilst wine and spirits are all about tasting (or else the supposedly more refined binging of the middle and upper classes-- see e.g. the whole "wine lady" shtick). The idea that a person perceived as not just casually active but as competitively fit would profess a love for beer (read: easy drunkenness) still seems somewhat improbable to many. The advent of craft beer, in all its seemingly endless styles, flavours, and ABVs promises to completely disassemble this complex of assumptions. And I plan to help it along by explaining precisely why I, along with surprising-- and growing-- number of other serious runners, think beer and serious running are not only compatible but highly complimentary.

And some of the reasons why quality beer may be one of running's greatest compliments (e.g in the same category as things like dirt trails, lightweight textiles, and GPS watches) are as follows:

\begin{enumerate}
    \item When brewed expertly, and with concern for character and body, beer is closer to food than beverage. While runners have traditionally guzzled watery pilsners and lagers to quench their thirst, craft beer aficionados have discovered that beer is best enjoyed after full hydration with water and before food calories are introduced in earnest (small snacks are fine). When completely hollowed out from a long run or workout, and just as the familiar post-run appetite suppression has lifted, the consumption of a robust and well balanced beer can be akin to a religious experience. The post-run, full-body hunger primes the taste buds to a state completely unknown to the sedentary person, enabling the drinker to experience the subtle complexities of flavour imparted by malt, yeast, and hops in their most vivid detail, and in the way that food, with it necessity to masticate, can't quite match. Drink good beer within an hour of running and you will drink it forever. (Best post-run beer styles are IPAs, sour ales, and some saisons).

    \item  Further to the above, beer is very high in carbs compared with the alternatives (wine and spirits). A little beer within an hour of your run will, along with healthy nutritional, beer-friendly, companions like cheese and crackers or carrots and hummus, aid your recovery.

    \item One of beer's principle ingredients is hops, and hops contain a brain-calming, soporific chemicals used for centuries as sleep-aids . In contrast to the poison-shock affect of other alcoholic beverages, beer will not fire you up (i.e. if sipped in small quantities and not guzzled). Timed properly, a little beer can be an effective tool in your post-run wind-down process. For this reason, it is never wise to consume beer before or between runs!

    \item Unlike the alternatives, beer comes in a wide range of ABVs (from kettle sours at 3.5\% to imperial stouts at 12-15\%) and in very portable single-serving packages (i.e small bottles and, increasingly, cans). Conventional wisdom about the incompatibility of serious fitness pursuits and alcohol consumption is, to some extent, correct. Timed poorly or in excess quantities, alcohol consumption WILL undermine your training and overall performance, mainly by interrupting your sleep and, in the longer term, impairing your general health. This makes beer the ideal alcoholic beverage for the serious runner who chooses to imbibe. Spirits are too alcoholic, and they don't address the aforementioned body hunger, and wine-- also relatively high in alcohol-- typically comes in larger bottles that, if opened, will tempt overindulgence.

    \item Beer, even the very best on offer, is much cheaper than alternatives of equivalent quality. And we all know that serious runners, if they are not hovering on the edge of penury, are a frugal bunch.
\end{enumerate}

Finally, a general note about alcohol consumption and athletic performance. The very long list of successful athletes who drink alcohol suggests that moderate consumption is compatible with serious training. This does not mean, however, that serious athletes are immune from the various factors that lead to severe, compulsive overindulgence within a small subset of all drinkers. Whether an athlete or not, complete abstention from alcohol is, obviously, a perfectly reasonable choice. In fact, the vast majority of younger athletes (those under the age of, say, 22) are best to wait till they have acquired fully developed adult impulse-control before adding beer to their daily regime. The consumption of beer by athletes requires a highly disciplined drinking style-- rather akin to the athletic lifestyle in general. As a rule, if you know you can't drink one unit of beer without wanting 3 more, best not to crack the first one. Besides, if you are drinking the quality of beverage I've been referring to throughout, you will experience relatively diminished pleasure after your first hit. Best, therefore, to save that second one for another day, when it will deliver it fullest flavour punch. The adage that guides my beer love is, increasingly: Drink less to enjoy more!

If you're interested in hearing of more of my thoughts on beer and running-- including the best styles for your training seasons; the top brews on offer in your particular area; profiles of top runners and their beers; and, the best running/beer tourism destinations, replete with bar/brewery and running route particulars-- visit my new blog, Hop Along. Cheers!



