\chapter{\dots I Die a Little: Running and Intimations of Mortality}
\chaptermark{I Die a Little}
\textit{Thursday, 10 January 2019}
\bigskip

It is early in the new century and the scene is the driveway of an old and dear running friend. It is post-run and a spring sun is warming us pleasantly. We are both newly 40 and fathers, yet we are still firmly in the grip of the sport that has possessed us since our early teens. Conversation turns to our declining physical capacities-- a process still subtle enough that we can mistake its signs for a bad month or year, if we want to engage in a little denial-- and I offer that the changes occurring deep in our cells, and only really perceptible to us in the form of a few lost seconds per kilometer at harder efforts, are actually intimations of death itself. In fact, I say, we are being slowed down ever so slightly by the same processes that will culminate in our total demise, if we are so lucky as to croak of "natural causes". And, as runners, we get to experience these intimations of mortality in measurable form because we are runners, and we still occasionally race and train to failure. Basking in our post-run hormonal glow, and feeling as vital as ever, we register this truth with resignation and a blunt, black humor whose exchange has been one of the foundations of our decades-long friendship. We're dying, sure, but we're not dead yet!

As Dylan (the other one, Bob) once sang, those who aren't busy being born are busy dying. In his youthful cockiness, he made it sound like getting busy being born rather than dying was somehow a choice, but he was not wrong in the gist. Now a decade and a half down the road from that aforementioned reflection on age-graded performance loss as death looking up your current address, I have noticed-- and remarked to many of my younger athletes-- how the injuries that bedevil me now, at 55, have their roots in physical quirks that I can remember becoming aware of from my first steps as a a serious athlete-- when I was awash in the stuff of growth and development and getting faster almost by the week. A funny feeling in the lower back here, a tight calf there-- there is nothing really new now but the persistence of the problems with which I now grapple. What used to take a week or two to resolve (and that would resolve completely, for a time) now turns into months and even years of limbo between injury and pain-free mobility, regardless of interventions and regardless of how much or little I run. My advice to the young, of course, is to attack weaknesses as soon as they identify themselves with whatever means you have available, because they will be always with you in some form or another, and will likely be the things that eventually put an end to your ability to improve. What I don't tell them (because youth is fragile, innocent, and because they will learn it in any case, the way we, all of us, must) is that their weak points, as they become increasingly difficult to manage, will soon whisper to them that they are, like all living things, bending towards the grave. I do not tell them that, if they are one day so inclined, they will be able to use their vigorous athletic lives, and their inevitable struggles with injuries and steady performance decline, as an instrument of some precision in tracking their earthly demise, and at an age when their non-athletic friends are still able to delude themselves that they "still have it" physically in the face of obvious signs that they can't possibly.

The question is, as runners, are we well served or burdened by pursuing an activity that, whether we like it or not, registers our physical decline in such fine detail-- one that, in effect, reminds us of our mortality on a monthly or yearly basis, like a tap on the shoulder from the reaper himself. If you're not measuring your physical capacities on a daily basis, I would imagine that it's possible to go years without any real sense of physical decline (provided one is not truly sick). A few extra pounds here and there can be attributed to temporary and reversible "lifestyle choices". Even a receding hairline can be easily filed away as "genetics". After all, people can get fat or go bald at almost any age. And, of course, there is simply forgetting, which happens to us all as we age. We will imagine we feel the same as we always have-- that we are essentially the same person psychologically and socially-- because we've forgotten how reality crackled on our nerve endings at earlier stages of our lives, and are only reminded that we have changed-- aged-- when we see nothing but absurdity and folly in the things young people do-- things we can dimly remember reveling in ourselves. It is easy, after all, to deny that time has been messing with us in incremental but unmistakable ways. But is it depressing to be reminded of one's mortality as often as serious runners are, if they persist with the activity into later decades, as so many people now do?

Short of living a parallel life in which we don't run and never have, we can't answer this question with complete confidence. The fact that some runners move on and embrace other serious fitness activities in middle age, however, is perhaps evidence that some of us can't bear to be reminded with such empirical precision of our physical decline from the pinnacle of youth. Better for some to try a new sport in which "personal bests" can still be attained, and which does not light up our old injuries quite so vividly. That others quit serious physical activity all together after their serious competitive days are over is perhaps another sign that having one's nose rubbed in one's steady physical decline may be too much for some to bear. Best for some, apparently, to cruise comfortably along until encountering one or two big reminders of death's imminence all at once. Though it's not easy to know what goes on in the minds of older runners who have been racing and training for decades, in spite of steadily declining performances, my experience in knowing and coaching lifelong runners (and in being one myself) is that running's regular intimations of mortality do not necessarily weigh heavily on veterans of the sport-- and not because we are in some kind of denial of the obvious. I happen to think that training and competing well past one's prime eventually makes one's mortality easier, not harder, to accept, for the simple reason that regular practice makes everything easier. Runners who have stuck with it from their primes into old age know that they have been dying by degrees since their late 30s; they have, in effect, been practicing coping with true morbidity from an age when non-athletes can still credibly convince themselves that they have achieved a kind of comfortable stasis between youth and old age.

As I age in the sport, I often think of the great Ed Whitlock, who left us a little more than a year ago. Ed was a champion at life, of course, but he also turned out to be a maestro of death. I do not speak from direct knowledge, but it seemed to me at the time of Ed's passing that, being more attuned to his body than probably anyone who has ever lived, he must have known that he was very sick, and that he was dying, even as he continued to train for his last record-breaking marathon performance. That he had made a trip to his birthplace in England at this time also suggests that he knew he was not at all well. Some of us-- runners and non-runners alike-- are able to attain a grace in death that others can't. And, like heroism in a dire emergency, it can be difficult to predict who will do what with the certain knowledge that their time is up. But I would like to believe that Ed Whitlock, while having every reason to believe that he would rival Methuselah with his longevity, was able to die with the apparent equanimity that he did, and at not much older than the average Canadian male, because his running had taught him daily that we are all steadily succumbing to entropy even when we feel most full of life. And rather than frightening him into inactivity, it seemed to compel him to embrace what he loved until the very end-- sweet, not bitter.




